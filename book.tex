\documentclass[letterpaper]{book}
\usepackage{lmodern}
\usepackage{hyperref}
\usepackage[shorthands=off,greek,main=english]{babel}
\newenvironment{greek}[2][]{\begin{otherlanguage}{greek}}{\end{otherlanguage}}

\title{How We Think}
\author{John Dewey}
\date{1901}

\begin{document}
\frontmatter
\maketitle
\pagestyle{plain}

\chapter{Preface}

Our schools are troubled with a multiplication of studies, each in turn
having its own multiplication of materials and principles. Our teachers
find their tasks made heavier in that they have come to deal with pupils
individually and not merely in mass. Unless these steps in advance are
to end in distraction, some clew of unity, some principle that makes for
simplification, must be found. This book represents the conviction that
the needed steadying and centralizing factor is found in adopting as the
end of endeavor that attitude of mind, that habit of thought, which we
call scientific. This scientific attitude of mind might, conceivably, be
quite irrelevant to teaching children and youth. But this book also
represents the conviction that such is not the case; that the native and
unspoiled attitude of childhood, marked by ardent curiosity, fertile
imagination, and love of experimental inquiry, is near, very near, to
the attitude of the scientific mind. If these pages assist any to
appreciate this kinship and to consider seriously how its recognition in
educational practice would make for individual happiness and the
reduction of social waste, the book will amply have served its purpose.

It is hardly necessary to enumerate the authors to whom I am indebted.
My fundamental indebtedness is to my wife, by whom the ideas of this
book were inspired, and through whose work in connection with the
Laboratory School, existing in Chicago between 1896 and 1903, the ideas
attained such concreteness as comes from embodiment and testing in
practice. It is a pleasure, also, to acknowledge indebtedness to the
intelligence and sympathy of those who coöperated as teachers and
supervisors in the conduct of that school, and especially to Mrs. Ella
Flagg Young, then a colleague in the University, and now Superintendent
of the Schools of Chicago.

{New York City}, December, 1909.

\setcounter{tocdepth}{2}
\tableofcontents

\mainmatter

\part{The Problem of Training Thought}

\chapter{What is Thought?}

\section{Varied Senses of the Term}

\marginpar{Four senses of thought, from the wider to the limited}

No words are oftener on our lips than \emph{thinking} and
\emph{thought}. So profuse and varied, indeed, is our use of these words
that it is not easy to define just what we mean by them. The aim of this
chapter is to find a single consistent meaning. Assistance may be had by
considering some typical ways in which the terms are employed. In the
first place \emph{thought} is used broadly, not to say loosely.
Everything that comes to mind, that "goes through our heads," is called
a thought. To think of a thing is just to be conscious of it in any way
whatsoever. Second, the term is restricted by excluding whatever is
directly presented; we think (or think of) only such things as we do not
directly see, hear, smell, or taste. Then, third, the meaning is further
limited to beliefs that rest upon some kind of evidence or testimony. Of
this third type, two kinds---or, rather, two degrees---must be
discriminated. In some cases, a belief is accepted with slight or almost
no attempt to state the grounds that support it. In other cases, the
ground or basis for a belief is deliberately sought and
its
adequacy to support the belief examined. This process is called
reflective thought; it alone is truly educative in value, and it forms,
accordingly, the principal subject of this volume. We shall now briefly
describe each of the four senses.

\marginpar{Chance and idle thinking}

I. In its loosest sense, thinking signifies everything that, as we say,
is "in our heads" or that "goes through our minds." He who offers "a
penny for your thoughts" does not expect to drive any great bargain. In
calling the objects of his demand \emph{thoughts}, he does not intend to
ascribe to them dignity, consecutiveness, or truth. Any idle fancy,
trivial recollection, or flitting impression will satisfy his demand.
Daydreaming, building of castles in the air, that loose flux of casual
and disconnected material that floats through our minds in relaxed
moments are, in this random sense, \emph{thinking}. More of our waking
life than we should care to admit, even to ourselves, is likely to be
whiled away in this inconsequential trifling with idle fancy and
unsubstantial hope.

\marginpar{Reflective thought is consecutive, not merely a sequence}

In this sense, silly folk and dullards \emph{think}. The story is told
of a man in slight repute for intelligence, who, desiring to be chosen
selectman in his New England town, addressed a knot of neighbors in this
wise: "I hear you don't believe I know enough to hold office. I wish you
to understand that I am thinking about something or other most of the
time." Now reflective thought is like this random coursing of things
through the mind in that it consists of a succession of things thought
of; but it is unlike, in that the mere chance occurrence of any chance
"something or other" in an irregular sequence does not suffice.
Reflection involves not simply a sequence of ideas, but a
\emph{con}sequence---a consecutive ordering in such a way
that
each determines the next as its proper outcome, while each in turn leans
back on its predecessors. The successive portions of the reflective
thought grow out of one another and support one another; they do not
come and go in a medley. Each phase is a step from something to
something---technically speaking, it is a term of thought. Each term
leaves a deposit which is utilized in the next term. The stream or flow
becomes a train, chain, or thread.

\marginpar{The restriction of \emph{thinking} to what goes beyond direct
observation}

II. Even when thinking is used in a broad sense, it is usually
restricted to matters not directly perceived: to what we do not see,
smell, hear, or touch. We ask the man telling a story if he saw a
certain incident happen, and his reply may be, "No, I only thought of
it." A note of invention, as distinct from faithful record of
observation, is present. Most important in this class are successions of
imaginative incidents and episodes which, having a certain coherence,
hanging together on a continuous thread, lie between kaleidoscopic
flights of fancy and considerations deliberately employed to establish a
conclusion. The imaginative stories poured forth by children possess all
degrees of internal congruity; some are disjointed, some are
articulated. When connected, they simulate reflective thought; indeed,
they usually occur in minds of logical capacity. These imaginative
enterprises often precede thinking of the close-knit type and prepare
the way for it. \marginpar{Reflective thought aims, however, at belief}
But \emph{they do not aim at knowledge, at belief about
facts or in truths}; and thereby they are marked off from reflective
thought even when they most resemble it. Those who express such thoughts
do not expect credence, but rather credit for a well-constructed plot or
a well-arranged climax. They produce good stories, not---unless by
chance---knowledge.
Such thoughts are an efflorescence of feeling; the enhancement of a mood
or sentiment is their aim; congruity of emotion, their binding tie.

\marginpar{Thought induces belief in two ways}

III. In its next sense, thought denotes belief resting upon some basis,
that is, real or supposed knowledge going beyond what is directly
present. It is marked by \emph{acceptance or rejection of something as
reasonably probable or improbable}. This phase of thought, however,
includes two such distinct types of belief that, even though their
difference is strictly one of degree, not of kind, it becomes
practically important to consider them separately. Some beliefs are
accepted when their grounds have not themselves been considered, others
are accepted because their grounds have been examined.

When we say, "Men used to think the world was flat," or, "I thought you
went by the house," we express belief: something is accepted, held to,
acquiesced in, or affirmed. But such thoughts may mean a supposition
accepted without reference to its real grounds. These may be adequate,
they may not; but their value with reference to the support they afford
the belief has not been considered.

Such thoughts grow up unconsciously and without reference to the
attainment of correct belief. They are picked up---we know not how. From
obscure sources and by unnoticed channels they insinuate themselves into
acceptance and become unconsciously a part of our mental furniture.
Tradition, instruction, imitation---all of which depend upon authority
in some form, or appeal to our own advantage, or fall in with a strong
passion---are responsible for them. Such thoughts are prejudices, that
is, prejudgments, not judgments proper that rest upon a survey of
evidence.\footnote{This mode of thinking in its contrast with thoughtful inquiry receives
special notice in the next chapter.}

\marginpar{Thinking in its best sense is that which considers the basis and
consequences of beliefs}

IV. Thoughts that result in belief have an importance attached to them
which leads to reflective thought, to conscious inquiry into the nature,
conditions, and bearings of the belief. To \emph{think} of whales and
camels in the clouds is to entertain ourselves with fancies, terminable
at our pleasure, which do not lead to any belief in particular. But to
think of the world as flat is to ascribe a quality to a real thing as
its real property. This conclusion denotes a connection among things and
hence is not, like imaginative thought, plastic to our mood. Belief in
the world's flatness commits him who holds it to thinking in certain
specific ways of other objects, such as the heavenly bodies, antipodes,
the possibility of navigation. It prescribes to him actions in
accordance with his conception of these objects.

The consequences of a belief upon other beliefs and upon behavior may be
so important, then, that men are forced to consider the grounds or
reasons of their belief and its logical consequences. This means
reflective thought---thought in its eulogistic and emphatic sense.

\marginpar{Reflective thought defined}

Men \emph{thought} the world was flat until Columbus \emph{thought} it
to be round. The earlier thought was a belief held because men had not
the energy or the courage to question what those about them accepted and
taught, especially as it was suggested and seemingly confirmed by
obvious sensible facts. The thought of Columbus was a \emph{reasoned
conclusion}. It marked the close of study into facts, of scrutiny and
revision of evidence, of working out the implications of various
hypotheses, and
of
comparing these theoretical results with one another and with known
facts. Because Columbus did not accept unhesitatingly the current
traditional theory, because he doubted and inquired, he arrived at his
thought. Skeptical of what, from long habit, seemed most certain, and
credulous of what seemed impossible, he went on thinking until he could
produce evidence for both his confidence and his disbelief. Even if his
conclusion had finally turned out wrong, it would have been a different
sort of belief from those it antagonized, because it was reached by a
different method. \emph{Active, persistent, and careful consideration of
any belief or supposed form of knowledge in the light of the grounds
that support it, and the further conclusions to which it tends},
constitutes reflective thought. Any one of the first three kinds of
thought may elicit this type; but once begun, it is a conscious and
voluntary effort to establish belief upon a firm basis of reasons.

\section{The Central Factor in Thinking}

\marginpar{There is a common element in all types of thought:}

There are, however, no sharp lines of demarcation between the various
operations just outlined. The problem of attaining correct habits of
reflection would be much easier than it is, did not the different modes
of thinking blend insensibly into one another. So far, we have
considered rather extreme instances of each kind in order to get the
field clearly before us. Let us now reverse this operation; let us
consider a rudimentary case of thinking, lying between careful
examination of evidence and a mere irresponsible stream of fancies. A
man is walking on a warm day. The sky was clear the last time he
observed it; but presently he notes, while occupied primarily with other
things, that the air is cooler. It occurs to him that it is probably
going
to
rain; looking up, he sees a dark cloud between him and the sun, and he
then quickens his steps. What, if anything, in such a situation can be
called thought? Neither the act of walking nor the noting of the cold is
a thought. Walking is one direction of activity; looking and noting are
other modes of activity. The likelihood that it will rain is, however,
something \emph{suggested}. The pedestrian \emph{feels} the cold; he
\emph{thinks of} clouds and a coming shower.

\marginpar{\emph{viz.} suggestion of something not observed}

\marginpar{But reflection involves also the relation of \emph{signifying}}

So far there is the same sort of situation as when one looking at a
cloud is reminded of a human figure and face. Thinking in both of these
cases (the cases of belief and of fancy) involves a noted or perceived
fact, followed by something else which is not observed but which is
brought to mind, suggested by the thing seen. One reminds us, as we say,
of the other. Side by side, however, with this factor of agreement in
the two cases of suggestion is a factor of marked disagreement. We do
not \emph{believe} in the face suggested by the cloud; we do not
consider at all the probability of its being a fact. There is no
\emph{reflective} thought. The danger of rain, on the contrary, presents
itself to us as a genuine possibility---as a possible fact of the same
nature as the observed coolness. Put differently, we do not regard the
cloud as meaning or indicating a face, but merely as suggesting it,
while we do consider that the coolness may mean rain. In the first case,
seeing an object, we just happen, as we say, to think of something else;
in the second, we consider the \emph{possibility and nature of the
connection between the object seen and the object suggested}. The seen
thing is regarded as in some way \emph{the ground or basis of belief} in
the suggested thing; it possesses the quality of
\emph{evidence}.

\marginpar{Various synonymous expressions for the function of signifying}

This function by which one thing signifies or indicates another, and
thereby leads us to consider how far one may be regarded as warrant for
belief in the other, is, then, the central factor in all reflective or
distinctively intellectual thinking. By calling up various situations to
which such terms as \emph{signifies} and \emph{indicates} apply, the
student will best realize for himself the actual facts denoted by the
words \emph{reflective thought}. Synonyms for these terms are: points
to, tells of, betokens, prognosticates, represents, stands for,
implies.\footnote{\emph{Implies} is more often used when a principle or general truth
brings about belief in some other truth; the other phrases are more
frequently used to denote the cases in which one fact or event leads us
to believe in something else.}
We also say one thing portends another; is ominous of another, or a
symptom of it, or a key to it, or (if the connection is quite obscure)
that it gives a hint, clue, or intimation.

\marginpar{Reflection and belief on evidence}

Reflection thus implies that something is believed in (or disbelieved
in), not on its own direct account, but through something else which
stands as witness, evidence, proof, voucher, warrant; that is, as
\emph{ground of belief}. At one time, rain is actually felt or directly
experienced; at another time, we infer that it has rained from the looks
of the grass and trees, or that it is going to rain because of the
condition of the air or the state of the barometer. At one time, we see
a man (or suppose we do) without any intermediary fact; at another time,
we are not quite sure what we see, and hunt for accompanying facts that
will serve as signs, indications, tokens of what is to be believed.

Thinking, for the purposes of this inquiry, is defined accordingly as
\emph{that operation in which present facts suggest other facts (or
truths) in such a way as to induce
belief
in the latter upon the ground or warrant of the former}. We do not put
beliefs that rest simply on inference on the surest level of assurance.
To say "I think so" implies that I do not as yet \emph{know} so. The
inferential belief may later be confirmed and come to stand as sure, but
in itself it always has a certain element of supposition.

\section{Elements in Reflective Thinking}

So much for the description of the more external and obvious aspects of
the fact called \emph{thinking}. Further consideration at once reveals
certain subprocesses which are involved in every reflective operation.
These are: (\emph{a}) a state of perplexity, hesitation, doubt; and
(\emph{b}) an act of search or investigation directed toward bringing to
light further facts which serve to corroborate or to nullify the
suggested belief.

\marginpar{The importance of uncertainty}

(\emph{a}) In our illustration, the shock of coolness generated
confusion and suspended belief, at least momentarily. Because it was
unexpected, it was a shock or an interruption needing to be accounted
for, identified, or placed. To say that the abrupt occurrence of the
change of temperature constitutes a problem may sound forced and
artificial; but if we are willing to extend the meaning of the word
\emph{problem} to whatever---no matter how slight and commonplace in
character---perplexes and challenges the mind so that it makes belief at
all uncertain, there is a genuine problem or question involved in this
experience of sudden change.

\marginpar{and of inquiry in order to test}

(\emph{b}) The turning of the head, the lifting of the eyes, the
scanning of the heavens, are activities adapted to bring to recognition
facts that will answer the question presented by the sudden coolness.
The facts as
they
first presented themselves were perplexing; they suggested, however,
clouds. The act of looking was an act to discover if this suggested
explanation held good. It may again seem forced to speak of this
looking, almost automatic, as an act of research or inquiry. But once
more, if we are willing to generalize our conceptions of our mental
operations to include the trivial and ordinary as well as the technical
and recondite, there is no good reason for refusing to give such a title
to the act of looking. The purport of this act of inquiry is to confirm
or to refute the suggested belief. New facts are brought to perception,
which either corroborate the idea that a change of weather is imminent,
or negate it.

\marginpar{Finding one's way an illustration of reflection}

Another instance, commonplace also, yet not quite so trivial, may
enforce this lesson. A man traveling in an unfamiliar region comes to a
branching of the roads. Having no sure knowledge to fall back upon, he
is brought to a standstill of hesitation and suspense. Which road is
right? And how shall perplexity be resolved? There are but two
alternatives: he must either blindly and arbitrarily take his course,
trusting to luck for the outcome, or he must discover grounds for the
conclusion that a given road is right. Any attempt to decide the matter
by thinking will involve inquiry into other facts, whether brought out
by memory or by further observation, or by both. The perplexed wayfarer
must carefully scrutinize what is before him and he must cudgel his
memory. He looks for evidence that will support belief in favor of
either of the roads---for evidence that will weight down one suggestion.
He may climb a tree; he may go first in this direction, then in that,
looking, in either case, for signs,
clues,
indications. He wants something in the nature of a signboard or a map,
and \emph{his reflection is aimed at the discovery of facts that will
serve this purpose}.

\marginpar{Possible, yet incompatible, suggestions}

The above illustration may be generalized. Thinking begins in what may
fairly enough be called a \emph{forked-road} situation, a situation
which is ambiguous, which presents a dilemma, which proposes
alternatives. As long as our activity glides smoothly along from one
thing to another, or as long as we permit our imagination to entertain
fancies at pleasure, there is no call for reflection. Difficulty or
obstruction in the way of reaching a belief brings us, however, to a
pause. In the suspense of uncertainty, we metaphorically climb a tree;
we try to find some standpoint from which we may survey additional facts
and, getting a more commanding view of the situation, may decide how the
facts stand related to one another.

\marginpar{Regulation of thinking by its purpose}

\emph{Demand for the solution of a perplexity is the steadying and
guiding factor in the entire process of reflection.} Where there is no
question of a problem to be solved or a difficulty to be surmounted, the
course of suggestions flows on at random; we have the first type of
thought described. If the stream of suggestions is controlled simply by
their emotional congruity, their fitting agreeably into a single picture
or story, we have the second type. But a question to be answered, an
ambiguity to be resolved, sets up an end and holds the current of ideas
to a definite channel. Every suggested conclusion is tested by its
reference to this regulating end, by its pertinence to the problem in
hand. This need of straightening out a perplexity also controls the kind
of inquiry undertaken. A traveler whose end is the most beautiful path
will look for other considerations
and
will test suggestions occurring to him on another principle than if he
wishes to discover the way to a given city. \emph{The problem fixes the
end of thought} and \emph{the end controls the process of thinking}.

\section{Summary}

\marginpar{Origin and stimulus}

We may recapitulate by saying that the origin of thinking is some
perplexity, confusion, or doubt. Thinking is not a case of spontaneous
combustion; it does not occur just on "general principles." There is
something specific which occasions and evokes it. General appeals to a
child (or to a grown-up) to think, irrespective of the existence in his
own experience of some difficulty that troubles him and disturbs his
equilibrium, are as futile as advice to lift himself by his boot-straps.

\marginpar{Suggestions and past experience}

Given a difficulty, the next step is suggestion of some way out---the
formation of some tentative plan or project, the entertaining of some
theory which will account for the peculiarities in question, the
consideration of some solution for the problem. The data at hand cannot
supply the solution; they can only suggest it. What, then, are the
sources of the suggestion? Clearly past experience and prior knowledge.
If the person has had some acquaintance with similar situations, if he
has dealt with material of the same sort before, suggestions more or
less apt and helpful are likely to arise. But unless there has been
experience in some degree analogous, which may now be represented in
imagination, confusion remains mere confusion. There is nothing upon
which to draw in order to clarify it. Even when a child (or a grown-up)
has a problem, to urge him to think when he has no prior experiences
involving some of the same conditions, is wholly
futile.

\marginpar{Exploration and testing}

If the suggestion that occurs is at once accepted, we have uncritical
thinking, the minimum of reflection. To turn the thing over in mind, to
reflect, means to hunt for additional evidence, for new data, that will
develop the suggestion, and will either, as we say, bear it out or else
make obvious its absurdity and irrelevance. Given a genuine difficulty
and a reasonable amount of analogous experience to draw upon, the
difference, \emph{par excellence}, between good and bad thinking is
found at this point. The easiest way is to accept any suggestion that
seems plausible and thereby bring to an end the condition of mental
uneasiness. Reflective thinking is always more or less troublesome
because it involves overcoming the inertia that inclines one to accept
suggestions at their face value; it involves willingness to endure a
condition of mental unrest and disturbance. Reflective thinking, in
short, means judgment suspended during further inquiry; and suspense is
likely to be somewhat painful. As we shall see later, the most important
factor in the training of good mental habits consists in acquiring the
attitude of suspended conclusion, and in mastering the various methods
of searching for new materials to corroborate or to refute the first
suggestions that occur. To maintain the state of doubt and to carry on
systematic and protracted inquiry---these are the essentials of
thinking.

\chapter{The Need for Training Thought}

\marginpar{Man the animal that thinks}

To expatiate upon the importance of thought would be absurd. The
traditional definition of man as "the thinking animal" fixes thought as
the essential difference between man and the brutes,---surely an
important matter. More relevant to our purpose is the question how
thought is important, for an answer to this question will throw light
upon the kind of training thought requires if it is to subserve its end.

\section{The Values of Thought}

\marginpar{The possibility of deliberate and intentional activity}

I. Thought affords the sole method of escape from purely impulsive or
purely routine action. A being without capacity for thought is moved
only by instincts and appetites, as these are called forth by outward
conditions and by the inner state of the organism. A being thus moved
is, as it were, pushed from behind. This is what we mean by the blind
nature of brute actions. The agent does not see or foresee the end for
which he is acting, nor the results produced by his behaving in one way
rather than in another. He does not "know what he is about." Where there
is thought, things present act as signs or tokens of things not yet
experienced. A thinking being can, accordingly, \emph{act on the basis
of the absent and the future}. Instead of being pushed into a mode of
action by the sheer urgency of forces,
whether
instincts or habits, of which he is not aware, a reflective agent is
drawn (to some extent at least) to action by some remoter object of
which he is indirectly aware.

\marginpar{Natural events come to be a language}

An animal without thought may go into its hole when rain threatens,
because of some immediate stimulus to its organism. A thinking agent
will perceive that certain given facts are probable signs of a future
rain, and will take steps in the light of this anticipated future. To
plant seeds, to cultivate the soil, to harvest grain, are intentional
acts, possible only to a being who has learned to subordinate the
immediately felt elements of an experience to those values which these
hint at and prophesy. Philosophers have made much of the phrases "book
of nature," "language of nature." Well, it is in virtue of the capacity
of thought that given things are significant of absent things, and that
nature speaks a language which may be interpreted. To a being who
thinks, things are records of their past, as fossils tell of the prior
history of the earth, and are prophetic of their future, as from the
present positions of heavenly bodies remote eclipses are foretold.
Shakespeare's "tongues in trees, books in the running brooks," expresses
literally enough the power superadded to existences when they appeal to
a thinking being. Upon the function of signification depend all
foresight, all intelligent planning, deliberation, and calculation.

\marginpar{The possibility of systematized foresight}

II. By thought man also develops and arranges artificial signs to remind
him in advance of consequences, and of ways of securing and avoiding
them. As the trait just mentioned makes the difference between savage
man and brute, so this trait makes the difference between civilized man
and savage. A savage who has been shipwrecked in a river may note
certain things
which
serve him as signs of danger in the future. But civilized man
deliberately \emph{makes} such signs; he sets up in advance of wreckage
warning buoys, and builds lighthouses where he sees signs that such
events may occur. A savage reads weather signs with great expertness;
civilized man institutes a weather service by which signs are
artificially secured and information is distributed in advance of the
appearance of any signs that could be detected without special methods.
A savage finds his way skillfully through a wilderness by reading
certain obscure indications; civilized man builds a highway which shows
the road to all. The savage learns to detect the signs of fire and
thereby to invent methods of producing flame; civilized man invents
permanent conditions for producing light and heat whenever they are
needed. The very essence of civilized culture is that we deliberately
erect monuments and memorials, lest we forget; and deliberately
institute, in advance of the happening of various contingencies and
emergencies of life, devices for detecting their approach and
registering their nature, for warding off what is unfavorable, or at
least for protecting ourselves from its full impact and for making more
secure and extensive what is favorable. All forms of artificial
apparatus are intentionally designed modifications of natural things in
order that they may serve better than in their natural estate to
indicate the hidden, the absent, and the remote.

\marginpar{The possibility of objects rich in quality}

III. Finally, thought confers upon physical events and objects a very
different status and value from that which they possess to a being that
does not reflect. These words are mere scratches, curious variations of
light and shade, to one to whom they are not linguistic signs. To him
for whom they are signs of other
things,
each has a definite individuality of its own, according to the meaning
that it is used to convey. \emph{Exactly the same holds of natural
objects.} A chair is a different object to a being to whom it
consciously suggests an opportunity for sitting down, repose, or
sociable converse, from what it is to one to whom it presents itself
merely as a thing to be smelled, or gnawed, or jumped over; a stone is
different to one who knows something of its past history and its future
use from what it is to one who only feels it directly through his
senses. It is only by courtesy, indeed, that we can say that an
unthinking animal experiences an \emph{object} at all---so largely is
anything that presents itself to us as an object made up by the
qualities it possesses as a sign of other things.

\marginpar{The nature of the objects an animal perceives}

An English logician (Mr.\ Venn) has remarked that it may be questioned
whether a dog \emph{sees} a rainbow any more than he apprehends the
political constitution of the country in which he lives. The same
principle applies to the kennel in which he sleeps and the meat that he
eats. When he is sleepy, he goes to the kennel; when he is hungry, he is
excited by the smell and color of meat; beyond this, in what sense does
he see an \emph{object}? Certainly he does not see a house---\emph{i.e.}
a thing with all the properties and relations of a permanent residence,
\emph{unless} he is capable of making what is present a uniform sign of
what is absent---unless he is capable of thought. Nor does he see what
he eats \emph{as} meat unless it suggests the absent properties by
virtue of which it is a certain joint of some animal, and is known to
afford nourishment. Just what is left of an \emph{object} stripped of
all such qualities of meaning, we cannot well say; but we can be sure
that the object is then a very different sort of thing from the objects
that we perceive.
There
is moreover no particular limit to the possibilities of growth in the
fusion of a thing as it is to sense and as it is to thought, or as a
sign of other things. The child today soon regards as constituent parts
of objects qualities that once it required the intelligence of a
Copernicus or a Newton to apprehend.

\marginpar{Mill on the business of life and the occupation of mind}

These various values of the power of thought may be summed up in the
following quotation from John Stuart Mill. "To draw inferences," he
says, "has been said to be the great business of life. Every one has
daily, hourly, and momentary need of ascertaining facts which he has not
directly observed: not from any general purpose of adding to his stock
of knowledge, but because the facts themselves are of importance to his
interests or to his occupations. The business of the magistrate, of the
military commander, of the navigator, of the physician, of the
agriculturist, \emph{is merely to judge of evidence and to act
accordingly}.... As they do this well or ill, so they discharge well or
ill the duties of their several callings. \emph{It is the only
occupation in which the mind never ceases to be
engaged.}"\footnote{Mill, \emph{System of Logic}, Introduction, § 5.}

\section{Importance of Direction in order to Realize these Values}

\marginpar{Thinking goes astray}

What a person has not only daily and hourly, but momentary need of
performing, is not a technical and abstruse matter; nor, on the other
hand, is it trivial and negligible. Such a function must be congenial to
the mind, and must be performed, in an unspoiled mind, upon every
fitting occasion. Just because, however, it is an operation of drawing
inferences, of basing conclusions upon evidence, of reaching belief
\emph{indirectly}, it
is
an operation that may go wrong as well as right, and hence is one that
needs safeguarding and training. The greater its importance the greater
are the evils when it is ill-exercised.

\marginpar{Ideas are our rulers---for better or for worse}

An earlier writer than Mill, John Locke (1632-1704), brings out the
importance of thought for life and the need of training so that its best
and not its worst possibilities will be realized, in the following
words: "No man ever sets himself about anything but upon some view or
other, which serves him for a reason for what he does; and whatsoever
faculties he employs, the understanding with such light as it has, well
or ill informed, constantly leads; and by that light, true or false, all
his operative powers are directed.... Temples have their sacred images,
and we see what influence they have always had over a great part of
mankind. But in truth the ideas and images in men's minds are the
invisible powers that constantly govern them, and to these they all,
universally, pay a ready submission. It is therefore of the highest
concernment that great care should be taken of the understanding, to
conduct it aright in the search of knowledge and in the judgments it
makes."\footnote{Locke, \emph{Of the Conduct of the Understanding}, first paragraph.}
If upon thought hang all deliberate activities and the uses we make of
all our other powers, Locke's assertion that it is of the highest
concernment that care should be taken of its conduct is a moderate
statement. While the power of thought frees us from servile subjection
to instinct, appetite, and routine, it also brings with it the occasion
and possibility of error and mistake. In elevating us above the brute,
it opens to us the possibility of failures to which the animal, limited
to instinct, cannot sink.

\section{Tendencies Needing Constant Regulation}

\marginpar{Physical and social sanctions of correct thinking}

Up to a certain point, the ordinary conditions of life, natural and
social, provide the conditions requisite for regulating the operations
of inference. The necessities of life enforce a fundamental and
persistent discipline for which the most cunningly devised artifices
would be ineffective substitutes. The burnt child dreads the fire; the
painful consequence emphasizes the need of correct inference much more
than would learned discourse on the properties of heat. Social
conditions also put a premium on correct inferring in matters where
action based on valid thought is socially important. These sanctions of
proper thinking may affect life itself, or at least a life reasonably
free from perpetual discomfort. The signs of enemies, of shelter, of
food, of the main social conditions, have to be correctly apprehended.

\marginpar{The serious limitations of such sanctions}

But this disciplinary training, efficacious as it is within certain
limits, does not carry us beyond a restricted boundary. Logical
attainment in one direction is no bar to extravagant conclusions in
another. A savage expert in judging signs of the movements and location
of animals that he hunts, will accept and gravely narrate the most
preposterous yarns concerning the origin of their habits and structures.
When there is no directly appreciable reaction of the inference upon the
security and prosperity of life, there are no natural checks to the
acceptance of wrong beliefs. Conclusions may be generated by a modicum
of fact merely because the suggestions are vivid and interesting; a
large accumulation of data may fail to suggest a proper conclusion
because existing customs are averse to entertaining it. Independent of
training, there is a "primitive
credulity"
which tends to make no distinction between what a trained mind calls
fancy and that which it calls a reasonable conclusion. The face in the
clouds is believed in as some sort of fact, merely because it is
forcibly suggested. Natural intelligence is no barrier to the
propagation of error, nor large but untrained experience to the
accumulation of fixed false beliefs. Errors may support one another
mutually and weave an ever larger and firmer fabric of misconception.
Dreams, the positions of stars, the lines of the hand, may be regarded
as valuable signs, and the fall of cards as an inevitable omen, while
natural events of the most crucial significance go disregarded. Beliefs
in portents of various kinds, now mere nook and cranny superstitions,
were once universal. A long discipline in exact science was required for
their conquest.

\marginpar{Superstition as natural a result as science}

In the mere function of suggestion, there is no difference between the
power of a column of mercury to portend rain, and that of the entrails
of an animal or the flight of birds to foretell the fortunes of war. For
all anybody can tell in advance, the spilling of salt is as likely to
import bad luck as the bite of a mosquito to import malaria. Only
systematic regulation of the conditions under which observations are
made and severe discipline of the habits of entertaining suggestions can
secure a decision that one type of belief is vicious and the other
sound. The substitution of scientific for superstitious habits of
inference has not been brought about by any improvement in the acuteness
of the senses or in the natural workings of the function of suggestion.
It is the result of regulation \emph{of the conditions} under which
observation and inference take
place.

\marginpar{General causes of bad thinking: Bacon's "idols"}

It is instructive to note some of the attempts that have been made to
classify the main sources of error in reaching beliefs. Francis Bacon,
for example, at the beginnings of modern scientific inquiry, enumerated
four such classes, under the somewhat fantastic title of "idols" (Gr.
\textgreek{ειδωλα}, images), spectral forms that allure the mind into
false paths. These he called the idols, or phantoms, of the (\emph{a})
tribe, (\emph{b}) the marketplace, (\emph{c}) the cave or den, and
(\emph{d}) the theater; or, less metaphorically, (\emph{a}) standing
erroneous methods (or at least temptations to error) that have their
roots in human nature generally; (\emph{b}) those that come from
intercourse and language; (\emph{c}) those that are due to causes
peculiar to a specific individual; and finally, (\emph{d}) those that
have their sources in the fashion or general current of a period.
Classifying these causes of fallacious belief somewhat differently, we
may say that two are intrinsic and two are extrinsic. Of the intrinsic,
one is common to all men alike (such as the universal tendency to notice
instances that corroborate a favorite belief more readily than those
that contradict it), while the other resides in the specific temperament
and habits of the given individual. Of the extrinsic, one proceeds from
generic social conditions---like the tendency to suppose that there is a
fact wherever there is a word, and no fact where there is no linguistic
term---while the other proceeds from local and temporary social
currents.

\marginpar{Locke on the influence of}

Locke's method of dealing with typical forms of wrong belief is less
formal and may be more enlightening. We can hardly do better than quote
his forcible and quaint language, when, enumerating different classes of
men, he shows different ways in which thought goes
wrong:

\marginpar{(\emph{a}) dependence on others,}

1. "The first is of those who seldom reason at all, but do and think
according to the example of others, whether parents, neighbors,
ministers, or who else they are pleased to make choice of to have an
implicit faith in, for the saving of themselves the pains and troubles
of thinking and examining for themselves."

\marginpar{(\emph{b}) self-interest,}

2. "This kind is of those who put passion in the place of reason, and
being resolved that shall govern their actions and arguments, neither
use their own, nor hearken to other people's reason, any farther than it
suits their humor, interest, or
party."\footnote{In another place he says: "Men's prejudices and inclinations impose
often upon themselves.... Inclination suggests and slides into discourse
favorable terms, which introduce favorable ideas; till at last by this
means that is concluded clear and evident, thus dressed up, which, taken
in its native state, by making use of none but precise determined ideas,
would find no admittance at all."
}

\marginpar{(\emph{c}) circumscribed experience}

3. "The third sort is of those who readily and sincerely follow reason,
but for want of having that which one may call large, sound, roundabout
sense, have not a full view of all that relates to the question.... They
converse but with one sort of men, they read but one sort of books, they
will not come in the hearing but of one sort of notions.... They have a
pretty traffic with known correspondents in some little creek ... but
will not venture out into the great ocean of knowledge." Men of
originally equal natural parts may finally arrive at very different
stores of knowledge and truth, "when all the odds between them has been
the different scope that has been given to their understandings to range
in, for the gathering up of information and furnishing their heads with
ideas and notions and observations, whereon to employ their
mind."\footnote{\emph{The Conduct of the Understanding}, § 3.}

In another portion of his
writings,\footnote{\emph{Essay Concerning Human Understanding}, bk. IV, ch. XX, "Of Wrong Assent or Error."}
Locke states the same ideas in slightly different form.

\marginpar{Effect of dogmatic principles,}

1. "That which is inconsistent with our \emph{principles} is so far from
passing for probable with us that it will not be allowed possible. The
reverence borne to these principles is so great, and their authority so
paramount to all other, that the testimony, not only of other men, but
the evidence of our own senses are often rejected, when they offer to
vouch anything contrary to these \emph{established rules}.... There is
nothing more ordinary than children's receiving into their minds
propositions ... from their parents, nurses, or those about them; which
being insinuated in their unwary as well as unbiased understandings, and
fastened by degrees, are at last (and this whether true or false)
riveted there by long custom and education, beyond all possibility of
being pulled out again. For men, when they are grown up, reflecting upon
their opinions and finding those of this sort to be as ancient in their
minds as their very memories, not having observed their early
insinuation, nor by what means they got them, they are apt to reverence
them as sacred things, and not to suffer them to be profaned, touched,
or questioned." They take them as standards "to be the great and
unerring deciders of truth and falsehood, and the judges to which they
are to appeal in all manner of controversies."

\marginpar{of closed minds,}

2. "Secondly, next to these are men whose understandings are cast into a
mold, and fashioned just to the size of a received hypothesis." Such
men, Locke goes on to say, while not denying the existence of facts and
evidence, cannot be convinced by the evidence
that
would decide them if their minds were not so closed by adherence to
fixed belief.

\marginpar{of strong passion,}

3. "Predominant Passions. Thirdly, probabilities which cross men's
appetites and prevailing passions run the same fate. Let ever so much
probability hang on one side of a covetous man's reasoning, and money on
the other, it is easy to foresee which will outweigh. Earthly minds,
like mud walls, resist the strongest batteries.

\marginpar{of dependence upon authority of others}

4. "Authority. The fourth and last wrong measure of probability I shall
take notice of, and which keeps in ignorance or error more people than
all the others together, is the giving up our assent to the common
received opinions, either of our friends or party, neighborhood or
country."

\marginpar{Causes of bad mental habits are social as well as inborn}

Both Bacon and Locke make it evident that over and above the sources of
misbelief that reside in the natural tendencies of the individual (like
those toward hasty and too far-reaching conclusions), social conditions
tend to instigate and confirm wrong habits of thinking by authority, by
conscious instruction, and by the even more insidious half-conscious
influences of language, imitation, sympathy, and suggestion. Education
has accordingly not only to safeguard an individual against the
besetting erroneous tendencies of his own mind---its rashness,
presumption, and preference of what chimes with self-interest to
objective evidence---but also to undermine and destroy the accumulated
and self-perpetuating prejudices of long ages. When social life in
general has become more reasonable, more imbued with rational
conviction, and less moved by stiff authority and blind passion,
educational agencies may be more positive and constructive than at
present, for they
will
work in harmony with the educative influence exercised willy-nilly by
other social surroundings upon an individual's habits of thought and
belief. At present, the work of teaching must not only transform natural
tendencies into trained habits of thought, but must also fortify the
mind against irrational tendencies current in the social environment,
and help displace erroneous habits already produced.

\section{Regulation Transforms Inference into Proof}

\marginpar{A leap is involved in all thinking}

Thinking is important because, as we have seen, it is that function in
which given or ascertained facts stand for or indicate others which are
not directly ascertained. But the process of reaching the absent from
the present is peculiarly exposed to error; it is liable to be
influenced by almost any number of unseen and unconsidered
causes,---past experience, received dogmas, the stirring of
self-interest, the arousing of passion, sheer mental laziness, a social
environment steeped in biased traditions or animated by false
expectations, and so on. The exercise of thought is, in the literal
sense of that word, \emph{inference}; by it one thing \emph{carries us
over} to the idea of, and belief in, another thing. It involves a jump,
a leap, a going beyond what is surely known to something else accepted
on its warrant. Unless one is an idiot, one simply cannot help having
all things and events suggest other things not actually present, nor can
one help a tendency to believe in the latter on the basis of the former.
The very inevitableness of the jump, the leap, to something unknown,
only emphasizes the necessity of attention to the conditions under which
it occurs so that the danger of a false step may be lessened and the
probability of a right landing
increased.

\marginpar{Hence, the need of regulation which, when adequate, makes proof}

Such attention consists in regulation (1) of the conditions under which
the function of suggestion takes place, and (2) of the conditions under
which credence is yielded to the suggestions that occur. Inference
controlled in these two ways (the study of which in detail constitutes
one of the chief objects of this book) forms \emph{proof}. To prove a
thing means primarily to try, to test it. The guest bidden to the
wedding feast excused himself because he had to \emph{prove} his oxen.
Exceptions are said to prove a rule; \emph{i.e.} they furnish instances
so extreme that they try in the severest fashion its applicability; if
the rule will stand such a test, there is no good reason for further
doubting it. Not until a thing has been tried---"tried out," in
colloquial language---do we know its true worth. Till then it may be
pretense, a bluff. But the thing that has come out victorious in a test
or trial of strength carries its credentials with it; it is approved,
because it has been proved. Its value is clearly evinced, shown,
\emph{i.e.} demonstrated. So it is with inferences. The mere fact that
inference in general is an invaluable function does not guarantee, nor
does it even help out the correctness of any particular inference. Any
inference may go astray; and as we have seen, there are standing
influences ever ready to assist its going wrong. \emph{What is
important, is that every inference shall be a tested inference};
\emph{or} (since often this is not possible) \emph{that we shall
discriminate between beliefs that rest upon tested evidence and those
that do not, and shall be accordingly on our guard as to the kind and
degree of assent yielded}.

\marginpar{The office of education in forming skilled}

\marginpar{powers of thinking}

While it is not the business of education to prove every statement made,
any more than to teach every possible item of information, it is its
business to
cultivate
deep-seated and effective habits of discriminating tested beliefs from
mere assertions, guesses, and opinions; to develop a lively, sincere,
and open-minded preference for conclusions that are properly grounded,
and to ingrain into the individual's working habits methods of inquiry
and reasoning appropriate to the various problems that present
themselves. No matter how much an individual knows as a matter of
hearsay and information, if he has not attitudes and habits of this
sort, he is not intellectually educated. He lacks the rudiments of
mental discipline. And since these habits are not a gift of nature (no
matter how strong the aptitude for acquiring them); since, moreover, the
casual circumstances of the natural and social environment are not
enough to compel their acquisition, the main office of education is to
supply conditions that make for their cultivation. The formation of
these habits is the Training of
Mind.

\chapter{Natural Resources in the Training of Thought}

\marginpar{Only native powers can be trained.}

In the last chapter we considered the need of transforming, through
training, the natural capacities of inference into habits of critical
examination and inquiry. The very importance of thought for life makes
necessary its control by education because of its natural tendency to go
astray, and because social influences exist that tend to form habits of
thought leading to inadequate and erroneous beliefs. Training must,
however, be itself based upon the natural tendencies,---that is, it must
find its point of departure in them. A being who could not think without
training could never be trained to think; one may have to learn to think
\emph{well}, but not to \emph{think}. Training, in short, must fall back
upon the prior and independent existence of natural powers; it is
concerned with their proper direction, not with creating them.

\marginpar{Hence, the one taught must take the initiative}

Teaching and learning are correlative or corresponding processes, as
much so as selling and buying. One might as well say he has sold when no
one has bought, as to say that he has taught when no one has learned.
And in the educational transaction, the initiative lies with the learner
even more than in commerce it lies with the buyer. If an individual can
learn to think only in the sense of learning to employ more economically
and
effectively powers he already possesses, even more truly one can teach
others to think only in the sense of appealing to and fostering powers
already active in them. Effective appeal of this kind is impossible
unless the teacher has an insight into existing habits and tendencies,
the natural resources with which he has to ally himself.

\marginpar{Three important natural resources}

Any inventory of the items of this natural capital is somewhat arbitrary
because it must pass over many of the complex details. But a statement
of the factors essential to thought will put before us in outline the
main elements. Thinking involves (as we have seen) the suggestion of a
conclusion for acceptance, and also search or inquiry to test the value
of the suggestion before finally accepting it. This implies (\emph{a}) a
certain fund or store of experiences and facts from which suggestions
proceed; (\emph{b}) promptness, flexibility, and fertility of
suggestions; and (\emph{c}) orderliness, consecutiveness,
appropriateness in what is suggested. Clearly, a person may be hampered
in any of these three regards: His thinking may be irrelevant, narrow,
or crude because he has not enough actual material upon which to base
conclusions; or because concrete facts and raw material, even if
extensive and bulky, fail to evoke suggestions easily and richly; or
finally, because, even when these two conditions are fulfilled, the
ideas suggested are incoherent and fantastic, rather than pertinent and
consistent.

\section{Curiosity}

\marginpar{Desire for fullness of experience:}

The most vital and significant factor in supplying the primary material
whence suggestion may issue is, without doubt, curiosity. The wisest of
the Greeks used
to
say that wonder is the mother of all science. An inert mind waits, as it
were, for experiences to be imperiously forced upon it. The pregnant
saying of Wordsworth:

{"The eye---it cannot choose but see;\\
} {We cannot bid the ear be still;\\
} {Our bodies feel, where'er they be,\\
} {Against or with our will"---\\
}

holds good in the degree in which one is naturally possessed by
curiosity. The curious mind is constantly alert and exploring, seeking
material for thought, as a vigorous and healthy body is on the \emph{qui
vive} for nutriment. Eagerness for experience, for new and varied
contacts, is found where wonder is found. Such curiosity is the only
sure guarantee of the acquisition of the primary facts upon which
inference must base itself.

\marginpar{(\emph{a}) physical}

(\emph{a}) In its first manifestations, curiosity is a vital overflow,
an expression of an abundant organic energy. A physiological uneasiness
leads a child to be "into everything,"---to be reaching, poking,
pounding, prying. Observers of animals have noted what one author calls
"their inveterate tendency to fool." "Rats run about, smell, dig, or
gnaw, without real reference to the business in hand. In the same way
Jack {[}a dog{]} scrabbles and jumps, the kitten wanders and picks, the
otter slips about everywhere like ground lightning, the elephant fumbles
ceaselessly, the monkey pulls things
about."\footnote{Hobhouse, \emph{Mind in Evolution}, p. 195. }
The most casual notice of the activities of a young child reveals a
ceaseless display of exploring and testing activity. Objects are sucked,
fingered, and thumped; drawn and pushed, handled and thrown; in short,
experimented
with, till they cease to yield new qualities. Such activities are hardly
intellectual, and yet without them intellectual activity would be feeble
and intermittent through lack of stuff for its operations.

\marginpar{(\emph{b}) social}

(\emph{b}) A higher stage of curiosity develops under the influence of
social stimuli. When the child learns that he can appeal to others to
eke out his store of experiences, so that, if objects fail to respond
interestingly to his experiments, he may call upon persons to provide
interesting material, a new epoch sets in. "What is that?" "Why?" become
the unfailing signs of a child's presence. At first this questioning is
hardly more than a projection into social relations of the physical
overflow which earlier kept the child pushing and pulling, opening and
shutting. He asks in succession what holds up the house, what holds up
the soil that holds the house, what holds up the earth that holds the
soil; but his questions are not evidence of any genuine consciousness of
rational connections. His \emph{why} is not a demand for scientific
explanation; the motive behind it is simply eagerness for a larger
acquaintance with the mysterious world in which he is placed. The search
is not for a law or principle, but only for a bigger fact. Yet there is
more than a desire to accumulate just information or heap up
disconnected items, although sometimes the interrogating habit threatens
to degenerate into a mere disease of language. In the feeling, however
dim, that the facts which directly meet the senses are not the whole
story, that there is more behind them and more to come from them, lies
the germ of \emph{intellectual} curiosity.

\marginpar{(\emph{c}) intellectual}

(\emph{c}) Curiosity rises above the organic and the social planes and
becomes intellectual in the degree in
which
it is transformed into interest in \emph{problems} provoked by the
observation of things and the accumulation of material. When the
question is not discharged by being asked of another, when the child
continues to entertain it in his own mind and to be alert for whatever
will help answer it, curiosity has become a positive intellectual force.
To the open mind, nature and social experience are full of varied and
subtle challenges to look further. If germinating powers are not used
and cultivated at the right moment, they tend to be transitory, to die
out, or to wane in intensity. This general law is peculiarly true of
sensitiveness to what is uncertain and questionable; in a few people,
intellectual curiosity is so insatiable that nothing will discourage it,
but in most its edge is easily dulled and blunted. Bacon's saying that
we must become as little children in order to enter the kingdom of
science is at once a reminder of the open-minded and flexible wonder of
childhood and of the ease with which this endowment is lost. Some lose
it in indifference or carelessness; others in a frivolous flippancy;
many escape these evils only to become incased in a hard dogmatism which
is equally fatal to the spirit of wonder. Some are so taken up with
routine as to be inaccessible to new facts and problems. Others retain
curiosity only with reference to what concerns their personal advantage
in their chosen career. With many, curiosity is arrested on the plane of
interest in local gossip and in the fortunes of their neighbors; indeed,
so usual is this result that very often the first association with the
word \emph{curiosity} is a prying inquisitiveness into other people's
business. With respect then to curiosity, the teacher has usually more
to learn than to teach. Rarely can he aspire to the office of kindling
or
even increasing it. His task is rather to keep alive the sacred spark of
wonder and to fan the flame that already glows. His problem is to
protect the spirit of inquiry, to keep it from becoming blasé from
overexcitement, wooden from routine, fossilized through dogmatic
instruction, or dissipated by random exercise upon trivial things.

\section{Suggestion}

Out of the subject-matter, whether rich or scanty, important or trivial,
of present experience issue suggestions, ideas, beliefs as to what is
not yet given. The function of suggestion is not one that can be
produced by teaching; while it may be modified for better or worse by
conditions, it cannot be destroyed. Many a child has tried his best to
see if he could not "stop thinking," but the flow of suggestions goes on
in spite of our will, quite as surely as "our bodies feel, where'er they
be, against or with our will." Primarily, naturally, it is not we who
think, in any actively responsible sense; thinking is rather something
that happens in us. Only so far as one has acquired control of the
method in which the function of suggestion occurs and has accepted
responsibility for its consequences, can one truthfully say, "\emph{I}
think so and so."

\marginpar{The dimensions of suggestion:}

\marginpar{(\emph{a}) ease}

The function of suggestion has a variety of aspects (or dimensions as we
may term them), varying in different persons, both in themselves and in
their mode of combination. These dimensions are ease or promptness,
extent or variety, and depth or persistence. (\emph{a}) The common
classification of persons into the dull and the bright is made primarily
on the basis of the readiness or facility with which suggestions follow
upon the
presentation
of objects and upon the happening of events. As the metaphor of dull and
bright implies, some minds are impervious, or else they absorb
passively. Everything presented is lost in a drab monotony that gives
nothing back. But others reflect, or give back in varied lights, all
that strikes upon them. The dull make no response; the bright flash back
the fact with a changed quality. An inert or stupid mind requires a
heavy jolt or an intense shock to move it to suggestion; the bright mind
is quick, is alert to react with interpretation and suggestion of
consequences to follow.

Yet the teacher is not entitled to assume stupidity or even dullness
merely because of irresponsiveness to school subjects or to a lesson as
presented by text-book or teacher. The pupil labeled hopeless may react
in quick and lively fashion when the thing-in-hand seems to him worth
while, as some out-of-school sport or social affair. Indeed, the school
subject might move him, were it set in a different context and treated
by a different method. A boy dull in geometry may prove quick enough
when he takes up the subject in connection with manual training; the
girl who seems inaccessible to historical facts may respond promptly
when it is a question of judging the character and deeds of people of
her acquaintance or of fiction. Barring physical defect or disease,
slowness and dullness in \emph{all} directions are comparatively rare.

\marginpar{(\emph{b}) range}

(\emph{b}) Irrespective of the difference in persons as to the ease and
promptness with which ideas respond to facts, there is a difference in
the number or range of the suggestions that occur. We speak truly, in
some cases, of the flood of suggestions; in others, there is but a
slender trickle. Occasionally, slowness of
outward
response is due to a great variety of suggestions which check one
another and lead to hesitation and suspense; while a lively and prompt
suggestion may take such possession of the mind as to preclude the
development of others. Too few suggestions indicate a dry and meager
mental habit; when this is joined to great learning, there results a
pedant or a Gradgrind. Such a person's mind rings hard; he is likely to
bore others with mere bulk of information. He contrasts with the person
whom we call ripe, juicy, and mellow.

A conclusion reached after consideration of a few alternatives may be
formally correct, but it will not possess the fullness and richness of
meaning of one arrived at after comparison of a greater variety of
alternative suggestions. On the other hand, suggestions may be too
numerous and too varied for the best interests of mental habit. So many
suggestions may rise that the person is at a loss to select among them.
He finds it difficult to reach any definite conclusion and wanders more
or less helplessly among them. So much suggests itself \emph{pro} and
\emph{con}, one thing leads on to another so naturally, that he finds it
difficult to decide in practical affairs or to conclude in matters of
theory. There is such a thing as too much thinking, as when action is
paralyzed by the multiplicity of views suggested by a situation. Or
again, the very number of suggestions may be hostile to tracing logical
sequences among them, for it may tempt the mind away from the necessary
but trying task of search for real connections, into the more congenial
occupation of embroidering upon the given facts a tissue of agreeable
fancies. The best mental habit involves a balance between paucity and
redundancy of
suggestions.

\marginpar{(\emph{c}) profundity}

(\emph{c}) \emph{Depth.} We distinguish between people not only upon the
basis of their quickness and fertility of intellectual response, but
also with respect to the plane upon which it occurs---the intrinsic
quality of the response.

One man's thought is profound while another's is superficial; one goes
to the roots of the matter, and another touches lightly its most
external aspects. This phase of thinking is perhaps the most untaught of
all, and the least amenable to external influence whether for
improvement or harm. Nevertheless, the conditions of the pupil's contact
with subject-matter may be such that he is compelled to come to quarters
with its more significant features, or such that he is encouraged to
deal with it upon the basis of what is trivial. The common assumptions
that, if the pupil only thinks, one thought is just as good for his
mental discipline as another, and that the end of study is the amassing
of information, both tend to foster superficial, at the expense of
significant, thought. Pupils who in matters of ordinary practical
experience have a ready and acute perception of the difference between
the significant and the meaningless, often reach in school subjects a
point where all things seem equally important or equally unimportant;
where one thing is just as likely to be true as another, and where
intellectual effort is expended not in discriminating between things,
but in trying to make verbal connections among words.

\marginpar{Balance of mind}

Sometimes slowness and depth of response are intimately connected. Time
is required in order to digest impressions, and translate them into
substantial ideas. "Brightness" may be but a flash in the pan. The "slow
but sure" person, whether man or child, is one in whom impressions sink
and accumulate, so that thinking is
done
at a deeper level of value than with a slighter load. Many a child is
rebuked for "slowness," for not "answering promptly," when his forces
are taking time to gather themselves together to deal effectively with
the problem at hand. In such cases, failure to afford time and leisure
conduce to habits of speedy, but snapshot and superficial, judgment. The
depth to which a sense of the problem, of the difficulty, sinks,
determines the quality of the thinking that follows; and any habit of
teaching which encourages the pupil for the sake of a successful
recitation or of a display of memorized information to glide over the
thin ice of genuine problems reverses the true method of mind training.

\marginpar{Individual differences}

It is profitable to study the lives of men and women who achieve in
adult life fine things in their respective callings, but who were called
dull in their school days. Sometimes the early wrong judgment was due
mainly to the fact that the direction in which the child showed his
ability was not one recognized by the good old standards in use, as in
the case of Darwin's interest in beetles, snakes, and frogs. Sometimes
it was due to the fact that the child dwelling habitually on a deeper
plane of reflection than other pupils---or than his teachers---did not
show to advantage when prompt answers of the usual sort were expected.
Sometimes it was due to the fact that the pupil's natural mode of
approach clashed habitually with that of the text or teacher, and the
method of the latter was assumed as an absolute basis of estimate.

\marginpar{Any subject may be intellectual}

In any event, it is desirable that the teacher should rid himself of the
notion that "thinking" is a single, unalterable faculty; that he should
recognize that it is a term denoting the various ways in which things
acquire
significance. It is desirable to expel also the kindred notion that some
subjects are inherently "intellectual," and hence possessed of an almost
magical power to train the faculty of thought. Thinking is specific, not
a machine-like, ready-made apparatus to be turned indifferently and at
will upon all subjects, as a lantern may throw its light as it happens
upon horses, streets, gardens, trees, or river. Thinking is specific, in
that different things suggest their own appropriate meanings, tell their
own unique stories, and in that they do this in very different ways with
different persons. As the growth of the body is through the assimilation
of food, so the growth of mind is through the logical organization of
subject-matter. Thinking is not like a sausage machine which reduces all
materials indifferently to one marketable commodity, but is a power of
following up and linking together the specific suggestions that specific
things arouse. Accordingly, any subject, from Greek to cooking, and from
drawing to mathematics, is intellectual, if intellectual at all, not in
its fixed inner structure, but in its function---in its power to start
and direct significant inquiry and reflection. What geometry does for
one, the manipulation of laboratory apparatus, the mastery of a musical
composition, or the conduct of a business affair, may do for another.

\section{Orderliness: Its Nature}

\marginpar{Continuity}

Facts, whether narrow or extensive, and conclusions suggested by them,
whether many or few, do not constitute, even when combined, reflective
thought. The suggestions must be \emph{organized}; they must be arranged
with reference to one another and with reference to the facts on which
they depend for proof. When
the
factors of facility, of fertility, and of depth are properly balanced or
proportioned, we get as the outcome continuity of thought. We desire
neither the slow mind nor yet the hasty. We wish neither random
diffuseness nor fixed rigidity. Consecutiveness means flexibility and
variety of materials, conjoined with singleness and definiteness of
direction. It is opposed both to a mechanical routine uniformity and to
a grasshopper-like movement. Of bright children, it is not infrequently
said that "they might do anything, if only they settled down," so quick
and apt are they in any particular response. But, alas, they rarely
settle.

On the other hand, it is not enough \emph{not} to be diverted. A deadly
and fanatic consistency is not our goal. Concentration does not mean
fixity, nor a cramped arrest or paralysis of the flow of suggestion. It
means variety and change of ideas combined into a \emph{single steady
trend moving toward a unified conclusion}. Thoughts are concentrated not
by being kept still and quiescent, but by being kept moving toward an
object, as a general concentrates his troops for attack or defense.
Holding the mind to a subject is like holding a ship to its course; it
implies constant change of place combined with unity of direction.
Consistent and orderly thinking is precisely such a change of
subject-matter. Consistency is no more the mere absence of contradiction
than concentration is the mere absence of diversion---which exists in
dull routine or in a person "fast asleep." All kinds of varied and
incompatible suggestions may sprout and be followed in their growth, and
yet thinking be consistent and orderly, provided each one of the
suggestions is viewed in relation to the main topic.

\marginpar{Practical demands enforce some degree of continuity}

In the main, for most persons, the primary
resource
in the development of orderly habits of thought is indirect, not direct.
Intellectual organization originates and for a time grows as an
accompaniment of the organization of the acts required to realize an
end, not as the result of a direct appeal to thinking power. The need of
thinking to accomplish something beyond thinking is more potent than
thinking for its own sake. All people at the outset, and the majority of
people probably all their lives, attain ordering of thought through
ordering of action. Adults normally carry on some occupation,
profession, pursuit; and this furnishes the continuous axis about which
their knowledge, their beliefs, and their habits of reaching and testing
conclusions are organized. Observations that have to do with the
efficient performance of their calling are extended and rendered
precise. Information related to it is not merely amassed and then left
in a heap; it is classified and subdivided so as to be available as it
is needed. Inferences are made by most men not from purely speculative
motives, but because they are involved in the efficient performance of
"the duties involved in their several callings." Thus their inferences
are constantly tested by results achieved; futile and scattering methods
tend to be discounted; orderly arrangements have a premium put upon
them. The event, the issue, stands as a constant check on the thinking
that has led up to it; and this discipline by efficiency in action is
the chief sanction, in practically all who are not scientific
specialists, of orderliness of thought.

Such a resource---the main prop of disciplined thinking in adult
life---is not to be despised in training the young in right intellectual
habits. There are, however, profound differences between the immature
and
the
adult in the matter of organized activity---differences which must be
taken seriously into account in any educational use of activities:
(\emph{i}) The external achievement resulting from activity is a more
urgent necessity with the adult, and hence is with him a more effective
means of discipline of mind than with the child; (\emph{ii}) The ends of
adult activity are more specialized than those of child activity.

\marginpar{Peculiar difficulty with children}

(\emph{i}) The selection and arrangement of appropriate lines of action
is a much more difficult problem as respects youth than it is in the
case of adults. With the latter, the main lines are more or less settled
by circumstances. The social status of the adult, the fact that he is a
citizen, a householder, a parent, one occupied in some regular
industrial or professional calling, prescribes the chief features of the
acts to be performed, and secures, somewhat automatically, as it were,
appropriate and related modes of thinking. But with the child there is
no such fixity of status and pursuit; there is almost nothing to dictate
that such and such a consecutive line of action, rather than another,
should be followed, while the will of others, his own caprice, and
circumstances about him tend to produce an isolated momentary act. The
absence of continued motivation coöperates with the inner plasticity of
the immature to increase the importance of educational training and the
difficulties in the way of finding consecutive modes of activities which
may do for child and youth what serious vocations and functions do for
the adult. In the case of children, the choice is so peculiarly exposed
to arbitrary factors, to mere school traditions, to waves of pedagogical
fad and fancy, to fluctuating social cross currents, that sometimes, in
sheer disgust at the inadequacy of results, a reaction
occurs
to the total neglect of overt activity as an educational factor, and a
recourse to purely theoretical subjects and methods.

\marginpar{Peculiar opportunity with children}

(\emph{ii}) This very difficulty, however, points to the fact that the
\emph{opportunity for selecting truly educative activities} is
indefinitely greater in child life than in adult. The factor of external
pressure is so strong with most adults that the educative value of the
pursuit---its reflex influence upon intelligence and character---however
genuine, is incidental, and frequently almost accidental. The problem
and the opportunity with the young is selection of orderly and
continuous modes of occupation, which, while they lead up to and prepare
for the indispensable activities of adult life, have their own
\emph{sufficient justification in their present reflex influence upon
the formation of habits of thought}.

\marginpar{Action and reaction between extremes}

Educational practice shows a continual tendency to oscillate between two
extremes with respect to overt and exertive activities. One extreme is
to neglect them almost entirely, on the ground that they are chaotic and
fluctuating, mere diversions appealing to the transitory unformed taste
and caprice of immature minds; or if they avoid this evil, are
objectionable copies of the highly specialized, and more or less
commercial, activities of adult life. If activities are admitted at all
into the school, the admission is a grudging concession to the necessity
of having occasional relief from the strain of constant intellectual
work, or to the clamor of outside utilitarian demands upon the school.
The other extreme is an enthusiastic belief in the almost magical
educative efficacy of any kind of activity, granted it is an activity
and not a passive absorption of academic and theoretic material. The
conceptions of play,
of
self-expression, of natural growth, are appealed to almost as if they
meant that opportunity for any kind of spontaneous activity inevitably
secures the due training of mental power; or a mythological brain
physiology is appealed to as proof that any exercise of the muscles
trains power of thought.

\marginpar{Locating the problem of education}

While we vibrate from one of these extremes to the other, the most
serious of all problems is ignored: the problem, namely, of discovering
and arranging the forms of activity (\emph{a}) which are most congenial,
best adapted, to the immature stage of development; (\emph{b}) which
have the most ulterior promise as preparation for the social
responsibilities of adult life; and (\emph{c}) which, \emph{at the same
time}, have the maximum of influence in forming habits of acute
observation and of consecutive inference. As curiosity is related to the
acquisition of material of thought, as suggestion is related to
flexibility and force of thought, so the ordering of activities, not
themselves primarily intellectual, is related to the forming of
intellectual powers of
consecutiveness.

\chapter{School Conditions and the Training of Thought}

\section{Introductory: Methods and Conditions}

\marginpar{Formal discipline}

The so-called faculty-psychology went hand in hand with the vogue of the
formal-discipline idea in education. If thought is a distinct piece of
mental machinery, separate from observation, memory, imagination, and
common-sense judgments of persons and things, then thought should be
trained by special exercises designed for the purpose, as one might
devise special exercises for developing the biceps muscles. Certain
subjects are then to be regarded as intellectual or logical subjects
\emph{par excellence}, possessed of a predestined fitness to exercise
the thought-faculty, just as certain machines are better than others for
developing arm power. With these three notions goes the fourth, that
method consists of a set of operations by which the machinery of thought
is set going and kept at work upon any subject-matter.

\marginpar{versus real thinking}

We have tried to make it clear in the previous chapters that there is no
single and uniform power of thought, but a multitude of different ways
in which specific things---things observed, remembered, heard of, read
about---evoke suggestions or ideas that are pertinent to the occasion
and fruitful in the sequel. Training is such development of curiosity,
suggestion, and habits of exploring and testing, as increases their
scope
and efficiency. A subject---any subject---is intellectual in the degree
in which \emph{with any given person} it succeeds in effecting this
growth. On this view the fourth factor, method, is concerned with
providing conditions so adapted to individual needs and powers as to
make for the permanent improvement of observation, suggestion, and
investigation.

\marginpar{True and false meaning of method}

The teacher's problem is thus twofold. On the one side, he needs (as we
saw in the last chapter) to be a student of individual traits and
habits; on the other side, he needs to be a student of the conditions
that modify for better or worse the directions in which individual
powers habitually express themselves. He needs to recognize that method
covers not only what he intentionally devises and employs for the
purpose of mental training, but also what he does without any conscious
reference to it,---anything in the atmosphere and conduct of the school
which reacts in any way upon the curiosity, the responsiveness, and the
orderly activity of children. The teacher who is an intelligent student
both of individual mental operations and of the effects of school
conditions upon those operations, can largely be trusted to develop for
himself methods of instruction in their narrower and more technical
sense---those best adapted to achieve results in particular subjects,
such as reading, geography, or algebra. In the hands of one who is not
intelligently aware of individual capacities and of the influence
unconsciously exerted upon them by the entire environment, even the best
of technical methods are likely to get an immediate result only at the
expense of deep-seated and persistent habits. We may group the
conditioning influences of the school environment under three heads: (1)
the mental attitudes and habits of
the
persons with whom the child is in contact; (2) the subjects studied; (3)
current educational aims and ideals.

\section{Influence of the Habits of Others}

Bare reference to the imitativeness of human nature is enough to suggest
how profoundly the mental habits of others affect the attitude of the
one being trained. Example is more potent than precept; and a teacher's
best conscious efforts may be more than counteracted by the influence of
personal traits which he is unaware of or regards as unimportant.
Methods of instruction and discipline that are technically faulty may be
rendered practically innocuous by the inspiration of the personal method
that lies back of them.

\marginpar{Response to environment fundamental in method}

To confine, however, the conditioning influence of the educator, whether
parent or teacher, to imitation is to get a very superficial view of the
intellectual influence of others. Imitation is but one case of a deeper
principle---that of stimulus and response. \emph{Everything the teacher
does, as well as the manner in which he does it, incites the child to
respond in some way or other, and each response tends to set the child's
attitude in some way or other.} Even the inattention of the child to the
adult is often a mode of response which is the result of unconscious
training.\footnote{
A child of four or five who had been repeatedly called to the house by
his mother with no apparent response on his own part, was asked if he
did not hear her. He replied quite judicially, "Oh, yes, but she doesn't
call very mad yet."
}
The teacher is rarely (and even then never entirely) a transparent
medium of access by another mind to a subject. With the young, the
influence of the teacher's personality is intimately fused with that of
the subject; the child does not
separate
nor even distinguish the two. And as the child's response is
\emph{toward} or \emph{away from} anything presented, he keeps up a
running commentary, of which he himself is hardly distinctly aware, of
like and dislike, of sympathy and aversion, not merely to the acts of
the teacher, but also to the subject with which the teacher is occupied.

\marginpar{Influence of teacher's own habits}

\marginpar{Judging others by ourselves}

The extent and power of this influence upon morals and manners, upon
character, upon habits of speech and social bearing, are almost
universally recognized. But the tendency to conceive of thought as an
isolated faculty has often blinded teachers to the fact that this
influence is just as real and pervasive in intellectual concerns.
Teachers, as well as children, stick more or less to the main points,
have more or less wooden and rigid methods of response, and display more
or less intellectual curiosity about matters that come up. And every
trait of this kind is an inevitable part of the teacher's method of
teaching. Merely to accept without notice slipshod habits of speech,
slovenly inferences, unimaginative and literal response, is to indorse
these tendencies, and to ratify them into habits---and so it goes
throughout the whole range of contact between teacher and student. In
this complex and intricate field, two or three points may well be
singled out for special notice. (\emph{a}) Most persons are quite
unaware of the distinguishing peculiarities of their own mental habit.
They take their own mental operations for granted, and unconsciously
make them the standard for judging the mental processes of
others.\footnote{
People who have \emph{number-forms}---\emph{i.e.} project number series
into space and see them arranged in certain shapes---when asked why they
have not mentioned the fact before, often reply that it never occurred
to them; they supposed that everybody had the same power.
}
Hence
there
is a tendency to encourage everything in the pupil which agrees with
this attitude, and to neglect or fail to understand whatever is
incongruous with it. The prevalent overestimation of the value, for
mind-training, of \emph{theoretic} subjects as compared with practical
pursuits, is doubtless due partly to the fact that the teacher's calling
tends to select those in whom the theoretic interest is specially strong
and to repel those in whom executive abilities are marked. Teachers
sifted out on this basis judge pupils and subjects by a like standard,
encouraging an intellectual one-sidedness in those to whom it is
naturally congenial, and repelling from study those in whom practical
instincts are more urgent.

\marginpar{Exaggeration of direct personal influence}

(\emph{b}) Teachers---and this holds especially of the stronger and
better teachers---tend to rely upon their personal strong points to hold
a child to his work, and thereby to substitute their personal influence
for that of subject-matter as a motive for study. The teacher finds by
experience that his own personality is often effective where the power
of the subject to command attention is almost nil; then he utilizes the
former more and more, until the pupil's relation to the teacher almost
takes the place of his relation to the subject. In this way the
teacher's personality may become a source of personal dependence and
weakness, an influence that renders the pupil indifferent to the value
of the subject for its own sake.

\marginpar{Independent thinking \emph{versus} "getting the answer"}

(\emph{c}) The operation of the teacher's own mental habit tends, unless
carefully watched and guided, to make the child a student of the
teacher's peculiarities rather than of the subjects that he is supposed
to study. His chief concern is to accommodate himself to what
the
teacher expects of him, rather than to devote himself energetically to
the problems of subject-matter. "Is this right?" comes to mean "Will
this answer or this process satisfy the teacher?"---instead of meaning,
"Does it satisfy the inherent conditions of the problem?" It would be
folly to deny the legitimacy or the value of the study of human nature
that children carry on in school; but it is obviously undesirable that
their chief intellectual problem should be that of producing an answer
approved by the teacher, and their standard of success be successful
adaptation to the requirements of another.

\section{Influence of the Nature of Studies}

\marginpar{Types of studies}

Studies are conventionally and conveniently grouped under these heads:
(1) Those especially involving the acquisition of skill in
performance---the school arts, such as reading, writing, figuring, and
music. (2) Those mainly concerned with acquiring
knowledge---"informational" studies, such as geography and history. (3)
Those in which skill in doing and bulk of information are relatively
less important, and appeal to abstract thinking, to "reasoning," is most
marked---"disciplinary" studies, such as arithmetic and formal
grammar.\footnote{
Of course, any one subject has all three aspects: \emph{e.g.} in
arithmetic, counting, writing, and reading numbers, rapid adding, etc.,
are cases of skill in doing; the tables of weights and measures are a
matter of information, etc.
}
Each of these groups of subjects has its own special pitfalls.

\marginpar{The abstract as the isolated}

(\emph{a}) In the case of the so-called disciplinary or pre-eminently
logical studies, there is danger of the isolation of intellectual
activity from the ordinary
affairs
of life. Teacher and student alike tend to set up a chasm between
logical thought as something abstract and remote, and the specific and
concrete demands of everyday events. The abstract tends to become so
aloof, so far away from application, as to be cut loose from practical
and moral bearing. The gullibility of specialized scholars when out of
their own lines, their extravagant habits of inference and speech, their
ineptness in reaching conclusions in practical matters, their
egotistical engrossment in their own subjects, are extreme examples of
the bad effects of severing studies completely from their ordinary
connections in life.

\marginpar{Overdoing the mechanical and automatic}

\marginpar{"Drill"}

(\emph{b}) The danger in those studies where the main emphasis is upon
acquisition of skill is just the reverse. The tendency is to take the
shortest cuts possible to gain the required end. This makes the subjects
\emph{mechanical}, and thus restrictive of intellectual power. In the
mastery of reading, writing, drawing, laboratory technique, etc., the
need of economy of time and material, of neatness and accuracy, of
promptness and uniformity, is so great that these things tend to become
ends in themselves, irrespective of their influence upon general mental
attitude. Sheer imitation, dictation of steps to be taken, mechanical
drill, may give results most quickly and yet strengthen traits likely to
be fatal to reflective power. The pupil is enjoined to do this and that
specific thing, with no knowledge of any reason except that by so doing
he gets his result most speedily; his mistakes are pointed out and
corrected for him; he is kept at pure repetition of certain acts till
they become automatic. Later, teachers wonder why the pupil reads with
so little expression, and figures with so little intelligent
consideration of the
terms
of his problem. In some educational dogmas and practices, the very idea
of training mind seems to be hopelessly confused with that of a drill
which hardly touches \emph{mind} at all---or touches it for the
worse---since it is wholly taken up with training skill in external
execution. This method reduces the "training" of human beings to the
level of animal training. Practical skill, modes of effective technique,
can be intelligently, non-mechanically \emph{used}, only when
intelligence has played a part in their \emph{acquisition}.

\marginpar{Wisdom \emph{versus} information}

(\emph{c}) Much the same sort of thing is to be said regarding studies
where emphasis traditionally falls upon bulk and accuracy of
information. The distinction between information and wisdom is old, and
yet requires constantly to be redrawn. Information is knowledge which is
merely acquired and stored up; wisdom is knowledge operating in the
direction of powers to the better living of life. Information, merely as
information, implies no special training of intellectual capacity;
wisdom is the finest fruit of that training. In school, amassing
information always tends to escape from the ideal of wisdom or good
judgment. The aim often seems to be---especially in such a subject as
geography---to make the pupil what has been called a "cyclopedia of
useless information." "Covering the ground" is the primary necessity;
the nurture of mind a bad second. Thinking cannot, of course, go on in a
vacuum; suggestions and inferences can occur only upon a basis of
information as to matters of fact.

But there is all the difference in the world whether the acquisition of
information is treated as an end in itself, or is made an integral
portion of the training of thought. The assumption that information
which
has
been accumulated apart from use in the recognition and solution of a
problem may later on be freely employed at will by thought is quite
false. The skill at the ready command of intelligence is the skill
acquired with the aid of intelligence; the only information which,
otherwise than by accident, can be put to logical use is that acquired
in the course of thinking. Because their knowledge has been achieved in
connection with the needs of specific situations, men of little
book-learning are often able to put to effective use every ounce of
knowledge they possess; while men of vast erudition are often swamped by
the mere bulk of their learning, because memory, rather than thinking,
has been operative in obtaining it.

\section{The Influence of Current Aims and Ideals}

It is, of course, impossible to separate this somewhat intangible
condition from the points just dealt with; for automatic skill and
quantity of information are educational ideals which pervade the whole
school. We may distinguish, however, certain tendencies, such as that to
judge education from the standpoint of external results, instead of from
that of the development of personal attitudes and habits. The ideal of
the \emph{product}, as against that of the mental \emph{process} by
which the product is attained, shows itself in both instruction and
moral discipline.

\marginpar{External results \emph{versus} processes}

(\emph{a}) In instruction, the external standard manifests itself in the
importance attached to the "correct answer." No one other thing,
probably, works so fatally against focussing the attention of teachers
upon the training of mind as the domination of \emph{their} minds by the
idea that the chief thing is to get pupils to recite their lessons
correctly.
As long as this end is uppermost (whether consciously or unconsciously),
training of mind remains an incidental and secondary consideration.
There is no great difficulty in understanding why this ideal has such
vogue. The large number of pupils to be dealt with, and the tendency of
parents and school authorities to demand speedy and tangible evidence of
progress, conspire to give it currency. Knowledge of
subject-matter---not of children---is alone exacted of teachers by this
aim; and, moreover, knowledge of subject-matter only in portions
definitely prescribed and laid out, and hence mastered with comparative
ease. Education that takes as its standard the improvement of the
intellectual attitude and method of students demands more serious
preparatory training, for it exacts sympathetic and intelligent insight
into the workings of individual minds, and a very wide and flexible
command of subject-matter---so as to be able to select and apply just
what is needed when it is needed. Finally, the securing of external
results is an aim that lends itself naturally to the mechanics of school
administration---to examinations, marks, gradings, promotions, and so
on.

\marginpar{Reliance upon others}

(\emph{b}) With reference to behavior also, the external ideal has a
great influence. Conformity of acts to precepts and rules is the
easiest, because most mechanical, standard to employ. It is no part of
our present task to tell just how far dogmatic instruction, or strict
adherence to custom, convention, and the commands of a social superior,
should extend in moral training; but since problems of conduct are the
deepest and most common of all the problems of life, the ways in which
they are met have an influence that radiates into every other mental
attitude, even those far remote from
any
direct or conscious moral consideration. Indeed, the \emph{deepest plane
of the mental attitude of every one is fixed by the way in which
problems of behavior are treated}. If the function of thought, of
serious inquiry and reflection, is reduced to a minimum in dealing with
them, it is not reasonable to expect habits of thought to exercise great
influence in less important matters. On the other hand, habits of active
inquiry and careful deliberation in the significant and vital problems
of conduct afford the best guarantee that the general structure of mind
will be
reasonable.

\chapter{The Means and End of Mental Training: The Psychological and the Logical}

\section{Introductory: The Meaning of Logical}

\marginpar{Special topic of this chapter}

In the preceding chapters we have considered (\emph{i}) what thinking
is; (\emph{ii}) the importance of its special training; (\emph{iii}) the
natural tendencies that lend themselves to its training; and (\emph{iv})
some of the special obstacles in the way of its training under school
conditions. We come now to the relation of \emph{logic} to the purpose
of mental training.

\marginpar{Three senses of term \emph{logical}}

\marginpar{The practical is the important meaning of \emph{logical}}

In its broadest sense, any thinking that ends in a conclusion is
logical---whether the conclusion reached be justified or fallacious;
that is, the term \emph{logical} covers both the logically good and the
illogical or the logically bad. In its narrowest sense, the term
\emph{logical} refers only to what is demonstrated to follow necessarily
from premises that are definite in meaning and that are either
self-evidently true, or that have been previously proved to be true.
Stringency of proof is here the equivalent of the logical. In this sense
mathematics and formal logic (perhaps as a branch of mathematics) alone
are strictly logical. Logical, however, is used in a third sense, which
is at once more vital and more practical; to denote, namely, the
systematic care, negative and positive, taken to safeguard reflection so
that it may yield the best results under the given conditions. If only
the word \emph{artificial} were associated with the
idea
of \emph{art}, or expert skill gained through voluntary apprenticeship
(instead of suggesting the factitious and unreal), we might say that
logical refers to artificial thought.

\marginpar{Care, thoroughness, and exactness the marks of the logical}

In this sense, the word \emph{logical} is synonymous with wide-awake,
thorough, and careful reflection---thought in its best sense
(\emph{ante}, p. 5). Reflection is turning a topic over in various
aspects and in various lights so that nothing significant about it shall
be overlooked---almost as one might turn a stone over to see what its
hidden side is like or what is covered by it. \emph{Thoughtfulness}
means, practically, the same thing as careful attention; to give our
mind to a subject is to give heed to it, to take pains with it. In
speaking of reflection, we naturally use the words \emph{weigh},
\emph{ponder}, \emph{deliberate}---terms implying a certain delicate and
scrupulous balancing of things against one another. Closely related
names are \emph{scrutiny}, \emph{examination}, \emph{consideration},
\emph{inspection}---terms which imply close and careful vision. Again,
to think is to relate things to one another definitely, to "put two and
two together" as we say. Analogy with the accuracy and definiteness of
mathematical combinations gives us such expressions as \emph{calculate},
\emph{reckon}, \emph{account for}; and even \emph{reason}
itself---\emph{ratio}. Caution, carefulness, thoroughness, definiteness,
exactness, orderliness, methodic arrangement, are, then, the traits by
which we mark off the logical from what is random and casual on one
side, and from what is academic and formal on the other.

\marginpar{Whole object of intellectual education is formation of logical
disposition}

\marginpar{False opposition of the logical and psychological}

No argument is needed to point out that the educator is concerned with
the logical in its practical and vital sense. Argument is perhaps needed
to show that the \emph{intellectual} (as distinct from the \emph{moral})
\emph{end of education is entirely and only the logical in this sense};
\emph{namely,
the formation of careful, alert, and thorough habits of thinking}. The
chief difficulty in the way of recognition of this principle is a false
conception of the relation between the psychological tendencies of an
individual and his logical achievements. If it be assumed---as it is so
frequently---that these have, intrinsically, nothing to do with each
other, then logical training is inevitably regarded as something foreign
and extraneous, something to be ingrafted upon the individual from
without, so that it is absurd to identify the object of education with
the development of logical power.

\marginpar{Opposing the \emph{natural} to the logical}

The conception that the psychology of individuals has no intrinsic
connections with logical methods and results is held, curiously enough,
by two opposing schools of educational theory. To one school, the
\emph{natural}\footnote{
Denoting whatever has to do with the natural constitution and functions
of an individual.
}
is primary and fundamental; and its tendency is to make little of
distinctly intellectual nurture. Its mottoes are freedom,
self-expression, individuality, spontaneity, play, interest, natural
unfolding, and so on. In its emphasis upon individual attitude and
activity, it sets slight store upon organized subject-matter, or the
material of study, and conceives \emph{method} to consist of various
devices for stimulating and evoking, in their natural order of growth,
the native potentialities of individuals.

\marginpar{Neglect of the innate logical resources}

\marginpar{Identification of logical with subject-matter, exclusively}

The other school estimates highly the value of the logical, but
conceives the natural tendency of individuals to be averse, or at least
indifferent, to logical achievement. It relies upon
\emph{subject-matter}---upon matter already defined and classified.
Method, then, has to do with the devices by which these characteristics
may be imported into a mind naturally reluctant and
rebellious.
Hence its mottoes are discipline, instruction, restraint, voluntary or
conscious effort, the necessity of tasks, and so on. From this point of
view studies, rather than attitudes and habits, embody the logical
factor in education. The mind becomes logical only by learning to
conform to an external subject-matter. To produce this conformity, the
study should first be analyzed (by text-book or teacher) into its
logical elements; then each of these elements should be defined;
finally, all of the elements should be arranged in series or classes
according to logical formulæ or general principles. Then the pupil
learns the definitions one by one; and progressively adding one to
another builds up the logical system, and thereby is himself gradually
imbued, from without, with logical quality.

\marginpar{Illustration from geography,}

This description will gain meaning through an illustration. Suppose the
subject is geography. The first thing is to give its definition, marking
it off from every other subject. Then the various abstract terms upon
which depends the scientific development of the science are stated and
defined one by one---pole, equator, ecliptic, zone,---from the simpler
units to the more complex which are formed out of them; then the more
concrete elements are taken in similar series: continent, island, coast,
promontory, cape, isthmus, peninsula, ocean, lake, coast, gulf, bay, and
so on. In acquiring this material, the mind is supposed not only to gain
important information, but, by accommodating itself to ready-made
logical definitions, generalizations, and classifications, gradually to
acquire logical habits.

\marginpar{from drawing}

This type of method has been applied to every subject taught in the
schools---reading, writing, music, physics, grammar, arithmetic. Drawing
for
example,
has been taught on the theory that since all pictorial representation is
a matter of combining straight and curved lines, the simplest procedure
is to have the pupil acquire the ability first to draw straight lines in
various positions (horizontal, perpendicular, diagonals at various
angles), then typical curves; and finally, to combine straight and
curved lines in various permutations to construct actual pictures. This
seemed to give the ideal "logical" method, beginning with analysis into
elements, and then proceeding in regular order to more and more complex
syntheses, each element being defined when used, and thereby clearly
understood.

\marginpar{Formal method}

Even when this method in its extreme form is not followed, few schools
(especially of the middle or upper elementary grades) are free from an
exaggerated attention to forms supposedly employed by the pupil if he
gets his result logically. It is thought that there are certain steps
arranged in a certain order, which express preëminently an understanding
of the subject, and the pupil is made to "analyze" his procedure into
these steps, \emph{i.e.} to learn a certain routine formula of
statement. While this method is usually at its height in grammar and
arithmetic, it invades also history and even literature, which are then
reduced, under plea of intellectual training, to "outlines," diagrams,
and schemes of division and subdivision. In memorizing this simulated
cut and dried copy of the logic of an adult, the child generally is
induced to stultify his own subtle and vital logical movement. The
adoption by teachers of this misconception of logical method has
probably done more than anything else to bring pedagogy into disrepute;
for to many persons "pedagogy" means precisely a set of mechanical,
self-conscious devices for replacing by
some
cast-iron external scheme the personal mental movement of the
individual.

\marginpar{Reaction toward lack of form and method}

A reaction inevitably occurs from the poor results that accrue from
these professedly "logical" methods. Lack of interest in study, habits
of inattention and procrastination, positive aversion to intellectual
application, dependence upon sheer memorizing and mechanical routine
with only a modicum of understanding by the pupil of what he is about,
show that the theory of logical definition, division, gradation, and
system does not work out practically as it is theoretically supposed to
work. The consequent disposition---as in every reaction---is to go to
the opposite extreme. The "logical" is thought to be wholly artificial
and extraneous; teacher and pupil alike are to turn their backs upon it,
and to work toward the expression of existing aptitudes and tastes.
Emphasis upon natural tendencies and powers as the only possible
starting-point of development is indeed wholesome. But the reaction is
false, and hence misleading, in what it ignores and denies: the presence
of genuinely intellectual factors in existing powers and interests.

\marginpar{Logic of subject-matter is logic of adult or trained mind}

What is conventionally termed logical (namely, the logical from the
standpoint of subject-matter) represents in truth the logic of the
trained adult mind. Ability to divide a subject, to define its elements,
and to group them into classes according to general principles
represents logical capacity at its best point reached \emph{after}
thorough training. The mind that habitually exhibits skill in divisions,
definitions, generalizations, and systematic recapitulations no longer
needs training in logical methods. But it is absurd to suppose that a
mind which needs training because it cannot perform these
operations
can begin where the expert mind stops. \emph{The logical from the
standpoint of subject-matter represents the goal, the last term of
training, not the point of departure.}

\marginpar{The immature mind has its own logic}

Hence, the \emph{psychological} and the \emph{logical} represent the two
ends of the same movement

In truth, the mind at every stage of development has its own logic. The
error of the notion that by appeal to spontaneous tendencies and by
multiplication of materials we may completely dismiss logical
considerations, lies in overlooking how large a part curiosity,
inference, experimenting, and testing already play in the pupil's life.
Therefore it underestimates the \emph{intellectual} factor in the more
spontaneous play and work of individuals---the factor that alone is
truly educative. Any teacher who is alive to the modes of thought
naturally operative in the experience of the normal child will have no
difficulty in avoiding the identification of the logical with a
ready-made organization of subject-matter, as well as the notion that
the only way to escape this error is to pay no attention to logical
considerations. Such a teacher will have no difficulty in seeing that
the real problem of intellectual education is the transformation of
natural powers into expert, tested powers: the transformation of more or
less casual curiosity and sporadic suggestion into attitudes of alert,
cautious, and thorough inquiry. He will see that the
\emph{psychological} and the \emph{logical}, instead of being opposed to
each other (or even independent of each other), are connected \emph{as
the earlier and the later stages in one continuous process of normal
growth}. The natural or psychological activities, even when not
consciously controlled by logical considerations, have their own
intellectual function and integrity; conscious and deliberate skill in
thinking, when it is achieved, makes habitual or second nature. The
first is already logical in spirit; the last, in presenting an ingrained
disposition
and attitude, is then as \emph{psychological} (as personal) as any
caprice or chance impulse could be.

\section{Discipline and Freedom}

\marginpar{True and false notions of discipline}

Discipline of mind is thus, in truth, a result rather than a cause. Any
mind is disciplined in a subject in which independent intellectual
initiative and control have been achieved. Discipline represents
original native endowment turned, through gradual exercise, into
effective power. So far as a mind is disciplined, control of method in a
given subject has been attained so that the mind is able to manage
itself independently without external tutelage. The aim of education is
precisely to develop intelligence of this independent and effective
type---a \emph{disciplined mind}. Discipline is positive and
constructive.

\marginpar{Discipline as drill}

Discipline, however, is frequently regarded as something negative---as a
painfully disagreeable forcing of mind away from channels congenial to
it into channels of constraint, a process grievous at the time but
necessary as preparation for a more or less remote future. Discipline is
then generally identified with drill; and drill is conceived after the
mechanical analogy of driving, by unremitting blows, a foreign substance
into a resistant material; or is imaged after the analogy of the
mechanical routine by which raw recruits are trained to a soldierly
bearing and habits that are naturally wholly foreign to their
possessors. Training of this latter sort, whether it be called
discipline or not, is not mental discipline. Its aim and result are not
\emph{habits of thinking}, but uniform \emph{external modes of action}.
By failing to ask what he means by discipline, many a teacher is misled
into supposing that he is
developing
mental force and efficiency by methods which in fact restrict and deaden
intellectual activity, and which tend to create mechanical routine, or
mental passivity and servility.

\marginpar{As independent power or freedom}

\marginpar{Freedom and external spontaneity}

When discipline is conceived in intellectual terms (as the habitual
power of effective mental attack), it is identified with freedom in its
true sense. For freedom of mind means mental power capable of
independent exercise, emancipated from the leading strings of others,
not mere unhindered external operation. When spontaneity or naturalness
is identified with more or less casual discharge of transitory impulses,
the tendency of the educator is to supply a multitude of stimuli in
order that spontaneous activity may be kept up. All sorts of interesting
materials, equipments, tools, modes of activity, are provided in order
that there may be no flagging of free self-expression. This method
overlooks some of the essential conditions of the attainment of genuine
freedom.

\marginpar{Some obstacle necessary for thought}

(\emph{a}) Direct immediate discharge or expression of an impulsive
tendency is fatal to thinking. Only when the impulse is to some extent
checked and thrown back upon itself does reflection ensue. It is,
indeed, a stupid error to suppose that arbitrary tasks must be imposed
from without in order to furnish the factor of perplexity and difficulty
which is the necessary cue to thought. Every vital activity of any depth
and range inevitably meets obstacles in the course of its effort to
realize itself---a fact that renders the search for artificial or
external problems quite superfluous. The difficulties that present
themselves within the development of an experience are, however, to be
cherished by the educator, not minimized, for they are the natural
stimuli
to reflective inquiry. Freedom does not consist in keeping up
uninterrupted and unimpeded external activity, but is something achieved
through conquering, by personal reflection, a way out of the
difficulties that prevent an immediate overflow and a spontaneous
success.

\marginpar{Intellectual factors are \emph{natural}}

(\emph{b}) The method that emphasizes the psychological and natural, but
yet fails to see what an important part of the natural tendencies is
constituted at every period of growth by curiosity, inference, and the
desire to test, cannot secure a \emph{natural development}. In natural
growth each successive stage of activity prepares unconsciously, but
thoroughly, the conditions for the manifestation of the next stage---as
in the cycle of a plant's growth. There is no ground for assuming that
"thinking" is a special, isolated natural tendency that will bloom
inevitably in due season simply because various sense and motor
activities have been freely manifested before; or because observation,
memory, imagination, and manual skill have been previously exercised
without thought. Only when thinking is constantly employed in using the
senses and muscles for the guidance and application of observations and
movements, is the way prepared for subsequent higher types of thinking.

\marginpar{Genesis of thought contemporaneous with genesis of any human mental
activity}

At present, the notion is current that childhood is almost entirely
unreflective---a period of mere sensory, motor, and memory development,
while adolescence suddenly brings the manifestation of thought and
reason.

Adolescence is not, however, a synonym for magic. Doubtless youth should
bring with it an enlargement of the horizon of childhood, a
susceptibility to larger concerns and issues, a more generous and a more
general standpoint toward nature and social life. This development
affords an opportunity for thinking of a more
comprehensive
and abstract type than has previously obtained. But thinking itself
remains just what it has been all the time: a matter of following up and
testing the conclusions suggested by the facts and events of life.
Thinking begins as soon as the baby who has lost the ball that he is
playing with begins to foresee the possibility of something not yet
existing---its recovery; and begins to forecast steps toward the
realization of this possibility, and, by experimentation, to guide his
acts by his ideas and thereby also test the ideas. Only by making the
most of the thought-factor, already active in the experiences of
childhood, is there any promise or warrant for the emergence of superior
reflective power at adolescence, or at any later period.

\marginpar{Fixation of bad mental habits}

(\emph{c}) In any case \emph{positive habits are being formed}: if not
habits of careful looking into things, then habits of hasty, heedless,
impatient glancing over the surface; if not habits of consecutively
following up the suggestions that occur, then habits of haphazard,
grasshopper-like guessing; if not habits of suspending judgment till
inferences have been tested by the examination of evidence, then habits
of credulity alternating with flippant incredulity, belief or unbelief
being based, in either case, upon whim, emotion, or accidental
circumstances. The only way to achieve traits of carefulness,
thoroughness, and continuity (traits that are, as we have seen, the
elements of the "logical") is by exercising these traits from the
beginning, and by seeing to it that conditions call for their exercise.

\marginpar{Genuine freedom is intellectual, not external}

Genuine freedom, in short, is intellectual; it rests in the trained
\emph{power of thought}, in ability to "turn things over," to look at
matters deliberately, to judge whether the amount and kind of evidence
requisite for
decision
is at hand, and if not, to tell where and how to seek such evidence. If
a man's actions are not guided by thoughtful conclusions, then they are
guided by inconsiderate impulse, unbalanced appetite, caprice, or the
circumstances of the moment. To cultivate unhindered, unreflective
external activity is to foster enslavement, for it leaves the person at
the mercy of appetite, sense, and
circumstance.

\part{Logical Considerations}

\chapter{The Analysis of a Complete Act of Thought}

\marginpar{Object of Part Two}

After a brief consideration in the first chapter of the nature of
reflective thinking, we turned, in the second, to the need for its
training. Then we took up the resources, the difficulties, and the aim
of its training. The purpose of this discussion was to set before the
student the general problem of the training of mind. The purport of the
second part, upon which we are now entering, is giving a fuller
statement of the nature and normal growth of thinking, preparatory to
considering in the concluding part the special problems that arise in
connection with its education.

In this chapter we shall make an analysis of the process of thinking
into its steps or elementary constituents, basing the analysis upon
descriptions of a number of extremely simple, but genuine, cases of
reflective
experience.\footnote{
These are taken, almost verbatim, from the class papers of students.
}

\marginpar{A simple case of practical deliberation}

1. "The other day when I was down town on 16th Street a clock caught my
eye. I saw that the hands pointed to 12.20. This suggested that I had an
engagement at 124th Street, at one o'clock. I reasoned
that
as it had taken me an hour to come down on a surface car, I should
probably be twenty minutes late if I returned the same way. I might save
twenty minutes by a subway express. But was there a station near? If
not, I might lose more than twenty minutes in looking for one. Then I
thought of the elevated, and I saw there was such a line within two
blocks. But where was the station? If it were several blocks above or
below the street I was on, I should lose time instead of gaining it. My
mind went back to the subway express as quicker than the elevated;
furthermore, I remembered that it went nearer than the elevated to the
part of 124th Street I wished to reach, so that time would be saved at
the end of the journey. I concluded in favor of the subway, and reached
my destination by one o'clock."

\marginpar{A simple case of reflection upon an observation}

2. "Projecting nearly horizontally from the upper deck of the ferryboat
on which I daily cross the river, is a long white pole, bearing a gilded
ball at its tip. It suggested a flagpole when I first saw it; its color,
shape, and gilded ball agreed with this idea, and these reasons seemed
to justify me in this belief. But soon difficulties presented
themselves. The pole was nearly horizontal, an unusual position for a
flagpole; in the next place, there was no pulley, ring, or cord by which
to attach a flag; finally, there were elsewhere two vertical staffs from
which flags were occasionally flown. It seemed probable that the pole
was not there for flag-flying.

"I then tried to imagine all possible purposes of such a pole, and to
consider for which of these it was best suited: (\emph{a}) Possibly it
was an ornament. But as all the ferryboats and even the tugboats carried
like
poles,
this hypothesis was rejected. (\emph{b}) Possibly it was the terminal of
a wireless telegraph. But the same considerations made this improbable.
Besides, the more natural place for such a terminal would be the highest
part of the boat, on top of the pilot house. (\emph{c}) Its purpose
might be to point out the direction in which the boat is moving.

"In support of this conclusion, I discovered that the pole was lower
than the pilot house, so that the steersman could easily see it.
Moreover, the tip was enough higher than the base, so that, from the
pilot's position, it must appear to project far out in front of the
boat. Moreover, the pilot being near the front of the boat, he would
need some such guide as to its direction. Tugboats would also need poles
for such a purpose. This hypothesis was so much more probable than the
others that I accepted it. I formed the conclusion that the pole was set
up for the purpose of showing the pilot the direction in which the boat
pointed, to enable him to steer correctly."

\marginpar{A simple case of reflection involving experiment}

3. "In washing tumblers in hot soapsuds and placing them mouth downward
on a plate, bubbles appeared on the outside of the mouth of the tumblers
and then went inside. Why? The presence of bubbles suggests air, which I
note must come from inside the tumbler. I see that the soapy water on
the plate prevents escape of the air save as it may be caught in
bubbles. But why should air leave the tumbler? There was no substance
entering to force it out. It must have expanded. It expands by increase
of heat or by decrease of pressure, or by both. Could the air have
become heated after the tumbler was taken from the hot suds? Clearly not
the air that was already
entangled
in the water. If heated air was the cause, cold air must have entered in
transferring the tumblers from the suds to the plate. I test to see if
this supposition is true by taking several more tumblers out. Some I
shake so as to make sure of entrapping cold air in them. Some I take out
holding mouth downward in order to prevent cold air from entering.
Bubbles appear on the outside of every one of the former and on none of
the latter. I must be right in my inference. Air from the outside must
have been expanded by the heat of the tumbler, which explains the
appearance of the bubbles on the outside.

"But why do they then go inside? Cold contracts. The tumbler cooled and
also the air inside it. Tension was removed, and hence bubbles appeared
inside. To be sure of this, I test by placing a cup of ice on the
tumbler while the bubbles are still forming outside. They soon reverse."

\marginpar{The three cases form a series}

These three cases have been purposely selected so as to form a series
from the more rudimentary to more complicated cases of reflection. The
first illustrates the kind of thinking done by every one during the
day's business, in which neither the data, nor the ways of dealing with
them, take one outside the limits of everyday experience. The last
furnishes a case in which neither problem nor mode of solution would
have been likely to occur except to one with some prior scientific
training. The second case forms a natural transition; its materials lie
well within the bounds of everyday, unspecialized experience; but the
problem, instead of being directly involved in the person's business,
arises indirectly out of his activity, and accordingly appeals to a
somewhat theoretic and impartial interest.
We
shall deal, in a later chapter, with the evolution of abstract thinking
out of that which is relatively practical and direct; here we are
concerned only with the common elements found in all the types.

\marginpar{Five distinct steps in reflection}

Upon examination, each instance reveals, more or less clearly, five
logically distinct steps: (\emph{i}) a felt difficulty; (\emph{ii}) its
location and definition; (\emph{iii}) suggestion of possible solution;
(\emph{iv}) development by reasoning of the bearings of the suggestion;
(\emph{v}) further observation and experiment leading to its acceptance
or rejection; that is, the conclusion of belief or disbelief.

\marginpar{1. The occurrence of a difficulty}

\marginpar{(\emph{a}) in the lack of adaptation of means to end}

1. The first and second steps frequently fuse into one. The difficulty
may be felt with sufficient definiteness as to set the mind at once
speculating upon its probable solution, or an undefined uneasiness and
shock may come first, leading only later to definite attempt to find out
what is the matter. Whether the two steps are distinct or blended, there
is the factor emphasized in our original account of
reflection---\emph{viz.} the perplexity or problem. In the first of the
three cases cited, the difficulty resides in the conflict between
conditions at hand and a desired and intended result, between an end and
the means for reaching it. The purpose of keeping an engagement at a
certain time, and the existing hour taken in connection with the
location, are not congruous. The object of thinking is to introduce
congruity between the two. The given conditions cannot themselves be
altered; time will not go backward nor will the distance between 16th
Street and 124th Street shorten itself. The problem is \emph{the
discovery of intervening terms which when inserted between the remoter
end and the given means will harmonize them with each
other}.

\marginpar{(\emph{b}) in identifying the character of an object}

In the second case, the difficulty experienced is the incompatibility of
a suggested and (temporarily) accepted belief that the pole is a
flagpole, with certain other facts. Suppose we symbolize the qualities
that suggest \emph{flagpole} by the letters \emph{a}, \emph{b},
\emph{c}; those that oppose this suggestion by the letters \emph{p},
\emph{q}, \emph{r}. There is, of course, nothing inconsistent in the
qualities themselves; but in pulling the mind to different and
incongruous conclusions they conflict---hence the problem. Here the
object is the discovery of some object (\emph{O}), of which \emph{a},
\emph{b}, \emph{c}, and \emph{p}, \emph{q}, \emph{r}, may all be
appropriate traits---just as, in our first case, it is to discover a
course of action which will combine existing conditions and a remoter
result in a single whole. The method of solution is also the same:
discovery of intermediate qualities (the position of the pilot house, of
the pole, the need of an index to the boat's direction) symbolized by
\emph{d}, \emph{g}, \emph{l}, \emph{o}, which bind together otherwise
incompatible traits.

\marginpar{(\emph{c}) in explaining an unexpected event}

In the third case, an observer trained to the idea of natural laws or
uniformities finds something odd or exceptional in the behavior of the
bubbles. The problem is to reduce the apparent anomalies to instances of
well-established laws. Here the method of solution is also to seek for
intermediary terms which will connect, by regular linkage, the seemingly
extraordinary movements of the bubbles with the conditions known to
follow from processes supposed to be operative.

\marginpar{2. Definition of the difficulty}

2. As already noted, the first two steps, the feeling of a discrepancy,
or difficulty, and the acts of observation that serve to define the
character of the difficulty may, in a given instance, telescope
together. In cases of striking novelty or unusual perplexity, the
difficulty, however, is likely to present itself at first as a shock,
as
emotional disturbance, as a more or less vague feeling of the
unexpected, of something queer, strange, funny, or disconcerting. In
such instances, there are necessary observations deliberately calculated
to bring to light just what is the trouble, or to make clear the
specific character of the problem. In large measure, the existence or
non-existence of this step makes the difference between reflection
proper, or safeguarded \emph{critical} inference and uncontrolled
thinking. Where sufficient pains to locate the difficulty are not taken,
suggestions for its resolution must be more or less random. Imagine a
doctor called in to prescribe for a patient. The patient tells him some
things that are wrong; his experienced eye, at a glance, takes in other
signs of a certain disease. But if he permits the suggestion of this
special disease to take possession prematurely of his mind, to become an
accepted conclusion, his scientific thinking is by that much cut short.
A large part of his technique, as a skilled practitioner, is to prevent
the acceptance of the first suggestions that arise; even, indeed, to
postpone the occurrence of any very definite suggestion till the
trouble---the nature of the problem---has been thoroughly explored. In
the case of a physician this proceeding is known as diagnosis, but a
similar inspection is required in every novel and complicated situation
to prevent rushing to a conclusion. The essence of critical thinking is
suspended judgment; and the essence of this suspense is inquiry to
determine the nature of the problem before proceeding to attempts at its
solution. This, more than any other thing, transforms mere inference
into tested inference, suggested conclusions into proof.

\marginpar{3. Occurrence of a suggested explanation or possible solution}

3. The third factor is suggestion. The situation
in
which the perplexity occurs calls up something not present to the
senses: the present location, the thought of subway or elevated train;
the stick before the eyes, the idea of a flagpole, an ornament, an
apparatus for wireless telegraphy; the soap bubbles, the law of
expansion of bodies through heat and of their contraction through cold.
(\emph{a}) Suggestion is the very heart of inference; it involves going
from what is present to something absent. Hence, it is more or less
speculative, adventurous. Since inference goes beyond what is actually
present, it involves a leap, a jump, the propriety of which cannot be
absolutely warranted in advance, no matter what precautions be taken.
Its control is indirect, on the one hand, involving the formation of
habits of mind which are at once enterprising and cautious; and on the
other hand, involving the selection and arrangement of the particular
facts upon perception of which suggestion issues. (\emph{b}) The
suggested conclusion so far as it is not accepted but only tentatively
entertained constitutes an idea. Synonyms for this are
\emph{supposition}, \emph{conjecture}, \emph{guess}, \emph{hypothesis},
and (in elaborate cases) \emph{theory}. Since suspended belief, or the
postponement of a final conclusion pending further evidence, depends
partly upon the presence of rival conjectures as to the best course to
pursue or the probable explanation to favor, \emph{cultivation of a
variety of alternative suggestions} is an important factor in good
thinking.

\marginpar{4. The rational elaboration of an idea}

4. The process of developing the bearings---or, as they are more
technically termed, the \emph{implications}---of any idea with respect
to any problem, is termed
\emph{reasoning}.\footnote{
This term is sometimes extended to denote the entire reflective
process---just as \emph{inference} (which in the sense of \emph{test} is
best reserved for the third step) is sometimes used in the same broad
sense. But \emph{reasoning} (or \emph{ratiocination}) seems to be
peculiarly adapted to express what the older writers called the
"notional" or "dialectic" process of developing the meaning of a given
idea.
}
As an idea is inferred from given facts, so
reasoning
sets out from an idea. The \emph{idea} of elevated road is developed
into the idea of difficulty of locating station, length of time occupied
on the journey, distance of station at the other end from place to be
reached. In the second case, the implication of a flagpole is seen to be
a vertical position; of a wireless apparatus, location on a high part of
the ship and, moreover, absence from every casual tugboat; while the
idea of index to direction in which the boat moves, when developed, is
found to cover all the details of the case.

Reasoning has the same effect upon a suggested solution as more intimate
and extensive observation has upon the original problem. Acceptance of
the suggestion in its first form is prevented by looking into it more
thoroughly. Conjectures that seem plausible at first sight are often
found unfit or even absurd when their full consequences are traced out.
Even when reasoning out the bearings of a supposition does not lead to
rejection, it develops the idea into a form in which it is more apposite
to the problem. Only when, for example, the conjecture that a pole was
an index-pole had been thought out into its bearings could its
particular applicability to the case in hand be judged. Suggestions at
first seemingly remote and wild are frequently so transformed by being
elaborated into what follows from them as to become apt and fruitful.
The development of an idea through reasoning helps at least to supply
the intervening or intermediate terms that link together into a
consistent whole apparently discrepant extremes (\emph{ante}, p.
72).

\marginpar{5. Corroboration of an idea and formation of a concluding belief}

5. The concluding and conclusive step is some kind of \emph{experimental
corroboration}, or verification, of the conjectural idea. Reasoning
shows that \emph{if} the idea be adopted, certain consequences follow.
So far the conclusion is hypothetical or conditional. If we look and
find present all the conditions demanded by the theory, and if we find
the characteristic traits called for by rival alternatives to be
lacking, the tendency to believe, to accept, is almost irresistible.
Sometimes direct observation furnishes corroboration, as in the case of
the pole on the boat. In other cases, as in that of the bubbles,
experiment is required; that is, \emph{conditions are deliberately
arranged in accord with the requirements of an idea or hypothesis to see
if the results theoretically indicated by the idea actually occur}. If
it is found that the experimental results agree with the theoretical, or
rationally deduced, results, and if there is reason to believe that
\emph{only} the conditions in question would yield such results, the
confirmation is so strong as to induce a conclusion---at least until
contrary facts shall indicate the advisability of its revision.

\marginpar{Thinking comes between observations at the beginning and at the end}

Observation exists at the beginning and again at the end of the process:
at the beginning, to determine more definitely and precisely the nature
of the difficulty to be dealt with; at the end, to test the value of
some hypothetically entertained conclusion. Between those two termini of
observation, we find the more distinctively \emph{mental} aspects of the
entire thought-cycle: (\emph{i}) inference, the suggestion of an
explanation or solution; and (\emph{ii}) reasoning, the development of
the bearings and implications of the suggestion. Reasoning requires some
experimental observation to confirm it, while experiment can be
economically and fruitfully conducted
only
on the basis of an idea that has been tentatively developed by
reasoning.

\marginpar{The trained mind one that judges the extent of each step advisable in a
given situation}

The disciplined, or logically trained, mind---the aim of the educative
process---is the mind able to judge how far each of these steps needs to
be carried in any particular situation. No cast-iron rules can be laid
down. Each case has to be dealt with as it arises, on the basis of its
importance and of the context in which it occurs. To take too much pains
in one case is as foolish---as illogical---as to take too little in
another. At one extreme, almost any conclusion that insures prompt and
unified action may be better than any long delayed conclusion; while at
the other, decision may have to be postponed for a long period---perhaps
for a lifetime. The trained mind is the one that best grasps the degree
of observation, forming of ideas, reasoning, and experimental testing
required in any special case, and that profits the most, in future
thinking, by mistakes made in the past. What is important is that the
mind should be sensitive to problems and skilled in methods of attack
and
solution.

\chapter{Systematic Inference: Induction and Deduction}

\section{The Double Movement of Reflection}

\marginpar{Back and forth between facts and meanings}

The characteristic outcome of thinking we saw to be the organization of
facts and conditions which, just as they stand, are isolated,
fragmentary, and discrepant, the organization being effected through the
introduction of connecting links, or middle terms. The facts as they
stand are the data, the raw material of reflection; their lack of
coherence perplexes and stimulates to reflection. There follows the
suggestion of some meaning which, \emph{if} it can be substantiated,
will give a whole in which various fragmentary and seemingly
incompatible data find their proper place. The meaning suggested
supplies a mental platform, an intellectual point of view, from which to
note and define the data more carefully, to seek for additional
observations, and to institute, experimentally, changed conditions.

\marginpar{Inductive and deductive}

There is thus a double movement in all reflection: a movement from the
given partial and confused data to a suggested comprehensive (or
inclusive) entire situation; and back from this suggested whole---which
as suggested is a \emph{meaning}, an idea---to the particular facts, so
as to connect these with one another and with additional facts to which
the suggestion has directed attention. Roughly speaking, the first of
these
movements
is inductive; the second deductive. A complete act of thought involves
both---it involves, that is, a fruitful interaction of observed (or
recollected) particular considerations and of inclusive and far-reaching
(general) meanings.

\marginpar{Hurry \emph{versus} caution}

This double movement \emph{to} and \emph{from} a meaning may occur,
however, in a casual, uncritical way, or in a cautious and regulated
manner. To think means, in any case, to bridge a gap in experience, to
bind together facts or deeds otherwise isolated. But we may make only a
hurried jump from one consideration to another, allowing our aversion to
mental disquietude to override the gaps; or, we may insist upon noting
the road traveled in making connections. We may, in short, accept
readily any suggestion that seems plausible; or we may hunt out
additional factors, new difficulties, to see whether the suggested
conclusion really ends the matter. The latter method involves definite
formulation of the connecting links; the statement of a principle, or,
in logical phrase, the use of a universal. If we thus formulate the
whole situation, the original data are transformed into premises of
reasoning; the final belief is a logical or \emph{rational} conclusion,
not a mere \emph{de facto} termination.

\marginpar{Continuity of relationship the mark of the latter}

The importance of \emph{connections binding isolated items into a
coherent single whole} is embodied in all the phrases that denote the
relation of premises and conclusions to each other. (1) The premises are
called grounds, foundations, bases, and are said to underlie, uphold,
support the conclusion. (2) We "descend" from the premises to the
conclusion, and "ascend" or "mount" in the opposite direction---as a
river may be continuously traced from source to sea or vice versa. So
the conclusion springs, flows, or is drawn from its
premises.
(3) The conclusion---as the word itself implies---closes, shuts in,
locks up together the various factors stated in the premises. We say
that the premises "contain" the conclusion, and that the conclusion
"contains" the premises, thereby marking our sense of the inclusive and
comprehensive unity in which the elements of reasoning are bound tightly
together.\footnote{
See Vailati, \emph{Journal of Philosophy, Psychology, and Scientific
Methods}, Vol. V, No. 12.}
Systematic inference, in short, means the \emph{recognition of definite
relations of interdependence between considerations previously
unorganized and disconnected, this recognition being brought about by
the discovery and insertion of new facts and properties}.

\marginpar{Scientific induction and deduction}

This more systematic thinking is, however, like the cruder forms in its
double movement, the movement \emph{toward} the suggestion or hypothesis
and the movement \emph{back} to facts. The difference is in the greater
conscious care with which each phase of the process is performed.
\emph{The conditions under which suggestions are allowed to spring up
and develop are regulated.} Hasty acceptance of any idea that is
plausible, that seems to solve the difficulty, is changed into a
conditional acceptance pending further inquiry. The idea is accepted as
a \emph{working hypothesis}, as something to guide investigation and
bring to light new facts, not as a final conclusion. When pains are
taken to make each aspect of the movement as accurate as possible, the
movement toward building up the idea is known as \emph{inductive
discovery} (\emph{induction}, for short); the movement toward
developing, applying, and testing, as \emph{deductive proof}
(\emph{deduction}, for short).

\marginpar{Particular and universal}

While induction moves from fragmentary details
(or
particulars) to a connected view of a situation (universal), deduction
begins with the latter and works back again to particulars, connecting
them and binding them together. The inductive movement is toward
\emph{discovery} of a binding principle; the deductive toward its
\emph{testing}---confirming, refuting, modifying it on the basis of its
capacity to interpret isolated details into a unified experience. So far
as we conduct each of these processes in the light of the other, we get
valid discovery or verified critical thinking.

\marginpar{Illustration from everyday experience}

A commonplace illustration may enforce the points of this formula. A man
who has left his rooms in order finds them upon his return in a state of
confusion, articles being scattered at random. Automatically, the notion
comes to his mind that burglary would account for the disorder. He has
not seen the burglars; their presence is not a fact of observation, but
is a thought, an idea. Moreover, the man has no special burglars in
mind; it is the \emph{relation}, the meaning of burglary---something
general---that comes to mind. The state of his room is perceived and is
particular, definite,---exactly as it is; burglars are inferred, and
have a general status. The state of the room is a \emph{fact}, certain
and speaking for itself; the presence of burglars is a possible
\emph{meaning} which may explain the facts.

\marginpar{of induction,}

So far there is an inductive tendency, suggested by particular and
present facts. In the same inductive way, it occurs to him that his
children are mischievous, and that they may have thrown the things
about. This rival hypothesis (or conditional principle of explanation)
prevents him from dogmatically accepting the first suggestion. Judgment
is held in suspense and a positive conclusion
postponed.

\marginpar{of deduction}

Then deductive movement begins. Further observations, recollections,
reasonings are conducted on the basis of a development of the ideas
suggested: \emph{if} burglars were responsible, such and such things
would have happened; articles of value would be missing. Here the man is
going from a general principle or relation to special features that
accompany it, to particulars,---not back, however, merely to the
original particulars (which would be fruitless or take him in a circle),
but to new details, the actual discovery or nondiscovery of which will
test the principle. The man turns to a box of valuables; some things are
gone; some, however, are still there. Perhaps he has himself removed the
missing articles, but has forgotten it. His experiment is not a decisive
test. He thinks of the silver in the sideboard---the children would not
have taken that nor would he absent-mindedly have changed its place. He
looks; all the solid ware is gone. The conception of burglars is
confirmed; examination of windows and doors shows that they have been
tampered with. Belief culminates; the original isolated facts have been
woven into a coherent fabric. The idea first suggested (inductively) has
been employed to reason out hypothetically certain additional
particulars not yet experienced, that \emph{ought} to be there, if the
suggestion is correct. Then new acts of observation have shown that the
particulars theoretically called for are present, and by this process
the hypothesis is strengthened, corroborated. This moving back and forth
between the observed facts and the conditional idea is kept up till a
coherent experience of an object is substituted for the experience of
conflicting details---or else the whole matter is given up as a bad job.

\marginpar{Science is the same operations carefully performed}

Sciences exemplify similar attitudes and
operations,
but with a higher degree of elaboration of the instruments of caution,
exactness and thoroughness. This greater elaboration brings about
specialization, an accurate marking off of various types of problems
from one another, and a corresponding segregation and classification of
the materials of experience associated with each type of problem. We
shall devote the remainder of this chapter to a consideration of the
devices by which the discovery, the development, and the testing of
meanings are scientifically carried on.

\section{Guidance of the Inductive Movement}

\marginpar{Guidance is indirect}

Control of the formation of suggestion is necessarily \emph{indirect},
not direct; imperfect, not perfect. Just because all discovery, all
apprehension involving thought of the new, goes from the known, the
present, to the unknown and absent, no rules can be stated that will
guarantee correct inference. Just what is suggested to a person in a
given situation depends upon his native constitution (his originality,
his genius), temperament, the prevalent direction of his interests, his
early environment, the general tenor of his past experiences, his
special training, the things that have recently occupied him
continuously or vividly, and so on; to some extent even upon an
accidental conjunction of present circumstances. These matters, so far
as they lie in the past or in external conditions, clearly escape
regulation. A suggestion simply does or does not occur; this or that
suggestion just happens, occurs, springs up. If, however, prior
experience and training have developed an attitude of patience in a
condition of doubt, a capacity for suspended judgment, and a liking for
inquiry, \emph{indirect} control of the course of suggestions is
possible.
The individual may return upon, revise, restate, enlarge, and analyze
\emph{the facts out of which suggestion springs}. Inductive methods, in
the technical sense, all have to do with regulating the conditions under
which \emph{observation, memory, and the acceptance of the testimony of
others} (\emph{the operations supplying the raw data}) proceed.

\marginpar{Method of indirect regulation}

Given the facts \emph{A B C D} on one side and certain individual habits
on the other, suggestion occurs automatically. But if the facts \emph{A
B C D} are carefully looked into and thereby resolved into the facts
\emph{A' B'' R S}, a suggestion will automatically present itself
different from that called up by the facts in their first form. To
inventory the facts, to describe exactly and minutely their respective
traits, to magnify artificially those that are obscure and feeble, to
reduce artificially those that are so conspicuous and glaring as to be
distracting,---these are ways of modifying the facts that exercise
suggestive force, and thereby indirectly guiding the formation of
suggested inferences.

\marginpar{Illustration from diagnosis}

Consider, for example, how a physician makes his diagnosis---his
inductive interpretation. If he is scientifically trained, he
suspends---postpones---reaching a conclusion in order that he may not be
led by superficial occurrences into a snap judgment. Certain conspicuous
phenomena may forcibly suggest typhoid, but he avoids a conclusion, or
even any strong preference for this or that conclusion until he has
greatly (\emph{i}) \emph{enlarged} the scope of his data, and
(\emph{ii}) rendered them more \emph{minute}. He not only questions the
patient as to his feelings and as to his acts prior to the disease, but
by various manipulations with his hands (and with instruments made for
the purpose) brings to light a large number of facts of which the
patient is quite unaware. The state of
temperature,
respiration, and heart-action is accurately noted, and their
fluctuations from time to time are exactly recorded. Until this
examination has worked \emph{out} toward a wider collection and
\emph{in} toward a minuter scrutiny of details, inference is deferred.

\marginpar{Summary: definition of scientific induction}

Scientific induction means, in short, \emph{all the processes by which
the observing and amassing of data are regulated with a view to
facilitating the formation of explanatory conceptions and theories}.
These devices are all directed toward selecting the precise facts to
which weight and significance shall attach in forming suggestions or
ideas. Specifically, this selective determination involves devices of
(1) elimination by analysis of what is likely to be misleading and
irrelevant, (2) emphasis of the important by collection and comparison
of cases, (3) deliberate construction of data by experimental variation.

\marginpar{Elimination of irrelevant meanings}

(1) It is a common saying that one must learn to discriminate between
observed facts and judgments based upon them. Taken literally, such
advice cannot be carried out; in every observed thing there is---if the
thing have any meaning at all---some consolidation of meaning with what
is sensibly and physically present, such that, if this were entirely
excluded, what is left would have no sense. A says: "I saw my brother."
The term \emph{brother}, however, involves a relation that cannot be
sensibly or physically observed; it is inferential in status. If A
contents himself with saying, "I saw a man," the factor of
classification, of intellectual reference, is less complex, but still
exists. If, as a last resort, A were to say, "Anyway, I saw a colored
object," some relationship, though more rudimentary and undefined, still
subsists. Theoretically, it is possible that
no
object was there, only an unusual mode of nerve stimulation. None the
less, the advice to discriminate what is observed from what is inferred
is sound practical advice. Its working import is that one should
eliminate or exclude \emph{those} inferences as to which experience has
shown that there is greatest liability to error. This, of course, is a
relative matter. Under ordinary circumstances no reasonable doubt would
attach to the observation, "I see my brother"; it would be pedantic and
silly to resolve this recognition back into a more elementary form.
Under other circumstances it might be a perfectly genuine question as to
whether A saw even a colored \emph{thing}, or whether the color was due
to a stimulation of the sensory optical apparatus (like "seeing stars"
upon a blow) or to a disordered circulation. In general, the scientific
man is one who knows that he is likely to be hurried to a conclusion,
and that part of this precipitancy is due to certain habits which tend
to make him "read" certain meanings into the situation that confronts
him, so that he must be on the lookout against errors arising from his
interests, habits, and current preconceptions.

\marginpar{The technique of conclusion}

The technique of scientific inquiry thus consists in various processes
that tend to exclude over-hasty "reading in" of meanings; devices that
aim to give a purely "objective" unbiased rendering of the data to be
interpreted. Flushed cheeks usually mean heightened temperature;
paleness means lowered temperature. The clinical thermometer records
automatically the actual temperature and hence checks up the habitual
associations that might lead to error in a given case. All the
instrumentalities of observation---the various -meters and -graphs and
-scopes---fill a
part
of their scientific rôle in helping to eliminate meanings supplied
because of habit, prejudice, the strong momentary preoccupation of
excitement and anticipation, and by the vogue of existing theories.
Photographs, phonographs, kymographs, actinographs, seismographs,
plethysmographs, and the like, moreover, give records that are
permanent, so that they can be employed by different persons, and by the
same person in different states of mind, \emph{i.e.} under the influence
of varying expectations and dominant beliefs. Thus purely personal
prepossessions (due to habit, to desire, to after-effects of recent
experience) may be largely eliminated. In ordinary language, the facts
are \emph{objectively}, rather than \emph{subjectively}, determined. In
this way tendencies to premature interpretation are held in check.

\marginpar{Collection of instances}

(2) Another important method of control consists in the multiplication
of cases or instances. If I doubt whether a certain handful gives a fair
sample, or representative, for purposes of judging value, of a whole
carload of grain, I take a number of handfuls from various parts of the
car and compare them. If they agree in quality, well and good; if they
disagree, we try to get enough samples so that when they are thoroughly
mixed the result will be a fair basis for an evaluation. This
illustration represents roughly the value of that aspect of scientific
control in induction which insists upon multiplying observations instead
of basing the conclusion upon one or a few cases.

\marginpar{This method not the whole of induction}

So prominent, indeed, is this aspect of inductive method that it is
frequently treated as the whole of induction. It is supposed that all
inductive inference is based upon collecting and comparing a number of
like cases. But in fact such comparison and collection is
a
secondary development within the process of securing a correct
conclusion in some single case. If a man infers from a single sample of
grain as to the grade of wheat of the car as a whole, it is induction
and, under certain circumstances, a \emph{sound} induction; other cases
are resorted to simply for the sake of rendering that induction more
guarded, and more probably correct. In like fashion, the reasoning that
led up to the burglary idea in the instance already cited (p. 83) was
inductive, though there was but one single case examined. The
particulars upon which the general meaning (or relation) of burglary was
grounded were simply the sum total of the unlike items and qualities
that made up the one case examined. Had this case presented very great
obscurities and difficulties, recourse might \emph{then} have been had
to examination of a number of similar cases. But this comparison would
not make inductive a process which was not previously of that character;
it would only render induction more wary and adequate. \emph{The object
of bringing into consideration a multitude of cases is to facilitate the
selection of the evidential or significant features upon which to base
inference in some single case.}

\marginpar{Contrast as important as likeness}

Accordingly, points of \emph{unlikeness} are as important as points of
\emph{likeness} among the cases examined. \emph{Comparison}, without
\emph{contrast}, does not amount to anything logically. In the degree in
which other cases observed or remembered merely duplicate the case in
question, we are no better off for purposes of inference than if we had
permitted our single original fact to dictate a conclusion. In the case
of the various samples of grain, it is the fact that the samples are
unlike, at least in the part of the carload from which they are taken,
that is important. Were it not for this unlikeness, their
likeness
in quality would be of no avail in assisting
inference.\footnote{
In terms of the phrases used in logical treatises, the so-called
"methods of agreement" (comparison) and "difference" (contrast) must
accompany each other or constitute a "joint method" in order to be of
logical use.
}
If we are endeavoring to get a child to regulate his conclusions about
the germination of a seed by taking into account a number of instances,
very little is gained if the conditions in all these instances closely
approximate one another. But if one seed is placed in pure sand, another
in loam, and another on blotting-paper, and if in each case there are
two conditions, one with and another without moisture, the unlike
factors tend to throw into relief the factors that are significant (or
"essential") for reaching a conclusion. Unless, in short, the observer
takes care to have the differences in the observed cases as extreme as
conditions allow, and unless he notes unlikenesses as carefully as
likenesses, he has no way of determining the evidential force of the
data that confront him.

\marginpar{Importance of exceptions and contrary cases}

Another way of bringing out this importance of unlikeness is the
emphasis put by the scientist upon \emph{negative} cases---upon
instances which it would seem ought to fall into line but which as
matter of fact do not. Anomalies, exceptions, things which agree in most
respects but disagree in some crucial point, are so important that many
of the devices of scientific technique are designed purely to detect,
record, and impress upon memory contrasting cases. Darwin remarked that
so easy is it to pass over cases that oppose a favorite generalization,
that he had made it a habit not merely to hunt for contrary instances,
but also to write down any exception he noted or thought of---as
otherwise it was almost sure to be
forgotten.

\section{Experimental Variation of Conditions}

\marginpar{Experiment the typical method of introducing contrast factors}

We have already trenched upon this factor of inductive method, the one
that is the most important of all wherever it is feasible.
Theoretically, one sample case \emph{of the right kind} will be as good
a basis for an inference as a thousand cases; but cases of the "right
kind" rarely turn up spontaneously. We have to search for them, and we
may have to \emph{make} them. If we take cases just as we find
them---whether one case or many cases---they contain much that is
irrelevant to the problem in hand, while much that is relevant is
obscure, hidden. The object of experimentation is the
\emph{construction, by regular steps taken on the basis of a plan
thought out in advance, of a typical, crucial case}, a case formed with
express reference to throwing light on the difficulty in question. All
inductive methods rest (as already stated, p. 85) upon regulation of the
conditions of observation and memory; experiment is simply the most
adequate regulation possible of these conditions. We try to make the
observation such that every factor entering into it, together with the
mode and the amount of its operation, may be open to recognition. Such
making of observations constitutes experiment.

\marginpar{Three advantages of experiment}

Such observations have many and obvious advantages over
observations---no matter how extensive---with respect to which we simply
wait for an event to happen or an object to present itself. Experiment
overcomes the defects due to (\emph{a}) the \emph{rarity}, (\emph{b})
the \emph{subtlety} and minuteness (or the violence), and (\emph{c}) the
rigid \emph{fixity} of facts as we ordinarily experience them. The
following quotations from Jevons's \emph{Elementary Lessons in Logic}
bring out all these points:

(\emph{i}) "We might have to wait years or centuries to
meet
accidentally with facts which we can readily produce at any moment in a
laboratory; and it is probable that most of the chemical substances now
known, and many excessively useful products would never have been
discovered at all by waiting till nature presented them spontaneously to
our observation."

This quotation refers to the infrequency or rarity of certain facts of
nature, even very important ones. The passage then goes on to speak of
the minuteness of many phenomena which makes them escape ordinary
experience:

(\emph{ii}) "Electricity doubtless operates in every particle of matter,
perhaps at every moment of time; and even the ancients could not but
notice its action in the loadstone, in lightning, in the Aurora
Borealis, or in a piece of rubbed amber. But in lightning electricity
was too intense and dangerous; in the other cases it was too feeble to
be properly understood. The science of electricity and magnetism could
only advance by getting regular supplies of electricity from the common
electric machine or the galvanic battery and by making powerful
electromagnets. Most, if not all, the effects which electricity produces
must go on in nature, but altogether too obscurely for observation."

Jevons then deals with the fact that, under ordinary conditions of
experience, phenomena which can be understood only by seeing them under
varying conditions are presented in a fixed and uniform way.

(\emph{iii}) "Thus carbonic acid is only met in the form of a gas,
proceeding from the combustion of carbon; but when exposed to extreme
pressure and cold, it is condensed into a liquid, and may even be
converted into a snowlike solid substance. Many other gases have
in
like manner been liquefied or solidified, and there is reason to believe
that every substance is capable of taking all three forms of solid,
liquid, and gas, if only the conditions of temperature and pressure can
be sufficiently varied. Mere observation of nature would have led us, on
the contrary, to suppose that nearly all substances were fixed in one
condition only, and could not be converted from solid into liquid and
from liquid into gas."

Many volumes would be required to describe in detail all the methods
that investigators have developed in various subjects for analyzing and
restating the facts of ordinary experience so that we may escape from
capricious and routine suggestions, and may get the facts in such a form
and in such a light (or context) that exact and far-reaching
explanations may be suggested in place of vague and limited ones. But
these various devices of inductive inquiry all have one goal in view:
the indirect regulation of the function of suggestion, or formation of
ideas; and, in the main, they will be found to reduce to some
combination of the three types of selecting and arranging subject-matter
just described.

\section{Guidance of the Deductive Movement}

\marginpar{Value of deduction for guiding induction}

Before dealing directly with this topic, we must note that systematic
regulation of induction depends upon the possession of a body of general
principles that may be applied deductively to the examination or
construction of particular cases as they come up. If the physician does
not know the general laws of the physiology of the human body, he has
little way of telling what is either peculiarly significant or
peculiarly
exceptional in any particular case that he is called upon to treat. If
he knows the laws of circulation, digestion, and respiration, he can
deduce the conditions that should normally be found in a given case.
These considerations give a base line from which the deviations and
abnormalities of a particular case may be measured. In this way,
\emph{the nature of the problem at hand is located and defined}.
Attention is not wasted upon features which though conspicuous have
nothing to do with the case; it is concentrated upon just those traits
which are out of the way and hence require explanation. A question well
put is half answered; \emph{i.e.} a difficulty clearly apprehended is
likely to suggest its own solution,---while a vague and miscellaneous
perception of the problem leads to groping and fumbling. Deductive
systems are necessary in order to put the question in a fruitful form.

\marginpar{"Reasoning a thing out"}

The control of the origin and development of hypotheses by deduction
does not cease, however, with locating the problem. Ideas as they first
present themselves are inchoate and incomplete. \emph{Deduction is their
elaboration into fullness and completeness of meaning} (see p. 76). The
phenomena which the physician isolates from the total mass of facts that
exist in front of him suggest, we will say, typhoid fever. Now this
conception of typhoid fever is one that is capable of development.
\emph{If} there is typhoid, \emph{wherever} there is typhoid, there are
certain results, certain characteristic symptoms. By going over mentally
the full bearing of the concept of typhoid, the scientist is instructed
as to further phenomena to be found. Its development gives him an
instrument of inquiry, of observation and experimentation. He can go to
work deliberately to see
whether
the case presents those features that it should have if the supposition
is valid. The deduced results form a basis for comparison with observed
results. Except where there is a system of principles capable of being
elaborated by theoretical reasoning, the process of testing (or proof)
of a hypothesis is incomplete and haphazard.

\marginpar{Such reasoning implies systematized knowledge,}

These considerations indicate the method by which the deductive movement
is guided. Deduction requires a system of allied ideas which may be
translated into one another by regular or graded steps. The question is
whether the facts that confront us can be identified as typhoid fever.
To all appearances, there is a great gap between them and typhoid. But
if we can, by some method of substitutions, go through a series of
intermediary terms (see p. 72), the gap may, after all, be easily
bridged. Typhoid may mean \emph{p} which in turn means \emph{o}, which
means \emph{n} which means \emph{m}, which is very similar to the data
selected as the key to the problem.

\marginpar{or definition and classification}

One of the chief objects of science is to provide for every typical
branch of subject-matter a set of meanings and principles so closely
interknit that any one implies some other according to definite
conditions, which under certain other conditions implies another, and so
on. In this way, various substitutions of equivalents are possible, and
reasoning can trace out, without having recourse to specific
observations, very remote consequences of any suggested principle.
Definition, general formulæ, and classification are the devices by which
the fixation and elaboration of a meaning into its detailed
ramifications are carried on. They are not ends in themselves---as they
are frequently regarded even in elementary education---but
instrumentalities for
facilitating
the development of a conception into the form where its applicability to
given facts may best be
tested.\footnote{These processes are further discussed in Chapter IX.}

\marginpar{The final control of deduction}

The final test of deduction lies in experimental observation.
Elaboration by reasoning may make a suggested idea very rich and very
plausible, but it will not settle the validity of that idea. Only if
facts can be observed (by methods either of collection or of
experimentation), that agree in detail and without exception with the
deduced results, are we justified in accepting the deduction as giving a
valid conclusion. Thinking, in short, must end as well as begin in the
domain of concrete observations, if it is to be complete thinking. And
the ultimate educative value of all deductive processes is measured by
the degree to which they become working tools in the creation and
development of new experiences.

\section{Some Educational Bearings of the Discussion}

\marginpar{Educational counterparts of false logical theories}

\marginpar{Isolation of "facts"}

Some of the points of the foregoing logical analysis may be clinched by
a consideration of their educational implications, especially with
reference to certain practices that grow out of a false separation by
which each is thought to be independent of the other and complete in
itself. (\emph{i}) In some school subjects, or at all events in some
topics or in some lessons, the pupils are immersed in details; their
minds are loaded with disconnected items (whether gleaned by observation
and memory, or accepted on hearsay and authority). Induction is treated
as beginning and ending with the amassing of facts, of particular
isolated pieces of information. That these items are educative only as
suggesting a view of some larger situation in which
the
particulars are included and thereby accounted for, is ignored. In
object lessons in elementary education and in laboratory instruction in
higher education, the subject is often so treated that the student fails
to "see the forest on account of the trees." Things and their qualities
are retailed and detailed, without reference to a more general character
which they stand for and mean. Or, in the laboratory, the student
becomes engrossed in the processes of manipulation,---irrespective of
the reason for their performance, without recognizing a typical problem
for the solution of which they afford the appropriate method. Only
deduction brings out and emphasizes consecutive relationships, and only
when \emph{relationships} are held in view does learning become more
than a miscellaneous scrap-bag.

\marginpar{Failure to follow up by reasoning}

(\emph{ii}) Again, the mind is allowed to hurry on to a vague notion of
the whole of which the fragmentary facts are portions, without any
attempt to become conscious of \emph{how} they are bound together as
parts of this whole. The student feels that "in a general way," as we
say, the facts of the history or geography lesson are related thus and
so; but "in a general way" here stands only for "in a vague way,"
somehow or other, with no clear recognition of just how.

The pupil is encouraged to form, on the basis of the particular facts, a
general notion, a conception of how they stand related; but no pains are
taken to make the student follow up the notion, to elaborate it and see
just what its bearings are upon the case in hand and upon similar cases.
The inductive inference, the guess, is formed by the student; if it
happens to be correct, it is at once accepted by the teacher; or if it
is false, it is rejected. If any amplification of the idea occurs, it
is
quite likely carried through by the teacher, who thereby assumes the
responsibility for its intellectual development. But a complete, an
integral, act of thought requires that the person making the suggestion
(the guess) be responsible also for reasoning out its bearings upon the
problem in hand; that he develop the suggestion at least enough to
indicate the ways in which it applies to and accounts for the specific
data of the case. Too often when a recitation does not consist in simply
testing the ability of the student to display some form of technical
skill, or to repeat facts and principles accepted on the authority of
text-book or lecturer, the teacher goes to the opposite extreme; and
after calling out the spontaneous reflections of the pupils, their
guesses or ideas about the matter, merely accepts or rejects them,
assuming himself the responsibility for their elaboration. In this way,
the function of suggestion and of interpretation is excited, but it is
not directed and trained. Induction is stimulated but is not carried
over into the \emph{reasoning} phase necessary to complete it.

In other subjects and topics, the deductive phase is isolated, and is
treated as if it were complete in itself. This false isolation may show
itself in either (and both) of two points; namely, at the beginning or
at the end of the resort to general intellectual procedure.

\marginpar{Isolation of deduction by commencing with it}

(\emph{iii}) Beginning with definitions, rules, general principles,
classifications, and the like, is a common form of the first error. This
method has been such a uniform object of attack on the part of all
educational reformers that it is not necessary to dwell upon it further
than to note that the mistake is, logically, due to the attempt to
introduce deductive considerations without first making acquaintance
with the particular facts
that
create a need for the generalizing rational devices. Unfortunately, the
reformer sometimes carries his objection too far, or rather locates it
in the wrong place. He is led into a tirade against \emph{all}
definition, all systematization, all use of general principles, instead
of confining himself to pointing out their futility and their deadness
when not properly motivated by familiarity with concrete experiences.

\marginpar{Isolation of deduction from direction of new observations}

(\emph{iv}) The isolation of deduction is seen, at the other end,
wherever there is failure to clinch and test the results of the general
reasoning processes by application to new concrete cases. The final
point of the deductive devices lies in their use in assimilating and
comprehending individual cases. No one understands a general principle
fully---no matter how adequately he can demonstrate it, to say nothing
of repeating it---till he can employ it in the mastery of new
situations, which, if they \emph{are} new, differ in manifestation from
the cases used in reaching the generalization. Too often the text-book
or teacher is contented with a series of somewhat perfunctory examples
and illustrations, and the student is not forced to carry the principle
that he has formulated over into further cases of his own experience. In
so far, the principle is inert and dead.

\marginpar{Lack of provision for experimentation}

(\emph{v}) It is only a variation upon this same theme to say that every
complete act of reflective inquiry makes provision for
experimentation---for testing suggested and accepted principles by
employing them for the active construction of new cases, in which new
qualities emerge. Only slowly do our schools accommodate themselves to
the general advance of scientific method. From the scientific side, it
is demonstrated that effective and integral thinking is possible only
where the
experimental
method in some form is used. Some recognition of this principle is
evinced in higher institutions of learning, colleges and high schools.
But in elementary education, it is still assumed, for the most part,
that the pupil's natural range of observations, supplemented by what he
accepts on hearsay, is adequate for intellectual growth. Of course it is
not necessary that laboratories shall be introduced under that name,
much less that elaborate apparatus be secured; but the entire scientific
history of humanity demonstrates that the conditions for complete mental
activity will not be obtained till adequate provision is made for the
carrying on of activities that actually modify physical conditions, and
that books, pictures, and even objects that are passively observed but
not manipulated do not furnish the provision
required.

\chapter{Judgment: The Interpretation of Facts}

\section{The Three Factors of Judging}

\marginpar{Good judgment}

A man of good judgment in a given set of affairs is a man in so far
educated, trained, whatever may be his literacy. And if our schools turn
out their pupils in that attitude of mind which is conducive to good
judgment in any department of affairs in which the pupils are placed,
they have done more than if they sent out their pupils merely possessed
of vast stores of information, or high degrees of skill in specialized
branches. To know what is \emph{good} judgment we need first to know
what judgment is.

\marginpar{Judgment and inference}

That there is an intimate connection between judgment and inference is
obvious enough. The aim of inference is to terminate itself in an
adequate judgment of a situation, and the course of inference goes on
through a series of partial and tentative judgments. What are these
units, these terms of inference when we examine them on their own
account? Their significant traits may be readily gathered from a
consideration of the operations to which the word \emph{judgment} was
originally applied: namely, the authoritative decision of matters in
legal controversy---the procedure of the \emph{judge on the bench}.
There are three such features: (1) a controversy, consisting of opposite
claims regarding the same objective situation; (2) a process of defining
and elaborating these claims and of sifting the facts adduced
to
support them; (3) a final decision, or sentence, closing the particular
matter in dispute and also serving as a rule or principle for deciding
future cases.

\marginpar{Uncertainty the antecedent of judgment}

1. Unless there is something doubtful, the situation is read off at a
glance; it is taken in on sight, \emph{i.e.} there is merely
apprehension, perception, recognition, not judgment. If the matter is
wholly doubtful, if it is dark and obscure throughout, there is a blind
mystery and again no judgment occurs. But if it suggests, however
vaguely, different meanings, rival possible interpretations, there is
some \emph{point at issue}, some \emph{matter at stake}. Doubt takes the
form of dispute, controversy; different sides compete for a conclusion
in their favor. Cases brought to trial before a judge illustrate neatly
and unambiguously this strife of alternative interpretations; but any
case of trying to clear up intellectually a doubtful situation
exemplifies the same traits. A moving blur catches our eye in the
distance; we ask ourselves: "What is it? Is it a cloud of whirling dust?
a tree waving its branches? a man signaling to us?" Something in the
total situation suggests each of these possible meanings. Only one of
them can possibly be sound; perhaps none of them is appropriate; yet
\emph{some} meaning the thing in question surely has. Which of the
alternative suggested meanings has the rightful claim? What does the
perception really mean? How is it to be interpreted, estimated,
appraised, placed? Every judgment proceeds from some such situation.

\marginpar{Judgment defines the issue,}

2. The hearing of the controversy, the trial, \emph{i.e.} the weighing
of alternative claims, divides into two branches, either of which, in a
given case, may be more conspicuous than the other. In the consideration
of a legal dispute, these two branches are sifting the evidence
and
selecting the rules that are applicable; they are "the facts" and "the
law" of the case. In judgment they are (\emph{a}) the determination of
the data that are important in the given case (compare the inductive
movement); and (\emph{b}) the elaboration of the conceptions or meanings
suggested by the crude data (compare the deductive movement). (\emph{a})
What portions or aspects of the situation are significant in controlling
the formation of the interpretation? (\emph{b}) Just what is the full
meaning and bearing of the conception that is used as a method of
interpretation? These questions are strictly correlative; the answer to
each depends upon the answer to the other. We may, however, for
convenience, consider them separately.

\marginpar{(\emph{a}) by selecting what facts are evidence}

(\emph{a}) In every actual occurrence, there are many details which are
part of the total occurrence, but which nevertheless are not significant
in relation to the point at issue. All parts of an experience are
equally present, but they are very far from being of equal value as
signs or as evidences. Nor is there any tag or label on any trait
saying: "This is important," or "This is trivial." Nor is intensity, or
vividness or conspicuousness, a safe measure of indicative and proving
value. The glaring thing may be totally insignificant in this particular
situation, and the key to the understanding of the whole matter may be
modest or hidden (compare p. 74). Features that are not significant are
distracting; they proffer their claims to be regarded as clues and cues
to interpretation, while traits that are significant do not appear on
the surface at all. Hence, judgment is required \emph{even in reference}
to the situation or event that is present to the senses; elimination or
rejection, selection, discovery, or bringing to light must take
place.
Till we have reached a final conclusion, rejection and selection must be
tentative or conditional. We select the things that we hope or trust are
cues to meaning. But if they do not suggest a situation that accepts and
includes them (see p. 81), we reconstitute our data, the facts of the
case; for we mean, intellectually, by the facts of the case \emph{those
traits that are used as evidence in reaching a conclusion or forming a
decision}.

\marginpar{Expertness in selecting evidence}

No hard and fast rules for this operation of selecting and rejecting, or
fixing upon the facts, can be given. It all comes back, as we say, to
the good judgment, the good sense, of the one judging. To be a good
judge is to have a sense of the relative indicative or signifying values
of the various features of the perplexing situation; to know what to let
go as of no account; what to eliminate as irrelevant; what to retain as
conducive to outcome; what to emphasize as a clue to the
difficulty.\footnote{ Compare what was said about \emph{analysis}. }
This power in ordinary matters we call \emph{knack}, \emph{tact},
\emph{cleverness}; in more important affairs, \emph{insight},
\emph{discernment}. In part it is instinctive or inborn; but it also
represents the funded outcome of long familiarity with like operations
in the past. Possession of this ability to seize what is evidential or
significant and to let the rest go is the mark of the expert, the
connoisseur, the \emph{judge}, in any matter.

\marginpar{Intuitive judgments}

Mill cites the following case, which is worth noting as an instance of
the extreme delicacy and accuracy to which may be developed this power
of sizing up the significant factors of a situation. "A Scotch
manufacturer procured from England, at a high rate of wages, a working
dyer, famous for producing very fine colors, with the view of teaching
to his other workmen the
same
skill. The workman came; but his method of proportioning the
ingredients, in which lay the secret of the effects he produced, was by
taking them up in handfuls, while the common method was to weigh them.
The manufacturer sought to make him turn his handling system into an
equivalent weighing system, that the general principles of his peculiar
mode of proceeding might be ascertained. This, however, the man found
himself quite unable to do, and could therefore impart his own skill to
nobody. He had, from individual cases of his own experience, established
a connection in his mind between fine effects of color and tactual
perceptions in handling his dyeing materials; and from these perceptions
he could, in any particular case, \emph{infer the means to be employed}
and the effects which would be produced." Long brooding over conditions,
intimate contact associated with keen interest, thorough absorption in a
multiplicity of allied experiences, tend to bring about those judgments
which we then call intuitive; but they are true judgments because they
are based on intelligent selection and estimation, with the solution of
a problem as the controlling standard. Possession of this capacity makes
the difference between the artist and the intellectual bungler.

Such is judging ability, in its completest form, as to the data of the
decision to be reached. But in any case there is a certain feeling along
for the way to be followed; a constant tentative picking out of certain
qualities to see what emphasis upon them would lead to; a willingness to
hold final selection in suspense; and to reject the factors entirely or
relegate them to a different position in the evidential scheme if other
features yield more solvent suggestions. Alertness, flexibility,
curiosity
are the essentials; dogmatism, rigidity, prejudice, caprice, arising
from routine, passion, and flippancy are fatal.

\marginpar{(\emph{b}) To decide an issue, the appropriate principles must also be
selected}

(\emph{b}) This selection of data is, of course, for the sake of
controlling the \emph{development and elaboration of the suggested
meaning in the light of which they are to be interpreted} (compare p.
76). An evolution of conceptions thus goes on simultaneously with the
determination of the facts; one possible meaning after another is held
before the mind, considered in relation to the data to which it is
applied, is developed into its more detailed bearings upon the data, is
dropped or tentatively accepted and used. We do not approach any problem
with a wholly naïve or virgin mind; we approach it with certain acquired
habitual modes of understanding, with a certain store of previously
evolved meanings, or at least of experiences from which meanings may be
educed. If the circumstances are such that a habitual response is called
directly into play, there is an immediate grasp of meaning. If the habit
is checked, and inhibited from easy application, a possible meaning for
the facts in question presents itself. No hard and fast rules decide
whether a meaning suggested is the right and proper meaning to follow
up. The individual's own good (or bad) judgment is the guide. There is
no label on any given idea or principle which says automatically, "Use
me in this situation"---as the magic cakes of Alice in Wonderland were
inscribed "Eat me." The thinker has to decide, to choose; and there is
always a risk, so that the prudent thinker selects warily, subject, that
is, to confirmation or frustration by later events. If one is not able
to estimate wisely what is relevant to the interpretation of a given
perplexing or doubtful issue, it
avails
little that arduous learning has built up a large stock of concepts. For
learning is not wisdom; information does not guarantee good judgment.
Memory may provide an antiseptic refrigerator in which to store a stock
of meanings for future use, but judgment selects and adopts the one used
in a given emergency---and without an emergency (some crisis, slight or
great) there is no call for judgment. No conception, even if it is
carefully and firmly established in the abstract, can at first safely be
more than a \emph{candidate} for the office of interpreter. Only greater
success than that of its rivals in clarifying dark spots, untying hard
knots, reconciling discrepancies, can elect it or prove it a valid idea
for the given situation.

\marginpar{Judging terminates in a \emph{decision} or statement}

3. The judgment when formed is a \emph{decision}; it closes (or
concludes) the question at issue. This determination not only settles
that particular case, but it helps fix a rule or method for deciding
similar matters in the future; as the sentence of the judge on the bench
both terminates that dispute and also forms a precedent for future
decisions. If the interpretation settled upon is not controverted by
subsequent events, a presumption is built up in favor of similar
interpretation in other cases where the features are not so obviously
unlike as to make it inappropriate. In this way, principles of judging
are gradually built up; a certain manner of interpretation gets weight,
authority. In short, meanings get \emph{standardized}, they become
logical concepts (see below, p. 118).

\section{The Origin and Nature of Ideas}

\marginpar{Ideas are conjectures employed in judging}

This brings us to the question of \emph{ideas in relation to
judgments}.\footnote{
The term \emph{idea} is also used popularly to denote (\emph{a}) a mere
fancy, (\emph{b}) an accepted belief, and also (\emph{c}) judgment
itself. But \emph{logically} it denotes a certain \emph{factor} in
judgment, as explained in the text.
}
Something in an obscure situation
suggests
something else as its meaning. If this meaning is at once accepted,
there is no reflective thinking, no genuine judging. Thought is cut
short uncritically; dogmatic belief, with all its attending risks, takes
place. But if the meaning suggested is held \emph{in suspense}, pending
examination and inquiry, there is true judgment. We stop and think, we
\emph{de-fer} conclusion in order to \emph{in-fer} more thoroughly. In
this process of being only conditionally accepted, accepted only for
examination, \emph{meanings become ideas}. \emph{That is to say, an idea
is a meaning that is tentatively entertained, formed, and used with
reference to its fitness to decide a perplexing situation,---a meaning
used as a tool of judgment.}

\marginpar{Or tools of interpretation}

Let us recur to our instance of a blur in motion appearing at a
distance. We wonder what \emph{the thing is}, \emph{i.e.} what the
\emph{blur means}. A man waving his arms, a friend beckoning to us, are
suggested as possibilities. To accept at once either alternative is to
arrest judgment. But if we treat what is suggested as only a suggestion,
a supposition, a possibility, it becomes an idea, having the following
traits: (\emph{a}) As merely a suggestion, it is a conjecture, a guess,
which in cases of greater dignity we call a hypothesis or a theory. That
is to say, it is \emph{a possible but as yet doubtful mode of
interpretation}. (\emph{b}) Even though doubtful, it has an office to
perform; namely, that of directing inquiry and examination. If this blur
means a friend beckoning, then careful observation should show certain
other traits. If it is a man driving unruly cattle, certain other traits
should be found. Let us look and see if these traits are found. Taken
merely as a doubt, an idea would paralyze inquiry. Taken merely as a
certainty, it would
arrest
inquiry. Taken as a doubtful possibility, it affords a standpoint, a
platform, a method of inquiry.

\marginpar{Pseudo-ideas}

Ideas are not then genuine ideas unless they are tools in a reflective
examination which tends to solve a problem. Suppose it is a question of
having the pupil grasp \emph{the idea} of the sphericity of the earth.
This is different from teaching him its sphericity \emph{as a fact}. He
may be shown (or reminded of) a ball or a globe, and be told that the
earth is round like those things; he may then be made to repeat that
statement day after day till the shape of the earth and the shape of the
ball are welded together in his mind. But he has not thereby acquired
any idea of the earth's sphericity; at most, he has had a certain image
of a sphere and has finally managed to image the earth after the analogy
of his ball image. To grasp sphericity as an idea, the pupil must first
have realized certain perplexities or confusing features in observed
facts and have had the idea of spherical shape suggested to him as a
possible way of accounting for the phenomena in question. Only by use as
a method of interpreting data so as to give them fuller meaning does
sphericity become a genuine idea. There may be a vivid image and no
idea; or there may be a fleeting, obscure image and yet an idea, if that
image performs the function of instigating and directing the observation
and relation of facts.

\marginpar{Ideas furnish the only alternative to "hit or miss" methods}

Logical ideas are like keys which are shaping with reference to opening
a lock. Pike, separated by a glass partition from the fish upon which
they ordinarily prey, will---so it is said---butt their heads against
the glass until it is literally beaten into them that they cannot get at
their food. Animals learn (when they learn at all) by a "cut and try"
method; by doing at
random
first one thing and another thing and then preserving the things that
happen to succeed. Action directed consciously by ideas---by suggested
meanings accepted for the sake of experimenting with them---is the sole
alternative both to bull-headed stupidity and to learning bought from
that dear teacher---chance experience.

\marginpar{They are methods of indirect attack}

It is significant that many words for intelligence suggest the idea of
circuitous, evasive activity---often with a sort of intimation of even
moral obliquity. The bluff, hearty man goes straight (and stupidly, it
is implied) at some work. The intelligent man is cunning, shrewd
(crooked), wily, subtle, crafty, artful, designing---the idea of
indirection is
involved.\footnote{
See Ward, \emph{Psychic Factors of Civilization}, p. 153.
}
An idea is a method of evading, circumventing, or surmounting through
reflection obstacles that otherwise would have to be attacked by brute
force. But ideas may lose their intellectual quality as they are
habitually used. When a child was first learning to recognize, in some
hesitating suspense, cats, dogs, houses, marbles, trees, shoes, and
other objects, ideas---conscious and tentative meanings---intervened as
methods of identification. Now, as a rule, the thing and the meaning are
so completely fused that there is no judgment and no idea proper, but
only automatic recognition. On the other hand, things that are, as a
rule, directly apprehended and familiar become subjects of judgment when
they present themselves in unusual contexts: as forms, distances, sizes,
positions when we attempt to draw them; triangles, squares, and circles
when they turn up, not in connection with familiar toys, implements, and
utensils, but as problems in geometry.

\section{Analysis and Synthesis}

\marginpar{Judging clears up things: analysis}

Through judging confused data are cleared up, and seemingly incoherent
and disconnected facts brought together. Things may have a peculiar
feeling for us, they may make a certain indescribable impression upon
us; the thing may \emph{feel} round (that is, present a quality which we
afterwards define as round), an act may seem rude (or what we afterwards
classify as rude), and yet this quality may be lost, absorbed, blended
in the total value of the situation. Only as we need to use just that
aspect of the original situation as a tool of grasping something
perplexing or obscure in another situation, do we abstract or detach the
quality so that it becomes individualized. Only because we need to
characterize the shape of some new object or the moral quality of some
new act, does the element of roundness or rudeness in the old experience
detach itself, and stand out as a distinctive feature. If the element
thus selected clears up what is otherwise obscure in the new experience,
if it settles what is uncertain, it thereby itself gains in positiveness
and definiteness of meaning. This point will meet us again in the
following chapter; here we shall speak of the matter only as it bears
upon the questions of analysis and synthesis.

\marginpar{Mental analysis is not like physical division}

\marginpar{Misapprehension of analysis in education}

Even when it is definitely stated that intellectual and physical
analyses are different sorts of operations, intellectual analysis is
often treated after the analogy of physical; as if it were the breaking
up of a whole into all its constituent parts in the mind instead of in
space. As nobody can possibly tell what breaking a whole into its parts
in the mind means, this conception leads to the further notion that
logical analysis is a mere enumeration and listing of all conceivable
qualities and
relations.
The influence upon education of this conception has been very
great.\footnote{
Thus arise all those falsely analytic methods in geography, reading,
writing, drawing, botany, arithmetic, which we have already considered
in another connection. (See p. 59.)
}
Every subject in the curriculum has passed through---or still remains
in---what may be called the phase of anatomical or morphological method:
the stage in which understanding the subject is thought to consist of
multiplying distinctions of quality, form, relation, and so on, and
attaching some name to each distinguished element. In normal growth,
specific properties are emphasized and so individualized only when they
serve to clear up a present difficulty. Only as they are involved in
judging some specific situation is there any motive or use for analyses,
\emph{i.e.} for emphasis upon some element or relation as peculiarly
significant.

\marginpar{Effects of premature formulation}

The same putting the cart before the horse, the product before the
process, is found in that overconscious formulation of methods of
procedure so current in elementary instruction. (See p. 60.) The method
that is employed in discovery, in reflective inquiry, cannot possibly be
identified with the method that emerges \emph{after} the discovery is
made. In the genuine operation of inference, the mind is in the attitude
of \emph{search}, of \emph{hunting}, of \emph{projection}, of
\emph{trying this and that}; when the conclusion is reached, the search
is at an end. The Greeks used to discuss: "How is learning (or inquiry)
possible? For either we know already what we are after, and then we do
not learn or inquire; or we do not know, and then we cannot inquire, for
we do not know what to look for." The dilemma is at least suggestive,
for it points to the true alternative: the use in inquiry of doubt, of
tentative suggestion, of
experimentation.
After we have reached the conclusion, a reconsideration of the steps of
the process to see what is helpful, what is harmful, what is merely
useless, will assist in dealing more promptly and efficaciously with
analogous problems in the future. In this way, more or less explicit
method is gradually built up. (Compare the earlier discussion on p. 62
of the psychological and the logical.)

\marginpar{Method comes before its formulation}

It is, however, a common assumption that unless the pupil from the
outset \emph{consciously recognizes and explicitly states} the method
logically implied in the result he is to reach, he will have \emph{no}
method, and his mind will work confusedly or anarchically; while if he
accompanies his performance with conscious statement of some form of
procedure (outline, topical analysis, list of headings and subheadings,
uniform formula) his mind is safeguarded and strengthened. As a matter
of fact, the development of \emph{an unconscious logical attitude and
habit} must come first. A conscious setting forth of the method
logically adapted for reaching an end is possible only after the result
has first been reached by more unconscious and tentative methods, while
it is valuable only when a review of the method that achieved success in
a given case will throw light upon a new, similar case. The ability to
fasten upon and single out (abstract, analyze) those features of one
experience which are logically best is hindered by premature insistence
upon their explicit formulation. It is repeated use that gives a
\emph{method} definiteness; and given this definiteness, precipitation
into formulated statement should follow naturally. But because teachers
find that the things which they themselves best understand are marked
off and defined in clear-cut ways, our schoolrooms are
pervaded
with the superstition that children are to begin with already
crystallized formulæ of method.

\marginpar{Judgment reveals the bearing or significance of facts: synthesis}

As analysis is conceived to be a sort of picking to pieces, so synthesis
is thought to be a sort of physical piecing together; and so imagined,
it also becomes a mystery. In fact, synthesis takes place wherever we
grasp the bearing of facts on a conclusion, or of a principle on facts.
As analysis is \emph{emphasis}, so synthesis is \emph{placing}; the one
causes the emphasized fact or property to stand out as significant; the
other gives what is selected its \emph{context}, or its connection with
what is signified. Every judgment is analytic in so far as it involves
discernment, discrimination, marking off the trivial from the important,
the irrelevant from what points to a conclusion; and it is synthetic in
so far as it leaves the mind with an inclusive situation within which
the selected facts are placed.

\marginpar{Analysis and synthesis are correlative}

Educational methods that pride themselves on being exclusively analytic
or exclusively synthetic are therefore (so far as they carry out their
boasts) incompatible with normal operations of judgment. Discussions
have taken place, for example, as to whether the teaching of geography
should be analytic or synthetic. The synthetic method is supposed to
begin with the partial, limited portion of the earth's surface already
familiar to the pupil, and then gradually piece on adjacent regions (the
county, the country, the continent, and so on) till an idea of the
entire globe is reached, or of the solar system that includes the globe.
The analytic method is supposed to begin with the physical whole, the
solar system or globe, and to work down through its constituent portions
till the immediate environment is reached. The underlying conceptions
are of physical wholes and
physical
parts. As matter of fact, we cannot assume that the portion of the earth
already familiar to the child is such a definite object, mentally, that
he can at once begin with it; his knowledge of it is misty and vague as
well as incomplete. Accordingly, mental progress will involve analysis
of it---emphasis of the features that are significant, so that they will
stand out clearly. Moreover, his own locality is not sharply marked off,
neatly bounded, and measured. His experience of it is already an
experience that involves sun, moon, and stars as parts of the scene he
surveys; it involves a changing horizon line as he moves about; that is,
even his more limited and local experience involves far-reaching factors
that take his imagination clear beyond his own street and village.
Connection, relationship with a larger whole, is already involved. But
his recognition of these relations is inadequate, vague, incorrect. He
needs to utilize the features of the local environment which are
understood to help clarify and enlarge his conceptions of the larger
geographical scene to which they belong. At the same time, not till he
has grasped the larger scene will many of even the commonest features of
his environment become intelligible. Analysis leads to synthesis; while
synthesis perfects analysis. As the pupil grows in comprehension of the
vast complicated earth in its setting in space, he also sees more
definitely the meaning of the familiar local details. This intimate
interaction between selective emphasis and interpretation of what is
selected is found wherever reflection proceeds normally. Hence the folly
of trying to set analysis and synthesis over against each
other.

\chapter{Meaning: Or Conceptions and Understanding}

\section{The Place of Meanings in Mental Life}

\marginpar{Meaning is central}

As in our discussion of judgment we were making more explicit what is
involved in inference, so in the discussion of meaning we are only
recurring to the central function of all reflection. For one thing to
\emph{mean}, \emph{signify}, \emph{betoken}, \emph{indicate}, or
\emph{point to}, another we saw at the outset to be the essential mark
of thinking (see p. 8). To find out what facts, just as they stand,
mean, is the object of all discovery; to find out what facts will carry
out, substantiate, support a given meaning, is the object of all
testing. When an inference reaches a satisfactory conclusion, we attain
a goal of meaning. The act of judging involves both the growth and the
application of meanings. In short, in this chapter we are not
introducing a new topic; we are only coming to closer quarters with what
hitherto has been constantly assumed. In the first section, we shall
consider the equivalence of meaning and understanding, and the two types
of understanding, direct and indirect.

\marginpar{I. {Meaning and Understanding}}

\marginpar{To understand is to grasp meaning}

If a person comes suddenly into your room and calls out "Paper," various
alternatives are possible. If you do not understand the English
language, there is simply a noise which may or may not act as a physical
stimulus
and irritant. But the noise is not an intellectual object; it does not
have intellectual value. (Compare above, p. 15.) To say that you do not
understand it and that it has no meaning are equivalents. If the cry is
the usual accompaniment of the delivery of the morning paper, the sound
will have meaning, intellectual content; you will understand it. Or if
you are eagerly awaiting the receipt of some important document, you may
assume that the cry means an announcement of its arrival. If (in the
third place) you understand the English language, but no context
suggests itself from your habits and expectations, the \emph{word} has
meaning, but not the whole event. You are then perplexed and incited to
think out, to hunt for, some explanation of the apparently meaningless
occurrence. If you find something that accounts for the performance, it
gets meaning; you come to understand it. As intelligent beings, we
presume the existence of meaning, and its absence is an anomaly. Hence,
if it should turn out that the person merely meant to inform you that
there was a scrap of paper on the sidewalk, or that paper existed
somewhere in the universe, you would think him crazy or yourself the
victim of a poor joke. To grasp a meaning, to understand, to identify a
thing in a situation in which it is important, are thus equivalent
terms; they express the nerves of our intellectual life. Without them
there is (\emph{a}) lack of intellectual content, or (\emph{b})
intellectual confusion and perplexity, or else (\emph{c}) intellectual
perversion---nonsense, insanity.

\marginpar{Knowledge and meaning}

All knowledge, all science, thus aims to grasp the meaning of objects
and events, and this process always consists in taking them out of their
apparent brute isolation as events, and finding them to be parts of
some
larger whole \emph{suggested by them}, which, in turn, \emph{accounts
for}, \emph{explains}, \emph{interprets them}; \emph{i.e.} renders them
significant. (Compare above, p. 75.) Suppose that a stone with peculiar
markings has been found. What do these scratches mean? So far as the
object forces the raising of this question, it is not understood; while
so far as the color and form that we see mean to us a stone, the object
is understood. It is such peculiar combinations of the understood and
the nonunderstood that provoke thought. If at the end of the inquiry,
the markings are decided to mean glacial scratches, obscure and
perplexing traits have been translated into meanings already understood:
namely, the moving and grinding power of large bodies of ice and the
friction thus induced of one rock upon another. Something already
understood in one situation has been transferred and applied to what is
strange and perplexing in another, and thereby the latter has become
plain and familiar, \emph{i.e.} understood. This summary illustration
discloses that our power to think effectively depends upon possession of
a capital fund of meanings which may be applied when desired. (Compare
what was said about deduction, p. 94.)

\marginpar{II. {Direct and Indirect Understanding}}

\marginpar{Direct and circuitous understanding}

In the above illustrations two types of grasping of meaning are
exemplified. When the English language is understood, the person grasps
at once the meaning of "paper." He may not, however, see any meaning or
sense in the performance as a whole. Similarly, the person identifies
the object on sight as a stone; there is no secret, no mystery, no
perplexity about that. But he does not understand the markings on it.
They
have
some meaning, but what is it? In one case, owing to familiar
acquaintance, the thing and its meaning, up to a certain point, are one.
In the other, the thing and its meaning are, temporarily at least,
sundered, and meaning has to be sought in order to understand the thing.
In one case understanding is direct, prompt, immediate; in the other, it
is roundabout and delayed.

\marginpar{Interaction of the two types}

Most languages have two sets of words to express these two modes of
understanding; one for the direct taking in or grasp of meaning, the
other for its circuitous apprehension, thus: \textgreek{γνωναι} and
\textgreek{ειδεναι} in Greek; \emph{noscere} and \emph{scire} in Latin;
\emph{kennen} and \emph{wissen} in German; \emph{connaître} and
\emph{savoir} in French; while in English to be \emph{acquainted with}
and to \emph{know of or about} have been suggested as
equivalents.\footnote{
James, \emph{Principles of Psychology}, vol. I, p. 221. To \emph{know}
and to \emph{know that} are perhaps more precise equivalents; compare "I
know him" and "I know that he has gone home." The former expresses a
fact simply; for the latter, evidence might be demanded and supplied.
}
Now our intellectual life consists of a peculiar interaction between
these two types of understanding. All judgment, all reflective
inference, presupposes some lack of understanding, a partial absence of
meaning. We reflect in order that we may get hold of the full and
adequate significance of what happens. Nevertheless, \emph{something}
must be already understood, the mind must be in possession of some
meaning which it has mastered, or else thinking is impossible. We think
in order to grasp meaning, but none the less every extension of
knowledge makes us aware of blind and opaque spots, where with less
knowledge all had seemed obvious and natural. A scientist brought into a
new district will find many things that he does not understand, where
the native savage
or
rustic will be wholly oblivious to any meanings beyond those directly
apparent. Some Indians brought to a large city remained stolid at the
sight of mechanical wonders of bridge, trolley, and telephone, but were
held spellbound by the sight of workmen climbing poles to repair wires.
Increase of the store of meanings makes us conscious of new problems,
while only through translation of the new perplexities into what is
already familiar and plain do we understand or solve these problems.
This is the constant spiral movement of knowledge.

\marginpar{Intellectual progress a rhythm}

Our progress in genuine knowledge always consists \emph{in part in the
discovery of something not understood in what had previously been taken
for granted as plain, obvious, matter-of-course, and in part in the use
of meanings that are directly grasped without question, as instruments
for getting hold of obscure, doubtful, and perplexing meanings}. No
object is so familiar, so obvious, so commonplace that it may not
unexpectedly present, in a novel situation, some problem, and thus
arouse reflection in order to understand it. No object or principle is
so strange, peculiar, or remote that it may not be dwelt upon till its
meaning becomes familiar---taken in on sight without reflection. We may
come to \emph{see}, \emph{perceive}, \emph{recognize}, \emph{grasp},
\emph{seize}, \emph{lay hold of} principles, laws, abstract
truths---\emph{i.e.} to understand their meaning in very immediate
fashion. Our intellectual progress consists, as has been said, in a
rhythm of direct understanding---technically called
\emph{ap}prehension---with indirect, mediated
understanding---technically called \emph{com}prehension.

\section{The Process of Acquiring Meanings}

\marginpar{Familiarity}

The first problem that comes up in connection with direct understanding
is how a store of directly
apprehensible
meanings is built up. How do we learn to view things on sight as
significant members of a situation, or as having, as a matter of course,
specific meanings? Our chief difficulty in answering this question lies
in the thoroughness with which the lesson of familiar things has been
learnt. Thought can more easily traverse an unexplored region than it
can undo what has been so thoroughly done as to be ingrained in
unconscious habit. We apprehend chairs, tables, books, trees, horses,
clouds, stars, rain, so promptly and directly that it is hard to realize
that as meanings they had once to be acquired,---the meanings are now so
much parts of the things themselves.

\marginpar{Confusion is prior to familiarity}

In an often quoted passage, Mr.\ James has said: "The baby, assailed by
eyes, ears, nose, skin, and entrails at once, feels it all as one great
blooming, buzzing
confusion."\footnote{
\emph{Principles of Psychology}, vol. I, p. 488.
}
Mr.\ James is speaking of a baby's world taken as a whole; the
description, however, is equally applicable to the way any new thing
strikes an adult, so far as the thing is really new and strange. To the
traditional "cat in a strange garret," everything is blurred and
confused; the wonted marks that label things so as to separate them from
one another are lacking. Foreign languages that we do not understand
always seem jabberings, babblings, in which it is impossible to fix a
definite, clear-cut, individualized group of sounds. The countryman in
the crowded city street, the landlubber at sea, the ignoramus in sport
at a contest between experts in a complicated game, are further
instances. Put an unexperienced man in a factory, and at first the work
seems to him a meaningless medley. All strangers of another race
proverbially look alike to the
visiting
foreigner. Only gross differences of size or color are perceived by an
outsider in a flock of sheep, each of which is perfectly individualized
to the shepherd. A diffusive blur and an indiscriminately shifting
suction characterize what we do not understand. The problem of the
acquisition of meaning by things, or (stated in another way) of forming
habits of simple apprehension, is thus the problem of introducing
(\emph{i}) \emph{definiteness} and \emph{distinction} and (\emph{ii})
\emph{consistency} or \emph{stability} of meaning into what is otherwise
vague and wavering.

\marginpar{Practical responses clarify confusion}

The acquisition of definiteness and of coherency (or constancy) of
meanings is derived primarily from practical activities. By rolling an
object, the child makes its roundness appreciable; by bouncing it, he
singles out its elasticity; by throwing it, he makes weight its
conspicuous distinctive factor. Not through the senses, but by means of
the reaction, the responsive adjustment, is the impression made
distinctive, and given a character marked off from other qualities that
call out unlike reactions. Children, for example, are usually quite slow
in apprehending differences of color. Differences from the standpoint of
the adult so glaring that it is impossible not to note them are
recognized and recalled with great difficulty. Doubtless they do not all
\emph{feel} alike, but there is no intellectual recognition of what
makes the difference. The redness or greenness or blueness of the object
does not tend to call out a reaction that is sufficiently peculiar to
give prominence or distinction to the color trait. Gradually, however,
certain characteristic habitual responses associate themselves with
certain things; the white becomes the sign, say, of milk and sugar, to
which the child reacts favorably; blue becomes the sign of a dress that
the child likes to wear, and so on: and
the
distinctive reactions tend to single out color qualities from other
things in which they had been submerged.

\marginpar{We identify by use or function}

Take another example. We have little difficulty in distinguishing from
one another rakes, hoes, plows and harrows, shovels and spades. Each has
its own associated characteristic use and function. We may have,
however, great difficulty in recalling the difference between serrate
and dentate, ovoid and obovoid, in the shapes and edges of leaves, or
between acids in \emph{ic} and in \emph{ous}. There is some difference;
but just what? Or, we know what the difference is; but which is which?
Variations in form, size, color, and arrangement of parts have much less
to do, and the uses, purposes, and functions of things and of their
parts much more to do, with distinctness of character and meaning than
we should be likely to think. What misleads us is the fact that the
qualities of form, size, color, and so on, are \emph{now} so distinct
that we fail to see that the problem is precisely to account for the way
in which they originally obtained their definiteness and
conspicuousness. So far as we sit passive before objects, they are not
distinguished out of a vague blur which swallows them all. Differences
in the pitch and intensity of sounds leave behind a different feeling,
but until we assume different attitudes toward them, or \emph{do}
something special in reference to them, their vague difference cannot be
\emph{intellectually} gripped and retained.

\marginpar{Children's drawings illustrate domination by value}

Children's drawings afford a further exemplification of the same
principle. Perspective does not exist, for the child's interest is not
in \emph{pictorial representation}, but in the \emph{things}
represented; and while perspective is essential to the former, it is no
part of the characteristic uses and values of the things themselves. The
house
is drawn with transparent walls, because the rooms, chairs, beds, people
inside, are the important things in the house-meaning; smoke always
comes out of the chimney---otherwise, why have a chimney at all? At
Christmas time, the stockings may be drawn almost as large as the house
or even so large that they have to be put outside of it:---in any case,
it is the scale of values in use that furnishes the scale for their
qualities, the pictures being diagrammatic reminders of these values,
not impartial records of physical and sensory qualities. One of the
chief difficulties felt by most persons in learning the art of pictorial
representation is that habitual uses and results of use have become so
intimately read into the character of things that it is practically
impossible to shut them out at will.

\marginpar{As do sounds used as language signs}

The acquiring of meaning by sounds, in virtue of which they become
words, is perhaps the most striking illustration that can be found of
the way in which mere sensory stimuli acquire definiteness and constancy
of meaning and are thereby themselves defined and interconnected for
purposes of recognition. Language is a specially good example because
there are hundreds or even thousands of words in which meaning is now so
thoroughly consolidated with physical qualities as to be directly
apprehended, while in the case of words it is easier to recognize that
this connection has been gradually and laboriously acquired than in the
case of physical objects such as chairs, tables, buttons, trees, stones,
hills, flowers, and so on, where it seems as if the union of
intellectual character and meaning with the physical fact were
aboriginal, and thrust upon us passively rather than acquired through
active explorations. And in the case of the meaning of words, we see
readily that it is by
making
sounds and noting the results which follow, by listening to the sounds
of others and watching the activities which accompany them, that a given
sound finally becomes the stable bearer of a meaning.

\marginpar{Summary}

Familiar acquaintance with meanings thus signifies that we have acquired
in the presence of objects definite attitudes of response which lead us,
without reflection, to anticipate certain possible consequences. The
definiteness of the expectation defines the meaning or takes it out of
the vague and pulpy; its habitual, recurrent character gives the meaning
constancy, stability, consistency, or takes it out of the fluctuating
and wavering.

\section{Conceptions and Meaning}

\marginpar{A conception is a definite meaning}

The word \emph{meaning} is a familiar everyday term; the words
\emph{conception}, \emph{notion}, are both popular and technical terms.
Strictly speaking, they involve, however, nothing new; any meaning
sufficiently individualized to be directly grasped and readily used, and
thus fixed by a word, is a conception or notion. Linguistically, every
common noun is the carrier of a meaning, while proper nouns and common
nouns with the word \emph{this} or \emph{that} prefixed, refer to the
things in which the meanings are exemplified. That thinking both employs
and expands notions, conceptions, is then simply saying that in
inference and judgment we use meanings, and that this use also corrects
and widens them.

\marginpar{which is standardized}

Various persons talk about an object not physically present, and yet all
get the same material of belief. The same person in different moments
often refers to the same object or kind of objects. The sense
experience, the physical conditions, the psychological conditions, vary,
but the same meaning is conserved. If
pounds
arbitrarily changed their weight, and foot rules their length, while we
were using them, obviously we could not weigh nor measure. This would be
our intellectual position if meanings could not be maintained with a
certain stability and constancy through a variety of physical and
personal changes.

\marginpar{By it we identify the unknown}

\marginpar{and supplement the sensibly present}

\marginpar{and also systematize things}

To insist upon the fundamental importance of conceptions would,
accordingly, only repeat what has been said. We shall merely summarize,
saying that conceptions, or standard meanings, are instruments
(\emph{i}) of identification, (\emph{ii}) of supplementation, and
(\emph{iii}) of placing in a system. Suppose a little speck of light
hitherto unseen is detected in the heavens. Unless there is a store of
meanings to fall back upon as tools of inquiry and reasoning, that speck
of light will remain just what it is to the senses---a mere speck of
light. For all that it leads to, it might as well be a mere irritation
of the optic nerve. Given the stock of meanings acquired in prior
experience, this speck of light is mentally attacked by means of
appropriate concepts. Does it indicate asteroid, or comet, or a
new-forming sun, or a nebula resulting from some cosmic collision or
disintegration? Each of these conceptions has its own specific and
differentiating characters, which are then sought for by minute and
persistent inquiry. As a result, then, the speck is identified, we will
say, as a comet. Through a standard meaning, it gets identity and
stability of character. Supplementation then takes place. All the known
qualities of comets are read into this particular thing, even though
they have not been as yet observed. All that the astronomers of the past
have learned about the paths and structure of comets becomes available
capital with which to interpret the
speck
of light. Finally, this comet-meaning is itself not isolated; it is a
related portion of the whole system of astronomic knowledge. Suns,
planets, satellites, nebulæ, comets, meteors, star dust---all these
conceptions have a certain mutuality of reference and interaction, and
when the speck of light is identified as meaning a comet, it is at once
adopted as a full member in this vast kingdom of beliefs.

\marginpar{Importance of system to knowledge}

Darwin, in an autobiographical sketch, says that when a youth he told
the geologist, Sidgwick, of finding a tropical shell in a certain gravel
pit. Thereupon Sidgwick said it must have been thrown there by some
person, adding: "But if it were really embedded there, it would be the
greatest misfortune to geology, because it would overthrow all that we
know about the superficial deposits of the Midland Counties"---since
they were glacial. And then Darwin adds: "I was then utterly astonished
at Sidgwick not being delighted at so wonderful a fact as a tropical
shell being found near the surface in the middle of England. Nothing
before had made me thoroughly realize \emph{that science consists in
grouping facts so that general laws or conclusions may be drawn from
them}." This instance (which might, of course, be duplicated from any
branch of science) indicates how scientific notions make explicit the
systematizing tendency involved in all use of concepts.

\section{What Conceptions are Not}

The idea that a conception is a meaning that supplies a standard rule
for the identification and placing of particulars may be contrasted with
some current misapprehensions of its nature.

\marginpar{A concept is not a bare residue}

1. Conceptions are not derived from a multitude
of
different definite objects by leaving out the qualities in which they
differ and retaining those in which they agree. The origin of concepts
is sometimes described to be as if a child began with a lot of different
particular things, say particular dogs; his own Fido, his neighbor's
Carlo, his cousin's Tray. Having all these different objects before him,
he analyzes them into a lot of different qualities, say (\emph{a})
color, (\emph{b}) size, (\emph{c}) shape, (\emph{d}) number of legs,
(\emph{e}) quantity and quality of hair, (\emph{f}) digestive organs,
and so on; and then strikes out all the unlike qualities (such as color,
size, shape, hair), retaining traits such as quadruped and domesticated,
which they all have in general.

\marginpar{but an active attitude}

As a matter of fact, the child begins with whatever significance he has
got out of the one dog he has seen, heard, and handled. He has found
that he can carry over from one experience of this object to subsequent
experience certain expectations of certain characteristic modes of
behavior---may expect these even before they show themselves. He tends
to assume this attitude of anticipation whenever any clue or stimulus
presents itself; whenever the object gives him any excuse for it. Thus
he might call cats little dogs, or horses big dogs. But finding that
other expected traits and modes of behavior are not fulfilled, he is
forced to throw out certain traits from the dog-meaning, while by
contrast (see p. 90) certain other traits are selected and emphasized.
As he further applies the meaning to other dogs, the dog-meaning gets
still further defined and refined. He does not begin with a lot of
ready-made objects from which he extracts a common meaning; he tries to
apply to every new experience whatever from his old experience will help
him understand
it,
and as this process of constant assumption and experimentation is
fulfilled and refuted by results, his conceptions get body and
clearness.

\marginpar{It is general because of its application}

2. Similarly, conceptions are general because of their use and
application, not because of their ingredients. The view of the origin of
conception in an impossible sort of analysis has as its counterpart the
idea that the conception is made up out of all the like elements that
remain after dissection of a number of individuals. Not so; the moment a
meaning is gained, it is a working tool of further apprehensions, an
instrument of understanding other things. Thereby the meaning is
\emph{extended} to cover them. Generality resides in application to the
comprehension of new cases, not in constituent parts. A collection of
traits left as the common residuum, the \emph{caput mortuum}, of a
million objects, would be merely a collection, an inventory or
aggregate, not a \emph{general idea}; a striking trait emphasized in any
one experience which then served to help understand some one other
experience, would become, in virtue of that service of application, in
so far general. Synthesis is not a matter of mechanical addition, but of
application of something discovered in one case to bring other cases
into line.

\section{Definition and Organization of Meanings}

\marginpar{Definiteness \emph{versus} vagueness}

\marginpar{In the abstract meaning is intension}

\marginpar{In its application it is extension}

A being that cannot understand at all is at least protected from
\emph{mis}-understandings. But beings that get knowledge by means of
inferring and interpreting, by judging what things signify in relation
to one another, are constantly exposed to the danger of
\emph{mis}-apprehension, \emph{mis}-understanding,
\emph{mis}-taking---taking a thing amiss. A constant source of
misunderstanding and mistake is indefiniteness of meaning. Through
vagueness
of
meaning we misunderstand other people, things, and ourselves; through
its ambiguity we distort and pervert. Conscious distortion of meaning
may be enjoyed as nonsense; erroneous meanings, if clear-cut, may be
followed up and got rid of. But vague meanings are too gelatinous to
offer matter for analysis, and too pulpy to afford support to other
beliefs. They evade testing and responsibility. Vagueness disguises the
unconscious mixing together of different meanings, and facilitates the
substitution of one meaning for another, and covers up the failure to
have any precise meaning at all. It is the aboriginal logical sin---the
source from which flow most bad intellectual consequences. Totally to
eliminate indefiniteness is impossible; to reduce it in extent and in
force requires sincerity and vigor. To be clear or perspicuous a meaning
must be detached, single, self-contained, homogeneous as it were,
throughout. The technical name for any meaning which is thus
individualized is \emph{intension}. The process of arriving at such
units of meaning (and of stating them when reached) is
\emph{definition}. The intension of the terms \emph{man}, \emph{river},
\emph{seed}, \emph{honesty}, \emph{capital}, \emph{supreme court}, is
the meaning that \emph{exclusively} and \emph{characteristically}
attaches to those terms. This meaning is set forth in the definitions of
those words. The test of the distinctness of a meaning is that it shall
successfully mark off a group of things that exemplify the meaning from
other groups, especially of those objects that convey nearly allied
meanings. The river-meaning (or character) must serve to
\emph{designate} the Rhone, the Rhine, the Mississippi, the Hudson, the
Wabash, in spite of their varieties of place, length, quality of water;
and must be such as \emph{not} to suggest ocean currents, ponds, or
brooks. This use of a
meaning
to mark off and group together a variety of distinct existences
constitutes its \emph{extension}.

\marginpar{Definition and division}

As definition sets forth intension, so division (or the reverse process,
classification) expounds extension. Intension and extension, definition
and division, are clearly correlative; in language previously used,
\emph{intension} is meaning as a principle of identifying particulars;
extension is the group of particulars identified and distinguished.
Meaning, as extension, would be wholly in the air or unreal, did it not
point to some object or group of objects; while objects would be as
isolated and independent intellectually as they seem to be spatially,
were they not bound into groups or classes on the basis of
characteristic meanings which they constantly suggest and exemplify.
Taken together, definition and division put us in possession of
individualized or definite meanings and indicate to what group of
objects meanings refer. They typify the fixation and the organization of
meanings. In the degree in which the meanings of any set of experiences
are so cleared up as to serve as principles for grouping those
experiences in relation to one another, that set of particulars becomes
a science; \emph{i.e.} definition and classification are the marks of a
science, as distinct from both unrelated heaps of miscellaneous
information and from the habits that introduce coherence into our
experience without our being aware of their operation.

Definitions are of three types, \emph{denotative}, \emph{expository},
\emph{scientific}. Of these, the first and third are logically
important, while the expository type is socially and pedagogically
important as an intervening step.

\marginpar{We define by picking out}

I. Denotative. A blind man can never have an adequate understanding of
the meaning of \emph{color} and \emph{red}; a seeing person can acquire
the knowledge only by
having
certain things designated in such a way as to fix attention upon some of
their qualities. This method of delimiting a meaning by calling out a
certain attitude toward objects may be called \emph{denotative} or
\emph{indicative}. It is required for all sense qualities---sounds,
tastes, colors---and equally for all emotional and moral qualities. The
meanings of \emph{honesty}, \emph{sympathy}, \emph{hatred}, \emph{fear},
must be grasped by having them presented in an individual's first-hand
experience. The reaction of educational reformers against linguistic and
bookish training has always taken the form of demanding recourse to
personal experience. However advanced the person is in knowledge and in
scientific training, understanding of a new subject, or a new aspect of
an old subject, must always be through these acts of experiencing
directly the existence or quality in question.

\marginpar{and also by combining what is already more definite,}

2. Expository. Given a certain store of meanings which have been
directly or denotatively marked out, language becomes a resource by
which imaginative combinations and variations may be built up. A color
may be defined to one who has not experienced it as lying between green
and blue; a tiger may be defined (\emph{i.e.} the idea of it made more
definite) by selecting some qualities from known members of the cat
tribe and combining them with qualities of size and weight derived from
other objects. Illustrations are of the nature of expository
definitions; so are the accounts of meanings given in a dictionary. By
taking better-known meanings and associating them,---the attained store
of meanings of the community in which one resides is put at one's
disposal. But in themselves these definitions are secondhand and
conventional; there is danger that instead of inciting one to effort
after personal experiences
that
will exemplify and verify them, they will be accepted on authority as
\emph{substitutes}.

\marginpar{and by discovering method of production}

3. Scientific. Even popular definitions serve as rules for identifying
and classifying individuals, but the purpose of such identifications and
classifications is mainly practical and social, not intellectual. To
conceive the whale as a fish does not interfere with the success of
whalers, nor does it prevent recognition of a whale when seen, while to
conceive it not as fish but as mammal serves the practical end equally
well, and also furnishes a much more valuable principle for scientific
identification and classification. Popular definitions select certain
fairly obvious traits as keys to classification. Scientific definitions
select \emph{conditions of causation, production, and generation} as
their characteristic material. The traits used by the popular definition
do not help us to understand why an object has its common meanings and
qualities; they simply state the fact that it does have them. Causal and
genetic definitions fix upon the way an object is constructed as the key
to its being a certain kind of object, and thereby explain why it has
its class or common traits.

\marginpar{Contrast of causal and descriptive definitions}

Science is the most perfect type of knowledge because it uses causal
definitions

If, for example, a layman of considerable practical experience were
asked what he meant or understood by \emph{metal}, he would probably
reply in terms of the qualities useful (\emph{i}) in recognizing any
given metal and (\emph{ii}) in the arts. Smoothness, hardness,
glossiness, and brilliancy, heavy weight for its size, would probably be
included in his definition, because such traits enable us to identify
specific things when we see and touch them; the serviceable properties
of capacity for being hammered and pulled without breaking, of being
softened by heat and hardened by cold, of retaining the shape and
form
given, of resistance to pressure and decay, would probably be
included---whether or not such terms as \emph{malleable} or
\emph{fusible} were used. Now a scientific conception, instead of using,
even with additions, traits of this kind, determines \emph{meaning on a
different basis}. The present definition of metal is about like this:
Metal means any chemical element that enters into combination with
oxygen so as to form a base, \emph{i.e.} a compound that combines with
an acid to form a salt. This scientific definition is founded, not on
directly perceived qualities nor on directly useful properties, but on
the \emph{way in which certain things are causally related to other
things}; \emph{i.e.} it denotes a relation. As chemical concepts become
more and more those of relationships of interaction in constituting
other substances, so physical concepts express more and more relations
of operation: mathematical, as expressing functions of dependence and
order of grouping; biological, relations of differentiation of descent,
effected through adjustment of various environments; and so on through
the sphere of the sciences. In short, our conceptions attain a maximum
of definite individuality and of generality (or applicability) in the
degree to which they show how things depend upon one another or
influence one another, instead of expressing the qualities that objects
possess statically. The ideal of a system of scientific conceptions is
to attain continuity, freedom, and flexibility of transition in passing
from any fact and meaning to any other; this demand is met in the degree
in which we lay hold of the dynamic ties that hold things together in a
continuously changing process---a principle that states insight into
mode of production or
growth.

\chapter{Concrete and Abstract Thinking}

\marginpar{False notions of concrete and abstract}

The maxim enjoined upon teachers, "to proceed from the concrete to the
abstract," is perhaps familiar rather than comprehended. Few who read
and hear it gain a clear conception of the starting-point, the concrete;
of the nature of the goal, the abstract; and of the exact nature of the
path to be traversed in going from one to the other. At times the
injunction is positively misunderstood, being taken to mean that
education should advance from things to thought---as if any dealing with
things in which thinking is not involved could possibly be educative. So
understood, the maxim encourages mechanical routine or sensuous
excitation at one end of the educational scale---the lower---and
academic and unapplied learning at the upper end.

Actually, all dealing with things, even the child's, is immersed in
inferences; things are clothed by the suggestions they arouse, and are
significant as challenges to interpretation or as evidences to
substantiate a belief. Nothing could be more unnatural than instruction
in things without thought; in sense-perceptions without judgments based
upon them. And if the abstract to which we are to proceed denotes
thought apart from things, the goal recommended is formal
and
empty, for effective thought always refers, more or less directly, to
things.

\marginpar{Direct and indirect understanding again}

Yet the maxim has a meaning which, understood and supplemented, states
the line of development of logical capacity. What is this signification?
Concrete denotes a meaning definitely marked off from other meanings so
that it is readily apprehended by itself. When we hear the words,
\emph{table}, \emph{chair}, \emph{stove}, \emph{coat}, we do not have to
reflect in order to grasp what is meant. The terms convey meaning so
directly that no effort at translating is needed. The meanings of some
terms and things, however, are grasped only by first calling to mind
more familiar things and then tracing out connections between them and
what we do not understand. Roughly speaking, the former kind of meanings
is concrete; the latter abstract.

\marginpar{What is familiar is mentally concrete}

To one who is thoroughly at home in physics and chemistry, the notions
of \emph{atom} and \emph{molecule} are fairly concrete. They are
constantly used without involving any labor of thought in apprehending
what they mean. But the layman and the beginner in science have first to
remind themselves of things with which they already are well acquainted,
and go through a process of slow translation; the terms \emph{atom} and
\emph{molecule} losing, moreover, their hard-won meaning only too easily
if familiar things, and the line of transition from them to the strange,
drop out of mind. The same difference is illustrated by any technical
terms: \emph{coefficient} and \emph{exponent} in algebra,
\emph{triangle} and \emph{square} in their geometric as distinct from
their popular meanings; \emph{capital} and \emph{value} as used in
political economy, and so on.

\marginpar{Practical things are familiar}

The difference as noted is purely relative to the intellectual progress
of an individual; what is
abstract
at one period of growth is concrete at another; or even the contrary, as
one finds that things supposed to be thoroughly familiar involve strange
factors and unsolved problems. There is, nevertheless, a general line of
cleavage which, deciding upon the whole what things fall within the
limits of familiar acquaintance and what without, marks off the concrete
and the abstract in a more permanent way. \emph{These limits are fixed
mainly by the demands of practical life.} Things such as sticks and
stones, meat and potatoes, houses and trees, are such constant features
of the environment of which we have to take account in order to live,
that their important meanings are soon learnt, and indissolubly
associated with objects. We are acquainted with a thing (or it is
familiar to us) when we have so much to do with it that its strange and
unexpected corners are rubbed off. The necessities of social intercourse
convey to adults a like concreteness upon such terms as \emph{taxes},
\emph{elections}, \emph{wages}, \emph{the law}, and so on. Things the
meaning of which I personally do not take in directly, appliances of
cook, carpenter, or weaver, for example, are nevertheless unhesitatingly
classed as concrete, since they are so directly connected with our
common social life.

\marginpar{The theoretical, or strictly intellectual, is abstract}

By contrast, the abstract is the \emph{theoretical}, or that not
intimately associated with practical concerns. The abstract thinker (the
man of pure science as he is sometimes called) deliberately abstracts
from application in life; that is, he leaves practical uses out of
account. This, however, is a merely negative statement. What remains
when connections with use and application are excluded? \emph{Evidently
only what has to do with knowing considered as an end in itself.} Many
notions of
science
are abstract, not only because they cannot be understood without a long
apprenticeship in the science (which is equally true of technical
matters in the arts), but also because the whole content of their
meaning has been framed for the sole purpose of facilitating further
knowledge, inquiry, and speculation. \emph{When thinking is used as a
means to some end, good, or value beyond itself, it is concrete; when it
is employed simply as a means to more thinking, it is abstract.} To a
theorist an idea is adequate and self-contained just because it engages
and rewards thought; to a medical practitioner, an engineer, an artist,
a merchant, a politician, it is complete only when employed in the
furthering of some interest in life---health, wealth, beauty, goodness,
success, or what you will.

\marginpar{Contempt for theory}

For the great majority of men under ordinary circumstances, the
practical exigencies of life are almost, if not quite, coercive. Their
main business is the proper conduct of their affairs. Whatever is of
significance only as affording scope for thinking is pallid and
remote---almost artificial. Hence the contempt felt by the practical and
successful executive for the "mere theorist"; hence his conviction that
certain things may be all very well in theory, but that they will not do
in practice; in general, the depreciatory way in which he uses the terms
\emph{abstract}, \emph{theoretical}, and \emph{intellectual}---as
distinct from \emph{intelligent}.

\marginpar{But theory is highly practical}

This attitude is justified, of course, under certain conditions. But
depreciation of theory does not contain the whole truth, as common or
practical sense recognizes. There is such a thing, even from the
common-sense standpoint, as being "too practical," as being so intent
upon the immediately practical as not to
see
beyond the end of one's nose or as to cut off the limb upon which one is
sitting. The question is one of limits, of degrees and adjustments,
rather than one of absolute separation. Truly practical men give their
minds free play about a subject without asking too closely at every
point for the advantage to be gained; exclusive preoccupation with
matters of use and application so narrows the horizon as in the long run
to defeat itself. It does not pay to tether one's thoughts to the post
of use with too short a rope. Power in action requires some largeness
and imaginativeness of vision. Men must at least have enough interest in
thinking for the sake of thinking to escape the limits of routine and
custom. Interest in knowledge for the sake of knowledge, in thinking for
the sake of the free play of thought, is necessary then to the
\emph{emancipation} of practical life---to make it rich and progressive.

We may now recur to the pedagogic maxim of going from the concrete to
the abstract.

\marginpar{Begin with the concrete means begin with practical manipulations}

1. Since the \emph{concrete} denotes thinking applied to activities for
the sake of dealing effectively with the difficulties that present
themselves practically, "beginning with the concrete" signifies that we
should at the outset make much of \emph{doing}; especially, make much in
occupations that are not of a routine and mechanical kind and hence
require intelligent selection and adaptation of means and materials. We
do not "follow the order of nature" when we multiply mere sensations or
accumulate physical objects. Instruction in number is not concrete
merely because splints or beans or dots are employed, while whenever the
use and bearing of number relations are clearly perceived, the number
idea is concrete even if figures alone are used. Just what sort
of
symbol it is best to use at a given time---whether blocks, or lines, or
figures---is entirely a matter of adjustment to the given case. If
physical things used in teaching number or geography or anything else do
not leave the mind illuminated with recognition of a \emph{meaning}
beyond themselves, the instruction that uses them is as abstract as that
which doles out ready-made definitions and rules; for it distracts
attention from ideas to mere physical excitations.

\marginpar{Confusion of the concrete with the sensibly isolated}

The conception that we have only to put before the senses particular
physical objects in order to impress certain ideas upon the mind amounts
almost to a superstition. The introduction of object lessons and
sense-training scored a distinct advance over the prior method of
linguistic symbols, and this advance tended to blind educators to the
fact that only a halfway step had been taken. Things and sensations
develop the child, indeed, but only because he \emph{uses} them in
mastering his body and in the scheme of his activities. Appropriate
continuous occupations or activities involve the use of natural
materials, tools, modes of energy, and do it in a way that compels
thinking as to what they mean, how they are related to one another and
to the realization of ends; while the mere isolated presentation of
things remains barren and dead. A few generations ago the great obstacle
in the way of reform of primary education was belief in the almost
magical efficacy of the symbols of language (including number) to
produce mental training; at present, belief in the efficacy of objects
just as objects, blocks the way. As frequently happens, the better is an
enemy of the best.

\marginpar{Transfer of interest to intellectual matters}

2. The interest in results, in the successful carrying on of an
activity, should be gradually transferred to
study
of objects---their properties, consequences, structures, causes, and
effects. The adult when at work in his life calling is rarely free to
devote time or energy---beyond the necessities of his immediate
action---to the study of what he deals with. (\emph{Ante}, p. 43.) The
educative activities of childhood should be so arranged that direct
interest in the activity and its outcome create a demand for attention
to matters that have a more and more \emph{indirect and remote}
connection with the original activity. The direct interest in
carpentering or shop work should yield organically and gradually an
interest in geometric and mechanical problems. The interest in cooking
should grow into an interest in chemical experimentation and in the
physiology and hygiene of bodily growth. The making of pictures should
pass to an interest in the technique of representation and the æsthetics
of appreciation, and so on. This development is what the term \emph{go}
signifies in the maxim "\emph{go} from the concrete to the abstract"; it
represents the dynamic and truly educative factor of the process.

\marginpar{Development of delight in the activity of thinking}

3. The outcome, the \emph{abstract} to which education is to proceed, is
an interest in intellectual matters for their own sake, a delight in
thinking for the sake of thinking. It is an old story that acts and
processes which at the outset are incidental to something else develop
and maintain an absorbing value of their own. So it is with thinking and
with knowledge; at first incidental to results and adjustments beyond
themselves, they attract more and more attention to themselves till they
become ends, not means. Children engage, unconstrainedly and
continually, in reflective inspection and testing for the sake of what
they are interested in doing successfully. Habits of thinking thus
generated may increase in
volume
and extent till they become of importance on their own account.

\marginpar{Examples of the transition}

The three instances cited in
\protect\hyperlink{ux40publicux40vhostux40gux40gutenbergux40htmlux40filesux4037423ux4037423-hux4037423-h-2.htm.htmlux5cux23CHAPTER_SIX}{Chapter
Six} represented an ascending cycle from the practical to the
theoretical. Taking thought to keep a personal engagement is obviously
of the concrete kind. Endeavoring to work out the meaning of a certain
part of a boat is an instance of an intermediate kind. The reason for
the existence and position of the pole is a practical reason, so that to
the architect the problem was purely concrete---the maintenance of a
certain system of action. But for the passenger on the boat, the problem
was theoretical, more or less speculative. It made no difference to his
reaching his destination whether he worked out the meaning of the pole.
The third case, that of the appearance and movement of the bubbles,
illustrates a strictly theoretical or abstract case. No overcoming of
physical obstacles, no adjustment of external means to ends, is at
stake. Curiosity, intellectual curiosity, is challenged by a seemingly
anomalous occurrence; and thinking tries simply to account for an
apparent exception in terms of recognized principles.

\marginpar{Theoretical knowledge never the whole end}

(\emph{i}) Abstract thinking, it should be noted, represents \emph{an}
end, not \emph{the} end. The power of sustained thinking on matters
remote from direct use is an outgrowth of practical and immediate modes
of thought, but not a substitute for them. The educational end is not
the destruction of power to think so as to surmount obstacles and adjust
means and ends; it is not its replacement by abstract reflection. Nor is
theoretical thinking a higher type of thinking than practical. A person
who has at command both types of thinking is of a higher order than he
who possesses only one. Methods that in
developing
abstract intellectual abilities weaken habits of practical or concrete
thinking, fall as much short of the educational ideal as do the methods
that in cultivating ability to plan, to invent, to arrange, to forecast,
fail to secure some delight in thinking irrespective of practical
consequences.

\marginpar{Nor that most congenial to the majority of pupils}

(\emph{ii}) Educators should also note the very great individual
differences that exist; they should not try to force one pattern and
model upon all. In many (probably the majority) the executive tendency,
the habit of mind that thinks for purposes of conduct and achievement,
not for the sake of knowing, remains dominant to the end. Engineers,
lawyers, doctors, merchants, are much more numerous in adult life than
scholars, scientists, and philosophers. While education should strive to
make men who, however prominent their professional interests and aims,
partake of the spirit of the scholar, philosopher, and scientist, no
good reason appears why education should esteem the one mental habit
inherently superior to the other, and deliberately try to transform the
type from practical to theoretical. Have not our schools (as already
suggested, p. 49) been one-sidedly devoted to the more abstract type of
thinking, thus doing injustice to the majority of pupils? Has not the
idea of a "liberal" and "humane" education tended too often in practice
to the production of technical, because overspecialized, thinkers?

\marginpar{Aim of education is a working balance}

The aim of education should be to secure a balanced interaction of the
two types of mental attitude, having sufficient regard to the
disposition of the individual not to hamper and cripple whatever powers
are naturally strong in him. The narrowness of individuals of strong
concrete bent needs to be liberalized. Every
opportunity
that occurs within their practical activities for developing curiosity
and susceptibility to intellectual problems should be seized. Violence
is not done to natural disposition, but the latter is broadened. As
regards the smaller number of those who have a taste for abstract,
purely intellectual topics, pains should be taken to multiply
opportunities and demands for the application of ideas; for translating
symbolic truths into terms of social life and its ends. Every human
being has both capabilities, and every individual will be more effective
and happier if both powers are developed in easy and close interaction
with each
other.

\chapter{Empirical and Scientific Thinking}

\section{Empirical Thinking}

\marginpar{Empirical thinking depends on past habits}

Apart from the development of scientific method, inferences depend upon
habits that have been built up under the influence of a number of
particular experiences not themselves arranged for logical purposes. A
says, "It will probably rain to-morrow." B asks, "Why do you think so?"
and A replies, "Because the sky was lowering at sunset." When B asks,
"What has that to do with it?" A responds, "I do not know, but it
generally does rain after such a sunset." He does not perceive any
\emph{connection} between the appearance of the sky and coming rain; he
is not aware of any continuity in the facts themselves---any law or
principle, as we usually say. He simply, from frequently recurring
conjunctions of the events, has associated them so that when he sees one
he thinks of the other. One \emph{suggests} the other, or is
\emph{associated} with it. A man may believe it will rain to-morrow
because he has consulted the barometer; but if he has no conception how
the height of the mercury column (or the position of an index moved by
its rise and fall) is connected with variations of atmospheric pressure,
and how these in turn are connected with the amount of moisture in the
air, his belief in the likelihood of rain is purely empirical. When men
lived in the open and got their living by hunting, fishing,
or
pasturing flocks, the detection of the signs and indications of weather
changes was a matter of great importance. A body of proverbs and maxims,
forming an extensive section of traditionary folklore, was developed.
But as long as there was no understanding \emph{why} or \emph{how}
certain events were signs, as long as foresight and weather shrewdness
rested simply upon repeated conjunction among facts, beliefs about the
weather were thoroughly empirical.

\marginpar{It is fairly adequate in some matters,}

In similar fashion learned men in the Orient learned to predict, with
considerable accuracy, the recurrent positions of the planets, the sun
and the moon, and to foretell the time of eclipses, without
understanding in any degree the laws of the movements of heavenly
bodies---that is, without having a notion of the continuities existing
among the facts themselves. They had learned from repeated observations
that things happened in about such and such a fashion. Till a
comparatively recent time, the truths of medicine were mainly in the
same condition. Experience had shown that "upon the whole," "as a rule,"
"generally or usually speaking," certain results followed certain
remedies, when symptoms were given. Our beliefs about human nature in
individuals (psychology) and in masses (sociology) are still very
largely of a purely empirical sort. Even the science of geometry, now
frequently reckoned a typical rational science, began, among the
Egyptians, as an accumulation of recorded observations about methods of
approximate mensuration of land surfaces; and only gradually assumed,
among the Greeks, scientific form.

The \emph{disadvantages} of purely empirical thinking are
obvious.

\marginpar{but is very apt to lead to false beliefs,}

1. While many empirical conclusions are, roughly speaking, correct;
while they are exact enough to be of great help in practical life; while
the presages of a weatherwise sailor or hunter may be more accurate,
within a certain restricted range, than those of a scientist who relies
wholly upon scientific observations and tests; while, indeed, empirical
observations and records furnish the raw or crude material of scientific
knowledge, yet the empirical method affords no way of discriminating
between right and wrong conclusions. Hence it is responsible for a
multitude of \emph{false} beliefs. The technical designation for one of
the commonest fallacies is \emph{post hoc, ergo propter hoc}; the belief
that because one thing comes \emph{after} another, it comes
\emph{because} of the other. Now this fallacy of method is the animating
principle of empirical conclusions, even when correct---the correctness
being almost as much a matter of good luck as of method. That potatoes
should be planted only during the crescent moon, that near the sea
people are born at high tide and die at low tide, that a comet is an
omen of danger, that bad luck follows the cracking of a mirror, that a
patent medicine cures a disease---these and a thousand like notions are
asseverated on the basis of empirical coincidence and conjunction.
Moreover, habits of expectation and belief are formed otherwise than by
a number of repeated similar cases.

\marginpar{and does not enable us to cope with the novel,}

2. The more numerous the experienced instances and the closer the watch
kept upon them, the greater is the trustworthiness of constant
conjunction as evidence of connection among the things themselves. Many
of our most important beliefs still have only this sort of warrant. No
one can yet tell, with certainty, the
necessary
cause of old age or of death---which are empirically the most certain of
all expectations. But even the most reliable beliefs of this type fail
when they confront the \emph{novel}. Since they rest upon past
uniformities, they are useless when further experience departs in any
considerable measure from ancient incident and wonted precedent.
Empirical inference follows the grooves and ruts that custom wears, and
has no track to follow when the groove disappears. So important is this
aspect of the matter that Clifford found the difference between ordinary
skill and scientific thought right here. "Skill enables a man to deal
with the same circumstances that he has met before, scientific thought
enables him to deal with different circumstances that he has never met
before." And he goes so far as to define scientific thinking as "the
application of old experience to new circumstances."

\marginpar{and leads to laziness and presumption,}

3. We have not yet made the acquaintance of the most harmful feature of
the empirical method. Mental inertia, laziness, unjustifiable
conservatism, are its probable accompaniments. Its general effect upon
mental attitude is more serious than even the specific wrong conclusions
in which it has landed. Wherever the chief dependence in forming
inferences is upon the conjunctions observed in past experience,
failures to agree with the usual order are slurred over, cases of
successful confirmation are exaggerated. Since the mind naturally
demands some principle of continuity, some connecting link between
separate facts and causes, forces are arbitrarily invented for that
purpose. Fantastic and mythological explanations are resorted to in
order to supply missing links. The pump brings water because nature
abhors a vacuum; opium makes men sleep because it has a
dormitive
potency; we recollect a past event because we have a faculty of memory.
In the history of the progress of human knowledge, out and out myths
accompany the first stage of empiricism; while "hidden essences" and
"occult forces" mark its second stage. By their very nature, these
"causes" escape observation, so that their explanatory value can be
neither confirmed nor refuted by further observation or experience.
Hence belief in them becomes purely traditionary. They give rise to
doctrines which, inculcated and handed down, become dogmas; subsequent
inquiry and reflection are actually stifled. (\emph{Ante}, p. 23.)

\marginpar{and to dogmatism}

Certain men or classes of men come to be the accepted guardians and
transmitters---instructors---of established doctrines. To question the
beliefs is to question their authority; to accept the beliefs is
evidence of loyalty to the powers that be, a proof of good citizenship.
Passivity, docility, acquiescence, come to be primal intellectual
virtues. Facts and events presenting novelty and variety are slighted,
or are sheared down till they fit into the Procrustean bed of habitual
belief. Inquiry and doubt are silenced by citation of ancient laws or a
multitude of miscellaneous and unsifted cases. This attitude of mind
generates dislike of change, and the resulting aversion to novelty is
fatal to progress. What will not fit into the established canons is
outlawed; men who make new discoveries are objects of suspicion and even
of persecution. Beliefs that perhaps originally were the products of
fairly extensive and careful observation are stereotyped into fixed
traditions and semi-sacred dogmas accepted simply upon authority, and
are mixed with fantastic conceptions that happen to have won the
acceptance of
authorities.

\section{Scientific Method}

\marginpar{Scientific thinking analyzes the present case}

In contrast with the empirical method stands the scientific. Scientific
method replaces the repeated conjunction or coincidence of separate
facts by discovery of a single comprehensive fact, effecting this
replacement by \emph{breaking up the coarse or gross facts of
observation into a number of minuter processes not directly accessible
to perception}.

\marginpar{Illustration from \emph{suction} of empirical method,}

If a layman were asked why water rises from the cistern when an ordinary
pump is worked, he would doubtless answer, "By suction." Suction is
regarded as a force like heat or pressure. If such a person is
confronted by the fact that water rises with a suction pump only about
thirty-three feet, he easily disposes of the difficulty on the ground
that all forces vary in their intensities and finally reach a limit at
which they cease to operate. The variation with elevation above the sea
level of the height to which water can be pumped is either unnoticed,
or, if noted, is dismissed as one of the curious anomalies in which
nature abounds.

\marginpar{of scientific method}

\marginpar{Relies on differences,}

Now the scientist advances by assuming that what seems to observation to
be a single total fact is in truth complex. He attempts, therefore, to
break up the single fact of water-rising-in-the-pipe into a number of
lesser facts. His method of proceeding is by \emph{varying conditions
one by one} so far as possible, and noting just what happens when a
given condition is eliminated. There are two methods for varying
conditions.\footnote{
The next two paragraphs repeat, for purposes of the present discussion,
what we have already noted in a different context. See p. 88 and p. 99.
}
The first is an extension of the empirical method of observation. It
consists in comparing very carefully the results of a great number of
observations which have
occurred
under accidentally \emph{different} conditions. The difference in the
rise of the water at different heights above the sea level, and its
total cessation when the distance to be lifted is, even at sea level,
more than thirty-three feet, are emphasized, instead of being slurred
over. The purpose is to find out what \emph{special conditions} are
present when the effect occurs and absent when it fails to occur. These
special conditions are then substituted for the gross fact, or regarded
as its principle---the key to understanding it.

\marginpar{and creates differences}

The method of analysis by comparing cases is, however, badly
handicapped; it can do nothing until it is presented with a certain
number of diversified cases. And even when different cases are at hand,
it will be questionable whether they vary in just these respects in
which it is important that they should vary in order to throw light upon
the question at issue. The method is passive and dependent upon external
accidents. Hence the superiority of the active or experimental method.
Even a small number of observations may suggest an explanation---a
hypothesis or theory. Working upon this suggestion, the scientist may
then \emph{intentionally} vary conditions and note what happens. If the
empirical observations have suggested to him the possibility of a
connection between air pressure on the water and the rising of the water
in the tube where air pressure is absent, he deliberately empties the
air out of the vessel in which the water is contained and notes that
suction no longer works; or he intentionally increases atmospheric
pressure on the water and notes the result. He institutes experiments to
calculate the weight of air at the sea level and at various levels
above, and compares the results of reasoning based upon the pressure of
air
of these various weights upon a certain volume of water with the results
actually obtained by observation. \emph{Observations formed by variation
of conditions on the basis of some idea or theory constitute
experiment.} Experiment is the chief resource in scientific reasoning
because it facilitates the picking out of significant elements in a
gross, vague whole.

\marginpar{Analysis and synthesis again}

Experimental thinking, or scientific reasoning, is thus a conjoint
process of \emph{analysis and synthesis}, or, in less technical
language, of discrimination and assimilation or identification. The
gross fact of water rising when the suction valve is worked is resolved
or discriminated into a number of independent variables, some of which
had never before been observed or even thought of in connection with the
fact. One of these facts, the weight of the atmosphere, is then
selectively seized upon as the key to the entire phenomenon. This
disentangling constitutes \emph{analysis}. But atmosphere and its
pressure or weight is a fact not confined to this single instance. It is
a fact familiar or at least discoverable as operative in a great number
of other events. In fixing upon this imperceptible and minute fact as
the essence or key to the elevation of water by the pump, the pump-fact
has thus been assimilated to a whole group of ordinary facts from which
it was previously isolated. This assimilation constitutes
\emph{synthesis}. Moreover, the fact of atmospheric pressure is itself a
case of one of the commonest of all facts---weight or gravitational
force. Conclusions that apply to the common fact of weight are thus
transferable to the consideration and interpretation of the
\emph{relatively} rare and exceptional case of the suction of water. The
suction pump is seen to be a case of the same kind or sort as the
siphon,
the
barometer, the rising of the balloon, and a multitude of other things
with which at first sight it has no connection at all. This is another
instance of the synthetic or assimilative phase of scientific thinking.

If we revert to the advantages of scientific over empirical thinking, we
find that we now have the clue to them.

\marginpar{Lessened liability to error}

(\emph{a}) The increased security, the added factor of certainty or
proof, is due to the substitution of the \emph{detailed and specific
fact} of atmospheric pressure for the gross and total and relatively
miscellaneous fact of suction. The latter is complex, and its complexity
is due to many unknown and unspecified factors; hence, any statement
about it is more or less random, and likely to be defeated by any
unforeseen variation of circumstances. \emph{Comparatively}, at least,
the minute and detailed fact of air pressure is a measurable and
definite fact---one that can be picked out and managed with assurance.

\marginpar{Ability to manage the new}

(\emph{b}) As analysis accounts for the added certainty, so synthesis
accounts for ability to cope with the novel and variable. Weight is a
much commoner fact than atmospheric weight, and this in turn is a much
commoner fact than the workings of the suction pump. To be able to
substitute the common and frequent fact for that which is relatively
rare and peculiar is to reduce the seemingly novel and exceptional to
cases of a general and familiar principle, and thus to bring them under
control for interpretation and prediction.

As Professor James says: "Think of heat as motion and whatever is true
of motion will be true of heat; but we have a hundred experiences of
motion for every one of heat. Think of rays passing through this lens as
cases of bending toward the perpendicular, and
you
substitute for the comparatively unfamiliar lens the very familiar
notion of a particular change in direction of a line, of which notion
every day brings us countless
examples."\footnote{\emph{Psychology}, vol. II. p. 342.}

\marginpar{Interest in the future or in progress}

(\emph{c}) The change of attitude from conservative reliance upon the
past, upon routine and custom, to faith in progress through the
intelligent regulation of existing conditions, is, of course, the reflex
of the scientific method of experimentation. The empirical method
inevitably magnifies the influences of the past; the experimental method
throws into relief the possibilities of the future. The empirical method
says, "\emph{Wait} till there is a sufficient number of cases;" the
experimental method says, "\emph{Produce} the cases." The former depends
upon nature's accidentally happening to present us with certain
conjunctions of circumstances; the latter deliberately and intentionally
endeavors to bring about the conjunction. By this method the notion of
progress secures scientific warrant.

\marginpar{Physical \emph{versus} logical force}

Ordinary experience is controlled largely by the direct strength and
intensity of various occurrences. What is bright, sudden, loud, secures
notice and is given a conspicuous rating. What is dim, feeble, and
continuous gets ignored, or is regarded as of slight importance.
Customary experience tends to the control of thinking by considerations
of \emph{direct and immediate strength} rather than by those of
importance in the long run. Animals without the power of forecast and
planning must, upon the whole, respond to the stimuli that are most
urgent at the moment, or cease to exist. These stimuli lose nothing of
their direct urgency and clamorous insistency when the thinking power
develops; and yet
thinking
demands the subordination of the immediate stimulus to the remote and
distant. The feeble and the minute may be of much greater importance
than the glaring and the big. The latter may be signs of a force that is
already exhausting itself; the former may indicate the beginnings of a
process in which the whole fortune of the individual is involved. The
prime necessity for scientific thought is that the thinker be freed from
the tyranny of sense stimuli and habit, and this emancipation is also
the necessary condition of progress.

\marginpar{Illustration from moving water}

Consider the following quotation: "When it first occurred to a
reflecting mind that moving water had a property identical with human or
brute force, namely, the property of setting other masses in motion,
overcoming inertia and resistance,---when the sight of the stream
suggested through this point of likeness the power of the animal,---a
new addition was made to the class of prime movers, and when
circumstances permitted, this power could become a substitute for the
others. It may seem to the modern understanding, familiar with water
wheels and drifting rafts, that the similarity here was an extremely
obvious one. But if we put ourselves back into an early state of mind,
when running water affected the mind \emph{by its brilliancy, its roar
and irregular devastation}, we may easily suppose that to identify this
with animal muscular energy was by no means an obvious
effort."\footnote{
Bain, \emph{The Senses and Intellect}, third American ed., 1879, p. 492
(italics not in original).
}

\marginpar{Value of abstraction}

If we add to these obvious sensory features the various social customs
and expectations which fix the attitude of the individual, the evil of
the subjection of free and fertile suggestion to empirical
considerations
becomes
clear. A certain power of \emph{abstraction}, of deliberate turning away
from the habitual responses to a situation, was required before men
could be emancipated to follow up suggestions that in the end are
fruitful.

\marginpar{Experience as inclusive of thought}

In short, the term \emph{experience} may be interpreted either with
reference to the \emph{empirical} or the \emph{experimental} attitude of
mind. Experience is not a rigid and closed thing; it is vital, and hence
growing. When dominated by the past, by custom and routine, it is often
opposed to the reasonable, the thoughtful. But experience also includes
the reflection that sets us free from the limiting influence of sense,
appetite, and tradition. Experience may welcome and assimilate all that
the most exact and penetrating thought discovers. Indeed, the business
of education might be defined as just such an emancipation and
enlargement of experience. Education takes the individual while he is
relatively plastic, before he has become so indurated by isolated
experiences as to be rendered hopelessly empirical in his habit of mind.
The attitude of childhood is naïve, wondering, experimental; the world
of man and nature is new. Right methods of education preserve and
perfect this attitude, and thereby short-circuit for the individual the
slow progress of the race, eliminating the waste that comes from inert
routine.

\part{The Training of Thought}

\chapter{Activity and the Training of Thought}

In this chapter we shall gather together and amplify considerations that
have already been advanced, in various passages of the preceding pages,
concerning the relation of \emph{action to thought}. We shall follow,
though not with exactness, the order of development in the unfolding
human being.

\section{The Early Stage of Activity}

\marginpar{1. The baby's problem determines his thinking}

The sight of a baby often calls out the question: "What do you suppose
he is thinking about?" By the nature of the case, the question is
unanswerable in detail; but, also by the nature of the case, we may be
sure about a baby's chief interest. His primary problem is mastery of
his body as a tool of securing comfortable and effective adjustments to
his surroundings, physical and social. The child has to learn to do
almost everything: to see, to hear, to reach, to handle, to balance the
body, to creep, to walk, and so on. Even if it be true that human beings
have even more instinctive reactions than lower animals, it is also true
that instinctive tendencies are much less perfect in men, and that most
of them
are
of little use till they are intelligently combined and directed. A
little chick just out of the shell will after a few trials peck at and
grasp grains of food with its beak as well as at any later time. This
involves a complicated coördination of the eye and the head. An infant
does not even begin to reach definitely for things that the eye sees
till he is several months old, and even then several weeks' practice is
required before he learns the adjustment so as neither to overreach nor
to underreach. It may not be literally true that the child will grasp
for the moon, but it is true that he needs much practice before he can
tell whether an object is within reach or not. The arm is thrust out
instinctively in response to a stimulus from the eye, and this tendency
is the origin of the ability to reach and grasp exactly and quickly; but
nevertheless final mastery requires observing and selecting the
successful movements, and arranging them in view of an end. \emph{These
operations of conscious selection and arrangement constitute thinking},
though of a rudimentary type.

\marginpar{Mastery of the body is an intellectual problem}

Since mastery of the bodily organs is necessary for all later
developments, such problems are both interesting and important, and
solving them supplies a very genuine training of thinking power. The joy
the child shows in learning to use his limbs, to translate what he sees
into what he handles, to connect sounds with sights, sights with taste
and touch, and the rapidity with which intelligence grows in the first
year and a half of life (the time during which the more fundamental
problems of the use of the organism are mastered), are sufficient
evidence that the development of physical control is not a physical but
an intellectual achievement.

\marginpar{2. The problem of social adjustment and intercourse}

Although in the early months the child is mainly
occupied
in learning to use his body to accommodate himself to physical
conditions in a comfortable way and to use things skillfully and
effectively, yet social adjustments are very important. In connection
with parents, nurse, brother, and sister, the child learns the signs of
satisfaction of hunger, of removal of discomfort, of the approach of
agreeable light, color, sound, and so on. His contact with physical
things is regulated by persons, and he soon distinguishes persons as the
most important and interesting of all the objects with which he has to
do. Speech, the accurate adaptation of sounds heard to the movements of
tongue and lips, is, however, the great instrument of social adaptation;
and with the development of speech (usually in the second year)
adaptation of the baby's activities to and with those of other persons
gives the keynote of mental life. His range of possible activities is
indefinitely widened as he watches what other persons do, and as he
tries to understand and to do what they encourage him to attempt. The
outline pattern of mental life is thus set in the first four or five
years. Years, centuries, generations of invention and planning, may have
gone to the development of the performances and occupations of the
adults surrounding the child. Yet for him their activities are direct
stimuli; they are part of his natural environment; they are carried on
in physical terms that appeal to his eye, ear, and touch. He cannot, of
course, appropriate their meaning directly through his senses; but they
furnish stimuli to which he responds, so that his attention is focussed
upon a higher order of materials and of problems. Were it not for this
process by which the achievements of one generation form the stimuli
that direct the activities of the next, the story of
civilization
would be writ in water, and each generation would have laboriously to
make for itself, if it could, its way out of savagery.

\marginpar{Social adjustment results in imitation but is not caused by it}

Imitation is one (though only one, see p. 47) of the means by which the
activities of adults supply stimuli which are so interesting, so varied,
so complex, and so novel, as to occasion a rapid progress of thought.
Mere imitation, however, would not give rise to thinking; if we could
learn like parrots by simply copying the outward acts of others, we
should never have to think; nor should we know, after we had mastered
the copied act, what was the meaning of the thing we had done. Educators
(and psychologists) have often assumed that acts which reproduce the
behavior of others are acquired merely by imitation. But a child rarely
learns by conscious imitation; and to say that his imitation is
unconscious is to say that it is not from his standpoint imitation at
all. The word, the gesture, the act, the occupation of another, falls in
line with \emph{some impulse already active} and suggests some
satisfactory mode of expression, some end in which it may find
fulfillment. Having this end of his own, the child then notes other
persons, as he notes natural events, to get further suggestions as to
means of its realization. He selects some of the means he observes,
tries them on, finds them successful or unsuccessful, is confirmed or
weakened in his belief in their value, and so continues selecting,
arranging, adapting, testing, till he can accomplish what he wishes. The
onlooker may then observe the resemblance of this act to some act of an
adult, and conclude that it was acquired by imitation, while as a matter
of fact it was acquired by attention, observation, selection,
experimentation, and confirmation by results.
Only
because this method is employed is there intellectual discipline and an
educative result. The presence of adult activities plays an enormous
rôle in the intellectual growth of the child because they add to the
natural stimuli of the world new stimuli which are more exactly adapted
to the needs of a human being, which are richer, better organized, more
complex in range, permitting more flexible adaptations, and calling out
novel reactions. But in utilizing these stimuli the child follows the
same methods that he uses when he is forced to think in order to master
his body.

\section{Play, Work, and Allied Forms of Activity}

\marginpar{Play indicates the domination of activity by meanings or ideas}

\marginpar{Organization of ideas involved in play}

When things become signs, when they gain a representative capacity as
standing for other things, play is transformed from mere physical
exuberance into an activity involving a mental factor. A little girl who
had broken her doll was seen to perform with the leg of the doll all the
operations of washing, putting to bed, and fondling, that she had been
accustomed to perform with the entire doll. The part stood for the
whole; she reacted not to the stimulus sensibly present, but to the
meaning suggested by the sense object. So children use a stone for a
table, leaves for plates, acorns for cups. So they use their dolls,
their trains, their blocks, their other toys. In manipulating them, they
are living not with the physical things, but in the large world of
meanings, natural and social, evoked by these things. So when children
play horse, play store, play house or making calls, they are
subordinating the physically present to the ideally signified. In this
way, a world of meanings, a store of concepts (so fundamental to all
intellectual achievement), is defined and built
up.
Moreover, not only do meanings thus become familiar acquaintances, but
they are organized, arranged in groups, made to cohere in connected
ways. A play and a story blend insensibly into each other. The most
fanciful plays of children rarely lose all touch with the mutual fitness
and pertinency of various meanings to one another; the "freest" plays
observe some principles of coherence and unification. They have a
beginning, middle, and end. In games, rules of order run through various
minor acts and bind them into a connected whole. The rhythm, the
competition, and coöperation involved in most plays and games also
introduce organization. There is, then, nothing mysterious or mystical
in the discovery made by Plato and remade by Froebel that play is the
chief, almost the only, mode of education for the child in the years of
later infancy.

\marginpar{The playful attitude}

\emph{Playfulness} is a more important consideration than play. The
former is an attitude of mind; the latter is a passing outward
manifestation of this attitude. When things are treated simply as
vehicles of suggestion, what is suggested overrides the thing. Hence the
playful attitude is one of freedom. The person is not bound to the
physical traits of things, nor does he care whether a thing really means
(as we say) what he takes it to represent. When the child plays horse
with a broom and cars with chairs, the fact that the broom does not
really represent a horse, or a chair a locomotive, is of no account. In
order, then, that playfulness may not terminate in arbitrary
fancifulness and in building up an imaginary world alongside the world
of actual things, it is necessary that the play attitude should
gradually pass into a work attitude.

\marginpar{The work attitude is interested in means and ends}

What is work---work not as mere external
performance,
but as attitude of mind? It signifies that the person is not content
longer to accept and to act upon the meanings that things suggest, but
demands congruity of meaning with the things themselves. In the natural
course of growth, children come to find irresponsible make-believe plays
inadequate. A fiction is too easy a way out to afford content. There is
not enough stimulus to call forth satisfactory mental response. When
this point is reached, the ideas that things suggest must be applied to
the things with some regard to fitness. A small cart, resembling a
"real" cart, with "real" wheels, tongue, and body, meets the mental
demand better than merely making believe that anything which comes to
hand is a cart. Occasionally to take part in setting a "real" table with
"real" dishes brings more reward than forever to make believe a flat
stone is a table and that leaves are dishes. The interest may still
center in the meanings, the things may be of importance only as
amplifying a certain meaning. So far the attitude is one of play. But
the meaning is now of such a character that it must find appropriate
embodiment in actual things.

The dictionary does not permit us to call such activities work.
Nevertheless, they represent a genuine passage of play into work. For
work (as a mental attitude, not as mere external performance)
\emph{means interest in the adequate embodiment of a meaning} (a
suggestion, purpose, aim) \emph{in objective form through the use of
appropriate materials and appliances}. Such an attitude takes advantage
of the meanings aroused and built up in free play, but \emph{controls
their development by seeing to it that they are applied to things in
ways consistent with the observable structure of the things
themselves}.

\marginpar{and in processes on account of their results}

The point of this distinction between play and work may be cleared up by
comparing it with a more usual way of stating the difference. In play
activity, it is said, the interest is in the activity for its own sake;
in work, it is in the product or result in which the activity
terminates. Hence the former is purely free, while the latter is tied
down by the end to be achieved. When the difference is stated in this
sharp fashion, there is almost always introduced a false, unnatural
separation between process and product, between activity and its
achieved outcome. The true distinction is not between an interest in
activity for its own sake and interest in the external result of that
activity, but between an interest in an activity just as it flows on
from moment to moment, and an interest in an activity as tending to a
culmination, to an outcome, and therefore possessing a thread of
continuity binding together its successive stages. Both may equally
exemplify interest in an activity "for its own sake"; but in one case
the activity in which the interest resides is more or less casual,
following the accident of circumstance and whim, or of dictation; in the
other, the activity is enriched by the sense that it leads somewhere,
that it amounts to something.

\marginpar{Consequences of the sharp separation of play and work}

Were it not that the false theory of the relation of the play and the
work attitudes has been connected with unfortunate modes of school
practice, insistence upon a truer view might seem an unnecessary
refinement. But the sharp break that unfortunately prevails between the
kindergarten and the grades is evidence that the theoretical distinction
has practical implications. Under the title of play, the former is
rendered unduly symbolic, fanciful, sentimental, and arbitrary; while
under the antithetical caption of work the latter
contains
many \emph{tasks externally assigned}. The former has no end and the
latter an end so remote that only the educator, not the child, is aware
that it is an end.

There comes a time when children must extend and make more exact their
acquaintance with existing things; must conceive ends and consequences
with sufficient definiteness to guide their actions by them, and must
acquire some technical skill in selecting and arranging means to realize
these ends. Unless these factors are gradually introduced in the earlier
play period, they must be introduced later abruptly and arbitrarily, to
the manifest disadvantage of both the earlier and the later stages.

\marginpar{False notions of imagination and utility}

The sharp opposition of play and work is usually associated with false
notions of utility and imagination. Activity that is directed upon
matters of home and neighborhood interest is depreciated as merely
utilitarian. To let the child wash dishes, set the table, engage in
cooking, cut and sew dolls' clothes, make boxes that will hold "real
things," and construct his own playthings by using hammer and nails,
excludes, so it is said, the æsthetic and appreciative factor,
eliminates imagination, and subjects the child's development to material
and practical concerns; while (so it is said) to reproduce symbolically
the domestic relationships of birds and other animals, of human father
and mother and child, of workman and tradesman, of knight, soldier, and
magistrate, secures a liberal exercise of mind, of great moral as well
as intellectual value. It has been even stated that it is over-physical
and utilitarian if a child plants seeds and takes care of growing plants
in the kindergarten; while reproducing dramatically operations of
planting, cultivating, reaping, and so on,
either
with no physical materials or with symbolic representatives, is highly
educative to the imagination and to spiritual appreciation. Toy dolls,
trains of cars, boats, and engines are rigidly excluded, and the employ
of cubes, balls, and other symbols for representing these social
activities is recommended on the same ground. The more unfitted the
physical object for its imagined purpose, such as a cube for a boat, the
greater is the supposed appeal to the imagination.

\marginpar{Imagination a medium of realizing the absent and significant}

There are several fallacies in this way of thinking. (\emph{a}) The
healthy imagination deals not with the unreal, but with the mental
realization of what is suggested. Its exercise is not a flight into the
purely fanciful and ideal, but a method of expanding and filling in what
is real. To the child the homely activities going on about him are not
utilitarian devices for accomplishing physical ends; they exemplify a
wonderful world the depths of which he has not sounded, a world full of
the mystery and promise that attend all the doings of the grown-ups whom
he admires. However prosaic this world may be to the adults who find its
duties routine affairs, to the child it is fraught with social meaning.
To engage in it is to exercise the imagination in constructing an
experience of wider value than any the child has yet mastered.

\marginpar{Only the already experienced can be symbolized}

(\emph{b}) Educators sometimes think children are reacting to a great
moral or spiritual truth when the children's reactions are largely
physical and sensational. Children have great powers of dramatic
simulation, and their physical bearing may seem (to adults prepossessed
with a philosophic theory) to indicate they have been impressed with
some lesson of chivalry, devotion, or nobility, when the children
themselves are occupied
only
with transitory physical excitations. To symbolize great truths far
beyond the child's range of actual experience is an impossibility, and
to attempt it is to invite love of momentary stimulation.

\marginpar{Useful work is not necessarily labor}

(\emph{c}) Just as the opponents of play in education always conceive of
play as mere amusement, so the opponents of direct and useful activities
confuse occupation with labor. The adult is acquainted with responsible
labor upon which serious financial results depend. Consequently he seeks
relief, relaxation, amusement. Unless children have prematurely worked
for hire, unless they have come under the blight of child labor, no such
division exists for them. Whatever appeals to them at all, appeals
directly on its own account. There is no contrast between doing things
for utility and for fun. Their life is more united and more wholesome.
To suppose that activities customarily performed by adults only under
the pressure of utility may not be done perfectly freely and joyously by
children indicates a lack of imagination. Not the thing done but the
quality of mind that goes into the doing settles what is utilitarian and
what is unconstrained and educative.

\section{Constructive Occupations}

\marginpar{The historic growth of sciences out of occupations}

The history of culture shows that mankind's scientific knowledge and
technical abilities have developed, especially in all their earlier
stages, out of the fundamental problems of life. Anatomy and physiology
grew out of the practical needs of keeping healthy and active; geometry
and mechanics out of demands for measuring land, for building, and for
making labor-saving machines; astronomy has been closely connected with
navigation, keeping record of the passage of time; botany grew
out
of the requirements of medicine and of agronomy; chemistry has been
associated with dyeing, metallurgy, and other industrial pursuits. In
turn, modern industry is almost wholly a matter of applied science; year
by year the domain of routine and crude empiricism is narrowed by the
translation of scientific discovery into industrial invention. The
trolley, the telephone, the electric light, the steam engine, with all
their revolutionary consequences for social intercourse and control, are
the fruits of science.

\marginpar{The intellectual possibilities of school occupations}

These facts are full of educational significance. Most children are
preëminently active in their tendencies. The schools have also taken
on---largely from utilitarian, rather than from strictly educative
reasons---a large number of active pursuits commonly grouped under the
head of manual training, including also school gardens, excursions, and
various graphic arts. Perhaps the most pressing problem of education at
the present moment is to organize and relate these subjects so that they
will become instruments for forming alert, persistent, and fruitful
intellectual habits. That they take hold of the more primary and native
equipment of children (appealing to their desire to do) is generally
recognized; that they afford great opportunity for training in
self-reliant and efficient social service is gaining acknowledgment. But
they may also be used for presenting \emph{typical problems to be solved
by personal reflection and experimentation, and by acquiring definite
bodies of knowledge leading later to more specialized scientific
knowledge}. There is indeed no magic by which mere physical activity or
deft manipulation will secure intellectual results. (See p. 43.) Manual
subjects may be taught by routine, by dictation, or by convention as
readily
as bookish subjects. But intelligent consecutive work in gardening,
cooking, or weaving, or in elementary wood and iron, may be planned
which will inevitably result in students not only amassing information
of practical and scientific importance in botany, zoölogy, chemistry,
physics, and other sciences, but (what is more significant) in their
becoming versed in methods of experimental inquiry and proof.

\marginpar{Reorganization of the course of study}

That the elementary curriculum is overloaded is a common complaint. The
only alternative to a reactionary return to the educational traditions
of the past lies in working out the intellectual possibilities resident
in the various arts, crafts, and occupations, and reorganizing the
curriculum accordingly. Here, more than elsewhere, are found the means
by which the blind and routine experience of the race may be transformed
into illuminated and emancipated
experiment.

\chapter{Language and the Training of Thought}

\section{Language as the Tool of Thinking}

\marginpar{Ambiguous position of language}

Speech has such a peculiarly intimate connection with thought as to
require special discussion. Although the very word logic comes from
logos (\textgreek{λογος}), meaning indifferently both word or speech,
and thought or reason, yet "words, words, words" denote intellectual
barrenness, a sham of thought. Although schooling has language as its
chief instrument (and often as its chief matter) of study, educational
reformers have for centuries brought their severest indictments against
the current use of language in the schools. The conviction that language
is necessary to thinking (is even identical with it) is met by the
contention that language perverts and conceals thought.

\marginpar{Language a necessary tool of thinking,}

\marginpar{for it alone fixes meanings}

Three typical views have been maintained regarding the relation of
thought and language: first, that they are identical; second, that words
are the garb or clothing of thought, necessary not for thought but only
for conveying it; and third (the view we shall here maintain) that while
language is not thought it is necessary for thinking as well as for its
communication. When it is said, however, that thinking is impossible
without language, we must recall that language includes much more than
oral and written speech. Gestures, pictures, monuments, visual images,
finger movements---anything
consciously
employed as a \emph{sign} is, logically, language. To say that language
is necessary for thinking is to say that signs are necessary. Thought
deals not with bare things, but with their \emph{meanings}, their
suggestions; and meanings, in order to be apprehended, must be embodied
in sensible and particular existences. Without meaning, things are
nothing but blind stimuli or chance sources of pleasure and pain; and
since meanings are not themselves tangible things, they must be anchored
by attachment to some physical existence. Existences that are especially
set aside to fixate and convey meanings are \emph{signs} or
\emph{symbols}. If a man moves toward another to throw him out of the
room, his movement is not a sign. If, however, the man points to the
door with his hand, or utters the sound \emph{go}, his movement is
reduced to a vehicle of meaning: it is a sign or symbol. In the case of
signs we care nothing for what they are in themselves, but everything
for what they signify and represent. \emph{Canis}, \emph{hund},
\emph{chien}, dog---it makes no difference what the outward thing is, so
long as the meaning is presented.

\marginpar{Limitations of natural symbols}

Natural objects are signs of other things and events. Clouds stand for
rain; a footprint represents game or an enemy; a projecting rock serves
to indicate minerals below the surface. The limitations of natural signs
are, however, great. (\emph{i}) The physical or direct sense excitation
tends to distract attention from what is meant or
indicated.\footnote{ Compare the quotation from Bain on p. 155. }
Almost every one will recall pointing out to a kitten or puppy some
object of food, only to have the animal devote himself to the hand
pointing, not to the thing pointed at. (\emph{ii}) Where natural signs
alone exist, we are mainly at the mercy of external happenings;
we
have to wait until the natural event presents itself in order to be
warned or advised of the possibility of some other event. (\emph{iii})
Natural signs, not being originally intended to be signs, are cumbrous,
bulky, inconvenient, unmanageable.

\marginpar{Artificial signs overcome these restrictions.}

It is therefore indispensable for any high development of thought that
there should be also intentional signs. Speech supplies the requirement.
Gestures, sounds, written or printed forms, are strictly physical
existences, but their native value is intentionally subordinated to the
value they acquire as representative of meanings. (\emph{i}) The direct
and sensible value of faint sounds and minute written or printed marks
is very slight. Accordingly, attention is not distracted from their
\emph{representative} function. (\emph{ii}) Their production is under
our direct control so that they may be produced when needed. When we can
make the word \emph{rain}, we do not have to wait for some physical
forerunner of rain to call our thoughts in that direction. We cannot
make the cloud; we can make the sound, and as a token of meaning the
sound serves the purpose as well as the cloud. (\emph{iii}) Arbitrary
linguistic signs are convenient and easy to manage. They are compact,
portable, and delicate. As long as we live we breathe; and modifications
by the muscles of throat and mouth of the volume and quality of the air
are simple, easy, and indefinitely controllable. Bodily postures and
gestures of the hand and arm are also employed as signs, but they are
coarse and unmanageable compared with modifications of breath to produce
sounds. No wonder that oral speech has been selected as the main stuff
of intentional intellectual signs. Sounds, while subtle, refined, and
easily modifiable, are transitory. This defect is met by the system of
written
and printed words, appealing to the eye. \emph{Litera scripta manet.}

Bearing in mind the intimate connection of meanings and signs (or
language), we may note in more detail what language does (1) for
specific meanings, and (2) for the organization of meanings.

I. Individual Meanings. A verbal sign (\emph{a}) selects, detaches, a
meaning from what is otherwise a vague flux and blur (see p. 121);
(\emph{b}) it retains, registers, stores that meaning; and (\emph{c})
applies it, when needed, to the comprehension of other things. Combining
these various functions in a mixture of metaphors, we may say that a
linguistic sign is a fence, a label, and a vehicle---all in one.

\marginpar{A sign makes a meaning distinct}

(\emph{a}) Every one has experienced how learning an appropriate name
for what was dim and vague cleared up and crystallized the whole matter.
Some meaning seems almost within reach, but is elusive; it refuses to
condense into definite form; the attaching of a word somehow (just how,
it is almost impossible to say) puts limits around the meaning, draws it
out from the void, makes it stand out as an entity on its own account.
When Emerson said that he would almost rather know the true name, the
poet's name, for a thing, than to know the thing itself, he presumably
had this irradiating and illuminating function of language in mind. The
delight that children take in demanding and learning the names of
everything about them indicates that meanings are becoming concrete
individuals to them, so that their commerce with things is passing from
the physical to the intellectual plane. It is hardly surprising that
savages attach a magic efficacy to words. To name anything is to give it
a title; to dignify and honor it
by
raising it from a mere physical occurrence to a meaning that is distinct
and permanent. To know the names of people and things and to be able to
manipulate these names is, in savage lore, to be in possession of their
dignity and worth, to master them.

\marginpar{A sign preserves a meaning}

(\emph{b}) Things come and go; or we come and go, and either way things
escape our notice. Our direct sensible relation to things is very
limited. The suggestion of meanings by natural signs is limited to
occasions of direct contact or vision. But a meaning fixed by a
linguistic sign is conserved for future use. Even if the thing is not
there to represent the meaning, the word may be produced so as to evoke
the meaning. Since intellectual life depends on possession of a store of
meanings, the importance of language as a tool of preserving meanings
cannot be overstated. To be sure, the method of storage is not wholly
aseptic; words often corrupt and modify the meanings they are supposed
to keep intact, but liability to infection is a price paid by every
living thing for the privilege of living.

\marginpar{A sign transfers a meaning}

(\emph{c}) When a meaning is detached and fixed by a sign, it is
possible to use that meaning in a new context and situation. This
transfer and reapplication is the key to all judgment and inference. It
would little profit a man to recognize that a given particular cloud was
the premonitor of a given particular rainstorm if his recognition ended
there, for he would then have to learn over and over again, since the
next cloud and the next rain are different events. No cumulative growth
of intelligence would occur; experience might form habits of physical
adaptation but it would not teach anything, for we should not be able to
use a prior experience consciously to anticipate and regulate a further
experience. To be able to
use
the past to judge and infer the new and unknown implies that, although
the past thing has gone, its \emph{meaning} abides in such a way as to
be applicable in determining the character of the new. Speech forms are
our great carriers: the easy-running vehicles by which meanings are
transported from experiences that no longer concern us to those that are
as yet dark and dubious.

\marginpar{Logical organization depends upon signs}

II. Organization of Meanings. In emphasizing the importance of signs in
relation to specific meanings, we have overlooked another aspect,
equally valuable. Signs not only mark off specific or individual
meanings, but they are also instruments of grouping meanings in relation
to one another. Words are not only names or titles of single meanings;
they also form \emph{sentences} in which meanings are organized in
relation to one another. When we say "That book is a dictionary," or
"That blur of light in the heavens is Halley's comet," we express a
\emph{logical} connection---an act of classifying and defining that goes
beyond the physical thing into the logical region of genera and species,
things and attributes. Propositions, sentences, bear the same relation
to judgments that distinct words, built up mainly by analyzing
propositions in their various types, bear to meanings or conceptions;
and just as words imply a sentence, so a sentence implies a larger whole
of consecutive discourse into which it fits. As is often said, grammar
expresses the unconscious logic of the popular mind. \emph{The chief
intellectual classifications that constitute the working capital of
thought have been built up for us by our mother tongue.} Our very lack
of explicit consciousness in using language that we are employing the
intellectual systematizations of the race shows how thoroughly
accustomed we have become to its logical distinctions and
groupings.

\section{The Abuse of Linguistic Methods in Education}

\marginpar{Teaching merely things, not educative}

Taken literally, the maxim, "Teach things, not words," or "Teach things
before words," would be the negation of education; it would reduce
mental life to mere physical and sensible adjustments. Learning, in the
proper sense, is not learning things, but the \emph{meanings} of things,
and this process involves the use of signs, or language in its generic
sense. In like fashion, the warfare of some educational reformers
against symbols, if pushed to extremes, involves the destruction of the
intellectual life, since this lives, moves, and has its being in those
processes of definition, abstraction, generalization, and classification
that are made possible by symbols alone. Nevertheless, these contentions
of educational reformers have been needed. The liability of a thing to
abuse is in proportion to the value of its right use.

\marginpar{But words separated from things are not true signs}

Symbols are themselves, as pointed out above, particular, physical,
sensible existences, like any other things. They are symbols only by
virtue of what they suggest and represent, \emph{i.e.} meanings.
(\emph{i}) They stand for these meanings to any individual only when he
has had \emph{experience} of some situation to which these meanings are
actually relevant. Words can detach and preserve a meaning only when the
meaning has been first involved in our own direct intercourse with
things. To attempt to give a meaning through a word alone without any
dealings with a thing is to deprive the word of intelligible
signification; against this attempt, a tendency only too prevalent in
education, reformers have protested. Moreover, there is a tendency to
assume that whenever there is a definite word or form of speech there is
also a definite idea; while, as a matter of fact, adults and children
alike are capable of using even precise verbal
formulæ
with only the vaguest and most confused sense of what they mean. Genuine
ignorance is more profitable because likely to be accompanied by
humility, curiosity, and open-mindedness; while ability to repeat
catch-phrases, cant terms, familiar propositions, gives the conceit of
learning and coats the mind with a varnish waterproof to new ideas.

\marginpar{Language tends to arrest personal inquiry and reflection}

(\emph{ii}) Again, although new combinations of words without the
intervention of physical things may supply new ideas, there are limits
to this possibility. Lazy inertness causes individuals to accept ideas
that have currency about them without personal inquiry and testing. A
man uses thought, perhaps, to find out what others believe, and then
stops. The ideas of others as embodied in language become substitutes
for one's own ideas. The use of linguistic studies and methods to halt
the human mind on the level of the attainments of the past, to prevent
new inquiry and discovery, to put the authority of tradition in place of
the authority of natural facts and laws, to reduce the individual to a
parasite living on the secondhand experience of others---these things
have been the source of the reformers' protest against the preëminence
assigned to language in schools.

\marginpar{Words as mere stimuli}

Finally, words that originally stood for ideas come, with repeated use,
to be mere counters; they become physical things to be manipulated
according to certain rules, or reacted to by certain operations without
consciousness of their meaning. Mr.\ Stout (who has called such terms
"substitute signs")remarks that "algebraical and arithmetical signs are
to a great extent used as mere substitute signs.... It is possible to
use signs of this kind whenever fixed and definite rules of
operation
can be derived from the nature of the things symbolized, so as to be
applied in manipulating the signs, without further reference to their
signification. A word is an instrument for thinking about the meaning
which it expresses; a substitute sign is a means of \emph{not} thinking
about the meaning which it symbolizes." The principle applies, however,
to ordinary words, as well as to algebraic signs; they also enable us to
use meanings so as to get results without thinking. In many respects,
signs that are means of not thinking are of great advantage; standing
for the familiar, they release attention for meanings that, being novel,
require conscious interpretation. Nevertheless, the premium put in the
schoolroom upon attainment of technical facility, upon skill in
producing external results (\emph{ante}, p. 51), often changes this
advantage into a positive detriment. In manipulating symbols so as to
recite well, to get and give correct answers, to follow prescribed
formulæ of analysis, the pupil's attitude becomes mechanical, rather
than thoughtful; verbal memorizing is substituted for inquiry into the
meaning of things. This danger is perhaps the one uppermost in mind when
verbal methods of education are attacked.

\section{The Use of Language in its Educational Bearings}

Language stands in a twofold relation to the work of education. On the
one hand, it is continually used in all studies as well as in all the
social discipline of the school; on the other, it is a distinct object
of study. We shall consider only the ordinary use of language, since its
effects upon habits of thought are much deeper than those of conscious
study.

\marginpar{Language not primarily intellectual in purpose}

The common statement that "language is the
expression
of thought" conveys only a half-truth, and a half-truth that is likely
to result in positive error. Language does express thought, but not
primarily, nor, at first, even consciously. The primary motive for
language is to influence (through the expression of desire, emotion, and
thought) the activity of others; its secondary use is to enter into more
intimate sociable relations with them; its employment as a conscious
vehicle of thought and knowledge is a tertiary, and relatively late,
formation. The contrast is well brought out by the statement of John
Locke that words have a double use,---"civil" and "philosophical." "By
their civil use, I mean such a communication of thoughts and ideas by
words as may serve for the upholding of common conversation and commerce
about the ordinary affairs and conveniences of civil life.... By the
philosophical use of words, I mean such a use of them as may serve to
convey the precise notions of things, and to express in general
propositions certain and undoubted truths."

\marginpar{Hence education has to transform it into an intellectual tool}

This distinction of the practical and social from the intellectual use
of language throws much light on the problem of the school in respect to
speech. That problem is \emph{to direct pupils' oral and written speech,
used primarily for practical and social ends, so that gradually it shall
become a conscious tool of conveying knowledge and assisting thought}.
How without checking the spontaneous, natural motives---motives to which
language owes its vitality, force, vividness, and variety---are we to
modify speech habits so as to render them accurate and flexible
\emph{intellectual} instruments? It is comparatively easy to encourage
the original spontaneous flow and not make language over into a servant
of reflective thought; it is comparatively easy to check
and
almost destroy (so far as the schoolroom is concerned) native aim and
interest, and to set up artificial and formal modes of expression in
some isolated and technical matters. The difficulty lies in making over
habits that have to do with "ordinary affairs and conveniences" into
habits concerned with "precise notions." The successful accomplishing of
the transformation requires (\emph{i}) enlargement of the pupil's
vocabulary; (\emph{ii}) rendering its terms more precise and accurate,
and (\emph{iii}) formation of habits of consecutive discourse.

\marginpar{To enlarge vocabulary, the fund of concepts should be enlarged}

(\emph{i}) Enlargement of vocabulary. This takes place, of course, by
wider intelligent contact with things and persons, and also vicariously,
by gathering the meanings of words from the context in which they are
heard or read. To grasp by either method a word in its meaning is to
exercise intelligence, to perform an act of intelligent selection or
analysis, and it is also to widen the fund of meanings or concepts
readily available in further intellectual enterprises (\emph{ante}, p.
126). It is usual to distinguish between one's active and one's passive
vocabulary, the latter being composed of the words that are understood
when they are heard or seen, the former of words that are used
intelligently. The fact that the passive vocabulary is ordinarily much
larger than the active indicates a certain amount of inert energy, of
power not freely controlled by an individual. Failure to use meanings
that are nevertheless understood reveals dependence upon external
stimulus, and lack of intellectual initiative. This mental laziness is
to some extent an artificial product of education. Small children
usually attempt to put to use every new word they get hold of, but when
they learn to read they are introduced to a large variety of terms that
there is no ordinary opportunity to
use.
The result is a kind of mental suppression, if not smothering. Moreover,
the meaning of words not actively used in building up and conveying
ideas is never quite clear-cut or complete.

\marginpar{Looseness of thinking accompanies a limited vocabulary}

While a limited vocabulary may be due to a limited range of experience,
to a sphere of contact with persons and things so narrow as not to
suggest or require a full store of words, it is also due to carelessness
and vagueness. A happy-go-lucky frame of mind makes the individual
averse to clear discriminations, either in perception or in his own
speech. Words are used loosely in an indeterminate kind of reference to
things, and the mind approaches a condition where practically everything
is just a thing-um-bob or a what-do-you-call-it. Paucity of vocabulary
on the part of those with whom the child associates, triviality and
meagerness in the child's reading matter (as frequently even in his
school readers and text-books), tend to shut down the area of mental
vision.

\marginpar{Command of language involves command of things}

We must note also the great difference between flow of words and command
of language. Volubility is not necessarily a sign of a large vocabulary;
much talking or even ready speech is quite compatible with moving round
and round in a circle of moderate radius. Most schoolrooms suffer from a
lack of materials and appliances save perhaps books---and even these are
"written down" to the supposed capacity, or incapacity, of children.
Occasion and demand for an enriched vocabulary are accordingly
restricted. The vocabulary of things studied in the schoolroom is very
largely isolated; it does not link itself organically to the range of
the ideas and words that are in vogue outside the school. Hence the
enlargement that takes place is often
nominal,
adding to the inert, rather than to the active, fund of meanings and
terms.

(\emph{ii}) Accuracy of vocabulary. One way in which the fund of words
and concepts is increased is by discovering and naming shades of
meaning---that is to say, by making the vocabulary more precise.
Increase in definiteness is as important relatively as is the
enlargement of the capital stock absolutely.

\marginpar{The \emph{general} as the vague and as the distinctly generic}

The first meanings of terms, since they are due to superficial
acquaintance with things, are general in the sense of being vague. The
little child calls all men papa; acquainted with a dog, he may call the
first horse he sees a big dog. Differences of quantity and intensity are
noted, but the fundamental meaning is so vague that it covers things
that are far apart. To many persons trees are just trees, being
discriminated only into deciduous trees and evergreens, with perhaps
recognition of one or two kinds of each. Such vagueness tends to persist
and to become a barrier to the advance of thinking. Terms that are
miscellaneous in scope are clumsy tools at best; in addition they are
frequently treacherous, for their ambiguous reference causes us to
confuse things that should be distinguished.

\marginpar{Twofold growth of words in sense or signification}

The growth of precise terms out of original vagueness takes place
normally in two directions: toward words that stand for relationships
and words that stand for highly individualized traits (compare what was
said about the development of meanings, p. 122); the first being
associated with abstract, the second with concrete, thinking. Some
Australian tribes are said to have no words for \emph{animal} or for
\emph{plant}, while they have specific names for every variety of plant
and animal in their neighborhoods. This minuteness of vocabulary
represents
progress toward definiteness, but in a one-sided way. Specific
properties are distinguished, but not
relationships.\footnote{
The term \emph{general} is itself an ambiguous term, meaning (in its
best logical sense) the related and also (in its natural usage) the
indefinite, the vague. \emph{General}, in the first sense, denotes the
discrimination of a principle or generic relation; in the second sense,
it denotes the absence of discrimination of specific or individual
properties.
}
On the other hand, students of philosophy and of the general aspects of
natural and social science are apt to acquire a store of terms that
signify relations without balancing them up with terms that designate
specific individuals and traits. The ordinary use of such terms as
\emph{causation}, \emph{law}, \emph{society}, \emph{individual},
\emph{capital}, illustrates this tendency.

\marginpar{Words alter their meanings so as to change their logical functions}

In the history of language we find both aspects of the growth of
vocabulary illustrated by changes in the sense of words: some words
originally wide in their application are narrowed to denote shades of
meaning; others originally specific are widened to express
relationships. The term \emph{vernacular}, now meaning mother speech,
has been generalized from the word \emph{verna}, meaning a slave born in
the master's household. \emph{Publication} has evolved its meaning of
communication by means of print, through restricting an earlier meaning
of any kind of communication---although the wider meaning is retained in
legal procedure, as publishing a libel. The sense of the word
\emph{average} has been generalized from a use connected with dividing
loss by shipwreck proportionately among various sharers in an
enterprise.\footnote{
A large amount of material illustrating the twofold change in the sense
of words will be found in Jevons, \emph{Lessons in Logic}.
}

\marginpar{Similar changes occur in the vocabulary of every student}

These historical changes assist the educator to appreciate the changes
that occur with individuals together with advance in intellectual
resources. In
studying
geometry, a pupil must learn both to narrow and to extend the meanings
of such familiar words as \emph{line}, \emph{surface}, \emph{angle},
\emph{square}, \emph{circle}; to narrow them to the precise meanings
involved in demonstrations; to extend them to cover generic relations
not expressed in ordinary usage. Qualities of color and size must be
excluded; relations of direction, of variation in direction, of limit,
must be definitely seized. A like transformation occurs, of course, in
every subject of study. Just at this point lies the danger, alluded to
above, of simply overlaying common meanings with new and isolated
meanings instead of effecting a genuine working-over of popular and
practical meanings into adequate logical tools.

\marginpar{The value of technical terms}

Terms used with intentional exactness so as to express a meaning, the
whole meaning, and only the meaning, are called \emph{technical}. For
educational purposes, a technical term indicates something relative, not
absolute; for a term is technical not because of its verbal form or its
unusualness, but because it is employed to fix a meaning precisely.
Ordinary words get a technical quality when used intentionally for this
end. Whenever thought becomes more accurate, a (relatively) technical
vocabulary grows up. Teachers are apt to oscillate between extremes in
regard to technical terms. On the one hand, these are multiplied in
every direction, seemingly on the assumption that learning a new piece
of terminology, accompanied by verbal description or definition, is
equivalent to grasping a new idea. When it is seen how largely the net
outcome is the accumulation of an isolated set of words, a jargon or
scholastic cant, and to what extent the natural power of judgment is
clogged by this accumulation, there is a reaction to the opposite
extreme. Technical terms are
banished:
"name words" exist but not nouns; "action words" but not verbs; pupils
may "take away," but not subtract; they may tell what four fives are,
but not what four times five are, and so on. A sound instinct underlies
this reaction---aversion to words that give the pretense, but not the
reality, of meaning. Yet the fundamental difficulty is not with the
word, but with the idea. If the idea is not grasped, nothing is gained
by using a more familiar word; if the idea is perceived, the use of the
term that exactly names it may assist in fixing the idea. Terms denoting
highly exact meanings should be introduced only sparingly, that is, a
few at a time; they should be led up to gradually, and great pains
should be taken to secure the circumstances that render precision of
meaning significant.

\marginpar{Importance of consecutive discourse}

(\emph{iii}) Consecutive discourse. As we saw, language connects and
organizes meanings as well as selects and fixes them. As every meaning
is set in the context of some situation, so every word in concrete use
belongs to some sentence (it may itself represent a condensed sentence),
and the sentence, in turn, belongs to some larger story, description, or
reasoning process. It is unnecessary to repeat what has been said about
the importance of continuity and ordering of meanings. We may, however,
note some ways in which school practices tend to interrupt
consecutiveness of language and thereby interfere harmfully with
systematic reflection. (\emph{a}) Teachers have a habit of monopolizing
continued discourse. Many, if not most, instructors would be surprised
if informed at the end of the day of the amount of time they have talked
as compared with any pupil. Children's conversation is often confined to
answering questions in brief phrases, or in single disconnected
sentences.
Expatiation
and explanation are reserved for the teacher, who often admits any hint
at an answer on the part of the pupil, and then amplifies what he
supposes the child must have meant. The habits of sporadic and
fragmentary discourse thus promoted have inevitably a disintegrating
intellectual influence.

\marginpar{Too minute questioning}

(\emph{b}) Assignment of too short lessons when accompanied (as it
usually is in order to pass the time of the recitation period) by minute
"analytic" questioning has the same effect. This evil is usually at its
height in such subjects as history and literature, where not
infrequently the material is so minutely subdivided as to break up the
unity of meaning belonging to a given portion of the matter, to destroy
perspective, and in effect to reduce the whole topic to an accumulation
of disconnected details all upon the same level. More often than the
teacher is aware, \emph{his} mind carries and supplies the background of
unity of meaning against which pupils project isolated scraps.

\marginpar{Making avoidance of error the aim}

(\emph{c}) Insistence upon avoiding error instead of attaining power
tends also to interruption of continuous discourse and thought. Children
who begin with something to say and with intellectual eagerness to say
it are sometimes made so conscious of minor errors in substance and form
that the energy that should go into constructive thinking is diverted
into anxiety not to make mistakes, and even, in extreme cases, into
passive quiescence as the best method of minimizing error. This tendency
is especially marked in connection with the writing of compositions,
essays, and themes. It has even been gravely recommended that little
children should always write on trivial subjects and in short sentences
because in that way they are less likely to make mistakes,
while
the teaching of writing to high school and college students occasionally
reduces itself to a technique for detecting and designating mistakes.
The resulting self-consciousness and constraint are only part of the
evil that comes from a negative
ideal.

\chapter{Observation and Information in the Training of Mind}

\marginpar{No thinking without acquaintance with facts}

Thinking is an ordering of subject-matter with reference to discovering
what it signifies or indicates. Thinking no more exists apart from this
arranging of subject-matter than digestion occurs apart from the
assimilating of food. The way in which the subject-matter is furnished
marks, therefore, a fundamental point. If the subject-matter is provided
in too scanty or too profuse fashion, if it comes in disordered array or
in isolated scraps, the effect upon habits of thought is detrimental. If
personal observation and communication of information by others (whether
in books or speech) are rightly conducted, half the logical battle is
won, for they are the channels of obtaining subject-matter.

\section{The Nature and Value of Observation}

\marginpar{Fallacy of making "facts" an end in themselves}

The protest, mentioned in the last chapter, of educational reformers
against the exaggerated and false use of language, insisted upon
personal and direct observation as the proper alternative course. The
reformers felt that the current emphasis upon the linguistic factor
eliminated all opportunity for first-hand acquaintance with real things;
hence they appealed to sense-perception to fill the gap. It is not
surprising that this enthusiastic zeal failed frequently to ask how and
why
observation is educative, and hence fell into the error of making
observation an end in itself and was satisfied with any kind of material
under any kind of conditions. Such isolation of observation is still
manifested in the statement that this faculty develops first, then that
of memory and imagination, and finally the faculty of thought. From this
point of view, observation is regarded as furnishing crude masses of raw
material, to which, later on, reflective processes may be applied. Our
previous pages should have made obvious the fallacy of this point of
view by bringing out the fact that simple concrete thinking attends all
our intercourse with things which is not on a purely physical level.

\marginpar{The sympathetic motive in extending acquaintance}

I. All persons have a natural desire---akin to curiosity---for a
widening of their range of acquaintance with persons and things. The
sign in art galleries that forbids the carrying of canes and umbrellas
is obvious testimony to the fact that simply to see is not enough for
many people; there is a feeling of lack of acquaintance until some
direct contact is made. This demand for fuller and closer knowledge is
quite different from any conscious interest in observation for its own
sake. Desire for expansion, for "self-realization," is its motive. The
interest is sympathetic, socially and æsthetically sympathetic, rather
than cognitive. While the interest is especially keen in children
(because their actual experience is so small and their possible
experience so large), it still characterizes adults when routine has not
blunted its edge. This sympathetic interest provides the medium for
carrying and binding together what would otherwise be a multitude of
items, diverse, disconnected, and of no intellectual use. These systems
are indeed social and æsthetic rather than consciously
intellectual;
but they provide the natural medium for more conscious intellectual
explorations. Some educators have recommended that nature study in the
elementary schools be conducted with a love of nature and a cultivation
of æsthetic appreciation in view rather than in a purely analytic
spirit. Others have urged making much of the care of animals and plants.
Both of these important recommendations have grown out of experience,
not out of theory, but they afford excellent exemplifications of the
theoretic point just made.

\marginpar{Analytic inspection for the sake of doing}

\marginpar{Direct and indirect sense training}

II. In normal development, specific analytic observations are originally
connected almost exclusively with the imperative need for noting means
and ends in carrying on activities. When one is \emph{doing} something,
one is compelled, if the work is to succeed (unless it is purely
routine), to use eyes, ears, and sense of touch as guides to action.
Without a constant and alert exercise of the senses, not even plays and
games can go on; in any form of work, materials, obstacles, appliances,
failures, and successes, must be intently watched. Sense-perception does
not occur for its own sake or for purposes of training, but because it
is an indispensable factor of success in doing what one is interested in
doing. Although not designed for sense-training, this method effects
sense-training in the most economical and thoroughgoing way. Various
schemes have been designed by teachers for cultivating sharp and prompt
observation of forms, as by writing words,---even in an unknown
language,---making arrangements of figures and geometrical forms, and
having pupils reproduce them after a momentary glance. Children often
attain great skill in quick seeing and full reproducing of even
complicated meaningless combinations. But such methods of
training---however
valuable as occasional games and diversions---compare very unfavorably
with the training of eye and hand that comes as an incident of work with
tools in wood or metals, or of gardening, cooking, or the care of
animals. Training by isolated exercises leaves no deposit, leads
nowhere; and even the technical skill acquired has little radiating
power, or transferable value. Criticisms made upon the training of
observation on the ground that many persons cannot correctly reproduce
the forms and arrangement of the figures on the face of their watches
misses the point because persons do not look at a watch to find out
whether four o'clock is indicated by IIII or by IV, but to find out what
time it is, and, if observation decides this matter, noting other
details is irrelevant and a waste of time. In the training of
observation the question of end and motive is all-important.

\marginpar{Scientific observations are linked to problems}

\marginpar{"Object-lessons" rarely supply problems}

III. The further, more intellectual or scientific, development of
observation follows the line of the growth of practical into theoretical
reflection already traced (\emph{ante},
\protect\hyperlink{ux40publicux40vhostux40gux40gutenbergux40htmlux40filesux4037423ux4037423-hux4037423-h-5.htm.htmlux5cux23CHAPTER_TEN}{Chapter
Ten}). As problems emerge and are dwelt upon, observation is directed
less to the facts that bear upon a practical aim and more upon what
bears upon a problem as such. What makes observations in schools often
intellectually ineffective is (more than anything else) that they are
carried on independently of a sense of a problem that they serve to
define or help to solve. The evil of this isolation is seen through the
entire educational system, from the kindergarten, through the elementary
and high schools, to the college. Almost everywhere may be found, at
some time, recourse to observations as if they were of complete and
final value in themselves, instead of the
means
of getting material that bears upon some difficulty and its solution. In
the kindergarten are heaped up observations regarding geometrical forms,
lines, surfaces, cubes, colors, and so on. In the elementary school,
under the name of "object-lessons," the form and properties of
objects,---apple, orange, chalk,---selected almost at random, are
minutely noted, while under the name of "nature study" similar
observations are directed upon leaves, stones, insects, selected in
almost equally arbitrary fashion. In high school and college, laboratory
and microscopic observations are carried on as if the accumulation of
observed facts and the acquisition of skill in manipulation were
educational ends in themselves.

Compare with these methods of isolated observations the statement of
Jevons that observation as conducted by scientific men is effective
"only when excited and guided by hope of verifying a theory"; and again,
"the number of things which can be observed and experimented upon are
infinite, and if we merely set to work to record facts without any
distinct purpose, our records will have no value." Strictly speaking,
the first statement of Jevons is too narrow. Scientific men institute
observations not merely to test an idea (or suggested explanatory
meaning), but also to locate the nature of a problem and thereby guide
the formation of a hypothesis. But the principle of his remark, namely,
that scientific men never make the accumulation of observations an end
in itself, but always a means to a general intellectual conclusion, is
absolutely sound. Until the force of this principle is adequately
recognized in education, observation will be largely a matter of
uninteresting dead work or of acquiring forms of technical skill that
are not available as intellectual
resources.

\section{Methods and Materials of Observation in the Schools} The best
methods in use in our schools furnish many suggestions for giving
observation its right place in mental training.

\marginpar{Observation should involve discovery}

I. They rest upon the sound assumption that observation is an
\emph{active} process. Observation is exploration, inquiry for the sake
of discovering something previously hidden and unknown, this something
being needed in order to reach some end, practical or theoretical.
Observation is to be discriminated from recognition, or perception of
what is familiar. The identification of something already understood is,
indeed, an indispensable function of further investigation (\emph{ante},
p. 119); but it is relatively automatic and passive, while observation
proper is searching and deliberate. Recognition refers to the already
mastered; observation is concerned with mastering the unknown. The
common notions that perception is like writing on a blank piece of
paper, or like impressing an image on the mind as a seal is imprinted on
wax or as a picture is formed on a photographic plate (notions that have
played a disastrous rôle in educational methods), arise from a failure
to distinguish between automatic recognition and the searching attitude
of genuine observation.

\marginpar{and suspense during an unfolding change}

II. Much assistance in the selection of appropriate material for
observation may be derived from considering the eagerness and closeness
of observation that attend the following of a story or drama. Alertness
of observation is at its height wherever there is "plot interest." Why?
Because of the balanced combination of the old and the new, of the
familiar and the unexpected. We hang on the lips of the story-teller
because of the element of mental suspense. Alternatives are
suggested,
but are left ambiguous, so that our whole being questions: What befell
next? Which way did things turn out? Contrast the ease and fullness with
which a child notes all the salient traits of a story, with the labor
and inadequacy of his observation of some dead and static thing where
nothing raises a question or suggests alternative outcomes.

\marginpar{This "plot interest" manifested in activity,}

When an individual is engaged in doing or making something (the activity
not being of such a mechanical and habitual character that its outcome
is assured), there is an analogous situation. Something is going to come
of what is present to the sense, but just what is doubtful. The plot is
unfolding toward success or failure, but just when or how is uncertain.
Hence the keen and tense observation of conditions and results that
attends constructive manual operations. Where the subject-matter is of a
more impersonal sort, the same principle of movement toward a dénouement
may apply. It is a commonplace that what is moving attracts notice when
that which is at rest escapes it. Yet too often it would almost seem as
if pains had been taken to deprive the material of school observations
of all life and dramatic quality, to reduce it to a dead and inert form.
Mere change is not enough, however. Vicissitude, alteration, motion,
excite observation; but if they merely excite it, there is no thought.
The changes must (like the incidents of a well-arranged story or plot)
take place in a certain cumulative order; each successive change must at
once remind us of its predecessor and arouse interest in its successor
if observations of change are to be logically fruitful.

\marginpar{and in cycles of growth}

Living beings, plants, and animals, fulfill the twofold requirement to
an extraordinary degree. Where
there
is growth, there is motion, change, process; and there is also
arrangement of the changes in a cycle. The first arouses, the second
organizes, observation. Much of the extraordinary interest that children
take in planting seeds and watching the stages of their growth is due to
the fact that a drama is enacting before their eyes; there is something
doing, each step of which is important in the destiny of the plant. The
great practical improvements that have occurred of late years in the
teaching of botany and zoölogy will be found, upon inspection, to
involve treating plants and animals as beings that act, that do
something, instead of as mere inert specimens having static properties
to be inventoried, named, and registered. Treated in the latter fashion,
observation is inevitably reduced to the falsely "analytic"
(\emph{ante}, p. 112),---to mere dissection and enumeration.

\marginpar{Observation of structure grows out of noting function}

There is, of course, a place, and an important place, for observation of
the mere static qualities of objects. When, however, the primary
interest is in \emph{function}, in what the object does, there is a
motive for more minute analytic study, for the observation of
\emph{structure}. Interest in noting an activity passes insensibly into
noting how the activity is carried on; the interest in what is
accomplished passes over into an interest in the organs of its
accomplishing. But when the beginning is made with the morphological,
the anatomical, the noting of peculiarities of form, size, color, and
distribution of parts, the material is so cut off from significance as
to be dead and dull. It is as natural for children to look intently for
the \emph{stomata} of a plant after they have become interested in its
function of breathing, as it is repulsive to attend minutely to them
when they are considered as isolated peculiarities of
structure.

\marginpar{Scientific observation}

III. As the center of interest of observations becomes less personal,
less a matter of means for effecting one's own ends, and less æsthetic,
less a matter of contribution of parts to a total emotional effect,
observation becomes more consciously intellectual in quality. Pupils
learn to observe for the sake (\emph{i}) of finding out what sort of
perplexity confronts them; (\emph{ii}) of inferring hypothetical
explanations for the puzzling features that observation reveals; and
(\emph{iii}) of testing the ideas thus suggested.

\marginpar{should be extensive}

\marginpar{and intensive}

In short, observation becomes scientific in nature. Of such observations
it may be said that they should follow a rhythm between the extensive
and the intensive. Problems become definite, and suggested explanations
significant by a certain alternation between a wide and somewhat loose
soaking in of relevant facts and a minutely accurate study of a few
selected facts. The wider, less exact observation is necessary to give
the student a feeling for the reality of the field of inquiry, a sense
of its bearings and possibilities, and to store his mind with materials
that imagination may transform into suggestions. The intensive study is
necessary for limiting the problem, and for securing the conditions of
experimental testing. As the latter by itself is too specialized and
technical to arouse intellectual growth, the former by itself is too
superficial and scattering for control of intellectual development. In
the sciences of life, field study, excursions, acquaintance with living
things in their natural habitats, may alternate with microscopic and
laboratory observation. In the physical sciences, phenomena of light, of
heat, of electricity, of moisture, of gravity, in their broad setting in
nature---their physiographic setting---should prepare for an exact study
of selected facts under conditions of
laboratory
control. In this way, the student gets the benefit of technical
scientific methods of discovery and testing, while he retains his sense
of the identity of the laboratory modes of energy with large out-of-door
realities, thereby avoiding the impression (that so often accrues) that
the facts studied are peculiar to the laboratory.

\section{Communication of Information}

\marginpar{Importance of hearsay acquaintance}

When all is said and done the field of fact open to any one observer by
himself is narrow. Into every one of our beliefs, even those that we
have worked out under the conditions of utmost personal, first-hand
acquaintance, much has insensibly entered from what we have heard or
read of the observations and conclusions of others. In spite of the
great extension of direct observation in our schools, the vast bulk of
educational subject-matter is derived from other sources---from
text-book, lecture, and viva-voce interchange. No educational question
is of greater import than how to get the most logical good out of
learning through transmission from others.

\marginpar{Logically, this ranks only as evidence or testimony}

Doubtless the chief meaning associated with the word \emph{instruction}
is this conveying and instilling of the results of the observations and
inferences of others. Doubtless the undue prominence in education of the
ideal of amassing information (\emph{ante}, p. 52) has its source in the
prominence of the learning of other persons. The problem then is how to
convert it into an intellectual asset. In logical terms, the material
supplied from the experience of others is \emph{testimony}: that is to
say, \emph{evidence} submitted by others to be employed by one's own
judgment in reaching a conclusion. How shall we treat the subject-matter
supplied by text-book and teacher so that it shall rank as material for
reflective
inquiry, not as ready-made intellectual pabulum to be accepted and
swallowed just as supplied by the store?

\marginpar{Communication by others should not encroach on observation,}

In reply to this question, we may say (\emph{i}) that the communication
of material should be \emph{needed}. That is to say, it should be such
as cannot readily be attained by personal observation. For teacher or
book to cram pupils with facts which, with little more trouble, they
could discover by direct inquiry is to violate their intellectual
integrity by cultivating mental servility. This does not mean that the
material supplied through communication of others should be meager or
scanty. With the utmost range of the senses, the world of nature and
history stretches out almost infinitely beyond. But the fields within
which direct observation is feasible should be carefully chosen and
sacredly protected.

\marginpar{should not be dogmatic in tone,}

(\emph{ii}) Material should be supplied by way of stimulus, not with
dogmatic finality and rigidity. When pupils get the notion that any
field of study has been definitely surveyed, that knowledge about it is
exhaustive and final, they may continue docile pupils, but they cease to
be students. All thinking whatsoever---so be it \emph{is}
thinking---contains a phase of originality. This originality does not
imply that the student's conclusion varies from the conclusions of
others, much less that it is a radically novel conclusion. His
originality is not incompatible with large use of materials and
suggestions contributed by others. Originality means personal interest
in the question, personal initiative in turning over the suggestions
furnished by others, and sincerity in following them out to a tested
conclusion. Literally, the phrase "Think for yourself" is tautological;
any thinking is thinking for one's
self.

\marginpar{should have relation to a personal problem,}

(\emph{iii}) The material furnished by way of information should be
relevant to a question that is vital in the student's own experience.
What has been said about the evil of observations that begin and end in
themselves may be transferred without change to communicated learning.
Instruction in subject-matter that does not fit into any problem already
stirring in the student's own experience, or that is not presented in
such a way as to arouse a problem, is worse than useless for
intellectual purposes. In that it fails to enter into any process of
reflection, it is useless; in that it remains in the mind as so much
lumber and débris, it is a barrier, an obstruction in the way of
effective thinking when a problem arises.

\marginpar{and to prior systems of experience}

Another way of stating the same principle is that material furnished by
communication must be such as to enter into some existing system or
organization of experience. All students of psychology are familiar with
the principle of apperception---that we assimilate new material with
what we have digested and retained from prior experiences. Now the
"apperceptive basis" of material furnished by teacher and text-book
should be found, as far as possible, in what the learner has derived
from more direct forms of his own experience. There is a tendency to
connect material of the schoolroom simply with the material of prior
school lessons, instead of linking it to what the pupil has acquired in
his out-of-school experience. The teacher says, "Do you not remember
what we learned from the book last week?"---instead of saying, "Do you
not recall such and such a thing that you have seen or heard?" As a
result, there are built up detached and independent systems of school
knowledge that inertly overlay
the
ordinary systems of experience instead of reacting to enlarge and refine
them. Pupils are taught to live in two separate worlds, one the world of
out-of-school experience, the other the world of books and
lessons.

\chapter{The Recitation and the Training of Thought}

\marginpar{Importance of the recitation}

In the recitation the teacher comes into his closest contact with the
pupil. In the recitation focus the possibilities of guiding children's
activities, influencing their language habits, and directing their
observations. In discussing the significance of the recitation as an
instrumentality of education, we are accordingly bringing to a head the
points considered in the last three chapters, rather than introducing a
new topic. The method in which the recitation is carried on is a crucial
test of a teacher's skill in diagnosing the intellectual state of his
pupils and in supplying the conditions that will arouse serviceable
mental responses: in short, of his art as a teacher.

\marginpar{Re-citing \emph{versus} reflecting}

The use of the word \emph{recitation} to designate the period of most
intimate intellectual contact of teacher with pupil and pupil with pupil
is a fateful fact. To re-cite is to cite again, to repeat, to tell over
and over. If we were to call this period \emph{reiteration}, the
designation would hardly bring out more clearly than does the word
\emph{recitation}, the complete domination of instruction by rehearsing
of secondhand information, by memorizing for the sake of producing
correct replies at the proper time. Everything that is said in this
chapter is insignificant in comparison with the primary truth that the
recitation is a place and time for stimulating and directing reflection,
and that reproducing
memorized
matter is only an incident---even though an indispensable incident---in
the process of cultivating a thoughtful attitude.

\section{The Formal Steps of Instruction}

\marginpar{Herbart's analysis of method of teaching}

But few attempts have been made to formulate a method, resting on
general principles, of conducting a recitation. One of these is of great
importance and has probably had more and better influence upon the
"hearing of lessons" than all others put together; namely, the analysis
by Herbart of a recitation into five successive steps. The steps are
commonly known as "the formal steps of instruction." The underlying
notion is that no matter how subjects vary in scope and detail there is
one and only one best way of mastering them, since there is a single
"general method" uniformly followed by the mind in effective attack upon
any subject. Whether it be a first-grade child mastering the rudiments
of number, a grammar-school pupil studying history, or a college student
dealing with philology, in each case the first step is preparation, the
second presentation, followed in turn by comparison and generalization,
ending in the application of the generalizations to specific and new
instances.

\marginpar{Illustration of method}

By preparation is meant asking questions to remind pupils of familiar
experiences of their own that will be useful in acquiring the new topic.
What one already knows supplies the means with which one apprehends the
unknown. Hence the process of learning the new will be made easier if
related ideas in the pupil's mind are aroused to activity---are brought
to the foreground of consciousness. When pupils take up the study of
rivers, they are first questioned about streams or
brooks
with which they are already acquainted; if they have never seen any,
they may be asked about water running in gutters. Somehow "apperceptive
masses" are stirred that will assist in getting hold of the new subject.
The step of preparation ends with statement of the aim of the lesson.
Old knowledge having been made active, new material is then "presented"
to the pupils. Pictures and relief models of rivers are shown; vivid
oral descriptions are given; if possible, the children are taken to see
an actual river. These two steps terminate the acquisition of particular
facts.

The next two steps are directed toward getting a general principle or
conception. The local river is compared with, perhaps, the Amazon, the
St. Lawrence, the Rhine; by this comparison accidental and unessential
features are eliminated and the river \emph{concept} is formed: the
elements involved in the river-meaning are gathered together and
formulated. This done, the resulting principle is fixed in mind and is
clarified by being applied to other streams, say to the Thames, the Po,
the Connecticut.

\marginpar{Comparison with our prior analysis of reflection}

If we compare this account of the methods of instruction with our own
analysis of a complete operation of thinking, we are struck by obvious
resemblances. In our statement (compare
\protect\hyperlink{ux40publicux40vhostux40gux40gutenbergux40htmlux40filesux4037423ux4037423-hux4037423-h-2.htm.htmlux5cux23CHAPTER_SIX}{Chapter
Six}) the "steps" are the occurrence of a problem or a puzzling
phenomenon; then observation, inspection of facts, to locate and clear
up the problem; then the formation of a hypothesis or the suggestion of
a possible solution together with its elaboration by reasoning; then the
testing of the elaborated idea by using it as a guide to new
observations and experimentations. In each account, there is the
sequence of (\emph{i}) specific facts
and
events, (\emph{ii}) ideas and reasonings, and (\emph{iii}) application
of their result to specific facts. In each case, the movement is
inductive-deductive. We are struck also by one difference: the
Herbartian method makes no reference to a difficulty, a discrepancy
requiring explanation, as the origin and stimulus of the whole process.
As a consequence, it often seems as if the Herbartian method deals with
thought simply as an incident in the process of acquiring information,
instead of treating the latter as an incident in the process of
developing thought.

The formal steps concern the teacher's preparation rather than the
recitation itself

Before following up this comparison in more detail, we may raise the
question whether the recitation should, in any case, follow a uniform
prescribed series of steps---even if it be admitted that this series
expresses the normal logical order. In reply, it may be said that just
because the order is logical, it represents the survey of subject-matter
made by one who already understands it, not the path of progress
followed by a mind that is learning. The former may describe a uniform
straight-way course, the latter must be a series of tacks, of zigzag
movements back and forth. In short, the formal steps indicate the points
that should be covered by the teacher in preparing to conduct a
recitation, but should not prescribe the actual course of teaching.

\marginpar{The teacher's problem}

Lack of any preparation on the part of a teacher leads, of course, to a
random, haphazard recitation, its success depending on the inspiration
of the moment, which may or may not come. Preparation in simply the
subject-matter conduces to a rigid order, the teacher examining pupils
on their exact knowledge of their text. But the teacher's problem---as a
teacher---does not reside in mastering a subject-matter, but in
adjusting a subject-matter to the nurture of thought. Now
the
formal steps indicate excellently well the questions a teacher should
ask in working out the problem of teaching a topic. What preparation
have my pupils for attacking this subject? What familiar experiences of
theirs are available? What have they already learned that will come to
their assistance? How shall I present the matter so as to fit
economically and effectively into their present equipment? What pictures
shall I show? To what objects shall I call their attention? What
incidents shall I relate? What comparisons shall I lead them to draw,
what similarities to recognize? What is the general principle toward
which the whole discussion should point as its conclusion? By what
applications shall I try to fix, to clear up, and to make real their
grasp of this general principle? What activities of their own may bring
it home to them as a genuinely significant principle?

\marginpar{Only flexibility of procedure gives a recitation vitality}

\marginpar{Any step may come first}

No teacher can fail to teach better if he has considered such questions
somewhat systematically. But the more the teacher has reflected upon
pupils' probable intellectual response to a topic from the various
stand-points indicated by the five formal steps, the more he will be
prepared to conduct the recitation in a flexible and free way, and yet
not let the subject go to pieces and the pupils' attention drift in all
directions; the less necessary will he find it, in order to preserve a
semblance of intellectual order, to follow some one uniform scheme. He
will be ready to take advantage of any sign of vital response that shows
itself from any direction. One pupil may already have some
inkling---probably erroneous---of a general principle. Application may
then come at the very beginning in order to show that the principle will
not work, and
thereby
induce search for new facts and a new generalization. Or the abrupt
presentation of some fact or object may so stimulate the minds of pupils
as to render quite superfluous any preliminary preparation. If pupils'
minds are at work at all, it is quite impossible that they should wait
until the teacher has conscientiously taken them through the steps of
preparation, presentation, and comparison before they form at least a
working hypothesis or generalization. Moreover, unless comparison of the
familiar and the unfamiliar is introduced at the beginning, both
preparation and presentation will be aimless and without logical motive,
isolated, and in so far meaningless. The student's mind cannot be
prepared at large, but only for something in particular, and
presentation is usually the best way of evoking associations. The
emphasis may fall now on the familiar concept that will help grasp the
new, now on the new facts that frame the problem; but in either case it
is comparison and contrast with the other term of the pair which gives
either its force. In short, to transfer the logical steps from the
points that the teacher needs to consider to uniform successive steps in
the conduct of a recitation, is to impose the logical review of a mind
that already understands the subject, upon the mind that is struggling
to comprehend it, and thereby to obstruct the logic of the student's own
mind.

\section{The Factors in the Recitation}

Bearing in mind that the formal steps represent intertwined factors of a
student's progress and not mileposts on a beaten highway, we may
consider each by itself. In so doing, it will be convenient to follow
the example of many of the Herbartians and reduce the steps
to
three: first, the apprehension of specific or particular facts; second,
rational generalization; third, application and verification.

\marginpar{Preparation is getting the sense of a problem}

I. The processes having to do with particular facts are preparation and
presentation. The best, indeed the only preparation is arousal to a
perception of something that needs explanation, something unexpected,
puzzling, peculiar. When the feeling of a genuine perplexity lays hold
of any mind (no matter how the feeling arises), that mind is alert and
inquiring, because stimulated from within. The shock, the bite, of a
question will force the mind to go wherever it is capable of going,
better than will the most ingenious pedagogical devices unaccompanied by
this mental ardor. It is the sense of a problem that forces the mind to
a survey and recall of the past to discover what the question means and
how it may be dealt with.

\marginpar{Pitfalls in preparation}

The teacher in his more deliberate attempts to call into play the
familiar elements in a student's experience, must guard against certain
dangers. (\emph{i}) The step of preparation must not be too long
continued or too exhaustive, or it defeats its own end. The pupil loses
interest and is bored, when a plunge \emph{in medias res} might have
braced him to his work. The preparation part of the recitation period of
some conscientious teachers reminds one of the boy who takes so long a
run in order to gain headway for a jump that when he reaches the line,
he is too tired to jump far. (\emph{ii}) The organs by which we
apprehend new material are our habits. To insist too minutely upon
turning over habitual dispositions into conscious ideas is to interfere
with their best workings. Some factors of familiar experience must
indeed be brought to conscious recognition, just as
transplanting
is necessary for the best growth of some plants. But it is fatal to be
forever digging up either experiences or plants to see how they are
getting along. Constraint, self-consciousness, embarrassment, are the
consequence of too much conscious refurbishing of familiar experiences.

\marginpar{Statement of aim of lesson}

Strict Herbartians generally lay it down that statement---by the
teacher---of the aim of a lesson is an indispensable part of
preparation. This preliminary statement of the aim of the lesson hardly
seems more intellectual in character, however, than tapping a bell or
giving any other signal for attention and transfer of thoughts from
diverting subjects. To the teacher the statement of an end is
significant, because he has already been at the end; from a pupil's
standpoint the statement of what he is \emph{going} to learn is
something of an Irish bull. If the statement of the aim is taken too
seriously by the instructor, as meaning more than a signal to attention,
its probable result is forestalling the pupil's own reaction, relieving
him of the responsibility of developing a problem and thus arresting his
mental initiative.

\marginpar{How much the teacher should tell or show}

It is unnecessary to discuss at length presentation as a factor in the
recitation, because our last chapter covered the topic under the
captions of observation and communication. The function of presentation
is to supply materials that force home the nature of a problem and
furnish suggestions for dealing with it. The practical problem of the
teacher is to preserve a balance between so little showing and telling
as to fail to stimulate reflection and so much as to choke thought.
Provided the student is genuinely engaged upon a topic, and provided the
teacher is willing to give the student a good deal of leeway as to what
he assimilates and retains (not requiring rigidly that everything be
grasped or
reproduced),
there is comparatively little danger that one who is himself
enthusiastic will communicate too much concerning a topic.

\marginpar{The pupil's responsibility for making out a reasonable case}

II. The distinctively rational phase of reflective inquiry consists, as
we have already seen, in the elaboration of an idea, or working
hypothesis, through conjoint comparison and contrast, terminating in
definition or formulation. (\emph{i}) So far as the recitation is
concerned, the primary requirement is that the student be held
responsible for working out mentally every suggested principle so as to
show what he means by it, how it bears upon the facts at hand, and how
the facts bear upon it. Unless the pupil is made responsible for
developing on his own account the \emph{reasonableness} of the guess he
puts forth, the recitation counts for practically nothing in the
training of reasoning power. A clever teacher easily acquires great
skill in dropping out the inept and senseless contributions of pupils,
and in selecting and emphasizing those in line with the result he wishes
to reach. But this method (sometimes called "suggestive questioning")
relieves the pupils of intellectual responsibility, save for acrobatic
agility in following the teacher's lead.

\marginpar{The necessity for mental leisure}

(\emph{ii}) The working over of a vague and more or less casual idea
into coherent and definite form is impossible without a pause, without
freedom from distraction. We say "Stop and think"; well, all reflection
involves, at some point, stopping external observations and reactions so
that an idea may mature. Meditation, withdrawal or abstraction from
clamorous assailants of the senses and from demands for overt action, is
as necessary at the reasoning stage, as are observation and experiment
at other periods. The metaphors of digestion
and
assimilation, that so readily occur to mind in connection with rational
elaboration, are highly instructive. A silent, uninterrupted
working-over of considerations by comparing and weighing alternative
suggestions, is indispensable for the development of coherent and
compact conclusions. Reasoning is no more akin to disputing or arguing,
or to the abrupt seizing and dropping of suggestions, than digestion is
to a noisy champing of the jaws. The teacher must secure opportunity for
leisurely mental digestion.

\marginpar{A typical central object necessary}

(\emph{iii}) In the process of comparison, the teacher must avert the
distraction that ensues from putting before the mind a number of facts
on the same level of importance. Since attention is selective, some one
object normally claims thought and furnishes the center of departure and
reference. This fact is fatal to the success of the pedagogical methods
that endeavor to conduct comparison on the basis of putting before the
mind a row of objects of equal importance. In comparing, the mind does
not naturally begin with objects \emph{a}, \emph{b}, \emph{c}, \emph{d},
and try to find the respect in which they agree. It begins with a single
object or situation more or less vague and inchoate in meaning, and
makes excursions to other objects in order to render understanding of
the central object consistent and clear. The mere multiplication of
objects of comparison is adverse to successful reasoning. Each fact
brought within the field of comparison should clear up some obscure
feature or extend some fragmentary trait of the primary object.

\marginpar{Importance of types}

In short, pains should be taken to see that the object on which thought
centers is \emph{typical}: material being typical when, although
individual or specific, it is such as readily and fruitfully suggests
the principles of an
entire
class of facts. No sane person begins to think about rivers wholesale or
at large. He begins with the one river that has presented some puzzling
trait. Then he studies other rivers to get light upon the baffling
features of this one, and at the same time he employs the characteristic
traits of his original object to reduce to order the multifarious
details that appear in connection with other rivers. This working back
and forth preserves unity of meaning, while protecting it from monotony
and narrowness. Contrast, unlikeness, throws significant features into
relief, and these become instruments for binding together into an
organized or coherent meaning dissimilar characters. The mind is
defended against the deadening influence of many isolated particulars
and also against the barrenness of a merely formal principle. Particular
cases and properties supply emphasis and concreteness; general
principles convert the particulars into a single system.

\marginpar{All insight into meaning effects generalization}

(\emph{iv}) Hence generalization is not a separate and single act; it is
rather a constant tendency and function of the entire discussion or
recitation. Every step forward toward an idea that comprehends, that
explains, that unites what was isolated and therefore puzzling,
generalizes. The little child generalizes as truly as the adolescent or
adult, even though he does not arrive at the same generalities. If he is
studying a river basin, his knowledge is generalized in so far as the
various details that he apprehends are found to be the effects of a
single force, as that of water pushing downward from gravity, or are
seen to be successive stages of a single history of formation. Even if
there were acquaintance with only one river, knowledge of it under such
conditions would be generalized
knowledge.

\marginpar{Insight into meaning requires formulation}

The factor of formulation, of conscious stating, involved in
generalization, should also be a constant function, not a single formal
act. Definition means essentially the growth of a meaning out of
vagueness into \emph{definiteness}. Such final verbal definition as
takes place should be only the culmination of a steady growth in
distinctness. In the reaction against ready-made verbal definitions and
rules, the pendulum should never swing to the opposite extreme, that of
neglecting to summarize the net meaning that emerges from dealing with
particular facts. Only as general summaries are made from time to time
does the mind reach a conclusion or a resting place; and only as
conclusions are reached is there an intellectual deposit available in
future understanding.

\marginpar{Generalization means capacity for application to the new}

III. As the last words indicate, application and generalization lie
close together. Mechanical skill for further use may be achieved without
any explicit recognition of a principle; nay, in routine and narrow
technical matters, conscious formulation may be a hindrance. But without
recognition of a principle, without generalization, the power gained
cannot be transferred to new and dissimilar matters. The inherent
significance of generalization is that it frees a meaning from local
restrictions; rather, generalization \emph{is} meaning so freed; it is
meaning emancipated from accidental features so as to be available in
new cases. The surest test for detecting a spurious generalization (a
statement general in verbal form but not accompanied by discernment of
meaning), is the failure of the so-called principle spontaneously to
extend itself. The essence of the general is application. (\emph{Ante},
p. 29.)

\marginpar{Fossilized \emph{versus} flexible principles}

The true purpose of exercises that apply rules and principles is, then,
not so much to drive or drill
them
in as to give adequate insight into an idea or principle. To treat
application as a separate final step is disastrous. In every judgment
some meaning is employed as a basis for estimating and interpreting some
fact; by this application the meaning is itself enlarged and tested.
When the general meaning is regarded as complete in itself, application
is treated as an external, non-intellectual use to which, for practical
purposes alone, it is advisable to put the meaning. The principle is one
self-contained thing; its use is another and independent thing. When
this divorce occurs, principles become fossilized and rigid; they lose
their inherent vitality, their self-impelling power.

\marginpar{Self-application a mark of genuine principles}

A true conception is a \emph{moving} idea, and it seeks outlet, or
application to the interpretation of particulars and the guidance of
action, as naturally as water runs downhill. In fine, just as reflective
thought requires particular facts of observation and events of action
for its origination, so it also requires particular facts and deeds for
its own consummation. "Glittering generalities" are inert because they
are spurious. Application is as much an intrinsic part of genuine
reflective inquiry as is alert observation or reasoning itself. Truly
general principles tend to apply themselves. The teacher needs, indeed,
to supply conditions favorable to use and exercise; but something is
wrong when artificial tasks have arbitrarily to be invented in order to
secure application for
principles.

\chapter{Some General Conclusions}

We shall conclude our survey of how we think and how we should think by
presenting some factors of thinking which should balance each other, but
which constantly tend to become so isolated that they work against each
other instead of cooperating to make reflective inquiry efficient.

\section{The Unconscious and the Conscious}

\marginpar{The \emph{understood} as the unconsciously assumed}

It is significant that one meaning of the term \emph{understood} is
something so thoroughly mastered, so completely agreed upon, as to be
\emph{assumed}; that is to say, taken as a matter of course without
explicit statement. The familiar "goes without saying" means "it is
understood." If two persons can converse intelligently with each other,
it is because a common experience supplies a background of mutual
understanding upon which their respective remarks are projected. To dig
up and to formulate this common background would be imbecile; it is
"understood"; that is, it is silently supplied and implied as the
taken-for-granted medium of intelligent exchange of ideas.

\marginpar{Inquiry as conscious formulation}

If, however, the two persons find themselves at cross-purposes, it is
necessary to dig up and compare the presuppositions, the implied
context, on the basis of which each is speaking. The implicit is made
explicit; what was unconsciously assumed is exposed to the light of
conscious day. In this way, the root of the
misunderstanding
is removed. Some such rhythm of the unconscious and the conscious is
involved in all fruitful thinking. A person in pursuing a consecutive
train of thoughts takes some system of ideas for granted (which
accordingly he leaves unexpressed, "unconscious") as surely as he does
in conversing with others. Some context, some situation, some
controlling purpose dominates his explicit ideas so thoroughly that it
does not need to be consciously formulated and expounded. Explicit
thinking goes on within the limits of what is implied or understood. Yet
the fact that reflection originates in a problem makes it necessary
\emph{at some points} consciously to inspect and examine this familiar
background. We have to turn upon some unconscious assumption and make it
explicit.

\marginpar{Rules cannot be given for attaining a balance}

No rules can be laid down for attaining the due balance and rhythm of
these two phases of mental life. No ordinance can prescribe at just what
point the spontaneous working of some unconscious attitude and habit is
to be checked till we have made explicit what is implied in it. No one
can tell in detail just how far the analytic inspection and formulation
are to be carried. We can say that they must be carried far enough so
that the individual will know what he is about and be able to guide his
thinking; but in a given case just how far is that? We can say that they
must be carried far enough to detect and guard against the source of
some false perception or reasoning, and to get a leverage on the
investigation; but such statements only restate the original difficulty.
Since our reliance must be upon the disposition and tact of the
individual in the particular case, there is no test of the success of an
education more important than the extent to which it nurtures a type of
mind competent
to
maintain an economical balance of the unconscious and the conscious.

\marginpar{The over-\emph{analytic} to be avoided}

The ways of teaching criticised in the foregoing pages as false
"analytic" methods of instruction (\emph{ante}, p. 112), all reduce
themselves to the mistake of directing explicit attention and
formulation to what would work better if left an unconscious attitude
and working assumption. To pry into the familiar, the usual, the
automatic, simply for the sake of making it conscious, simply for the
sake of formulating it, is both an impertinent interference, and a
source of boredom. To be forced to dwell consciously upon the accustomed
is the essence of ennui; to pursue methods of instruction that have that
tendency is deliberately to cultivate lack of interest.

\marginpar{The detection of error, the clinching of truth, demand conscious
statement}

On the other hand, what has been said in criticism of merely routine
forms of skill, what has been said about the importance of having a
genuine problem, of introducing the novel, and of reaching a deposit of
general meaning weighs on the other side of the scales. It is as fatal
to good thinking to fail to make conscious the standing source of some
error or failure as it is to pry needlessly into what works smoothly. To
over-simplify, to exclude the novel for the sake of prompt skill, to
avoid obstacles for the sake of averting errors, is as detrimental as to
try to get pupils to formulate everything they know and to state every
step of the process employed in getting a result. Where the shoe
pinches, analytic examination is indicated. When a topic is to be
clinched so that knowledge of it will carry over into an effective
resource in further topics, conscious condensation and summarizing are
imperative. In the early stage of acquaintance with a subject, a good
deal of unconstrained unconscious mental play about it may
be
permitted, even at the risk of some random experimenting; in the later
stages, conscious formulation and review may be encouraged. Projection
and reflection, going directly ahead and turning back in scrutiny,
should alternate. Unconsciousness gives spontaneity and freshness;
consciousness, conviction and control.

\section{Process and Product}

\marginpar{Play and work again}

A like balance in mental life characterizes process and product. We met
one important phase of this adjustment in considering play and work. In
play, interest centers in activity, without much reference to its
outcome. The sequence of deeds, images, emotions, suffices on its own
account. In work, the end holds attention and controls the notice given
to means. Since the difference is one of direction of interest, the
contrast is one of emphasis, not of cleavage. When comparative
prominence in consciousness of activity or outcome is transformed into
isolation of one from the other, play degenerates into fooling, and work
into drudgery.

\marginpar{Play should not be fooling,}

By "fooling" we understand a series of disconnected temporary overflows
of energy dependent upon whim and accident. When all reference to
outcome is eliminated from the sequence of ideas and acts that make
play, each member of the sequence is cut loose from every other and
becomes fantastic, arbitrary, aimless; mere fooling follows. There is
some inveterate tendency to fool in children as well as in animals; nor
is the tendency wholly evil, for at least it militates against falling
into ruts. But when it is excessive in amount, dissipation and
disintegration follow; and the only way of preventing this consequence
is to make regard for results enter into even the freest play
activity.

\marginpar{nor work, drudgery}

Exclusive interest in the result alters work to drudgery. For by
drudgery is meant those activities in which the interest in the outcome
does not suffuse the means of getting the result. Whenever a piece of
work becomes drudgery, the process of doing loses all value for the
doer; he cares solely for what is to be had at the end of it. The work
itself, the putting forth of energy, is hateful; it is just a necessary
evil, since without it some important end would be missed. Now it is a
commonplace that in the work of the world many things have to be done
the doing of which is not intrinsically very interesting. However, the
argument that children should be kept doing drudgery-tasks because
thereby they acquire power to be faithful to distasteful duties, is
wholly fallacious. Repulsion, shirking, and evasion are the consequences
of having the repulsive imposed---not loyal love of duty. Willingness to
work for ends by means of acts not naturally attractive is best attained
by securing such an appreciation of the value of the end that a sense of
its value is transferred to its means of accomplishment. Not interesting
in themselves, they borrow interest from the result with which they are
associated.

\marginpar{Balance of playfulness and seriousness the intellectual ideal}

\marginpar{Free play of mind}

\marginpar{is normal in childhood}

The intellectual harm accruing from divorce of work and play, product
and process, is evidenced in the proverb, "All work and no play makes
Jack a dull boy." That the obverse is true is perhaps sufficiently
signalized in the fact that fooling is so near to foolishness. To be
playful and serious at the same time is possible, and it defines the
ideal mental condition. Absence of dogmatism and prejudice, presence of
intellectual curiosity and flexibility, are manifest in the free play of
the mind upon a topic. To give the mind
this
free play is not to encourage toying with a subject, but is to be
interested in the unfolding of the subject on its own account, apart
from its subservience to a preconceived belief or habitual aim. Mental
play is open-mindedness, faith in the power of thought to preserve its
own integrity without external supports and arbitrary restrictions.
Hence free mental play involves seriousness, the earnest following of
the development of subject-matter. It is incompatible with carelessness
or flippancy, for it exacts accurate noting of every result reached in
order that every conclusion may be put to further use. What is termed
the interest in truth for its own sake is certainly a serious matter,
yet this pure interest in truth coincides with love of the free play of
thought.

In spite of many appearances to the contrary---usually due to social
conditions of either undue superfluity that induces idle fooling or
undue economic pressure that compels drudgery---childhood normally
realizes the ideal of conjoint free mental play and thoughtfulness.
Successful portrayals of children have always made their wistful
intentness at least as obvious as their lack of worry for the morrow. To
live in the present is compatible with condensation of far-reaching
meanings in the present. Such enrichment of the present for its own sake
is the just heritage of childhood and the best insurer of future growth.
The child forced into premature concern with economic remote results may
develop a surprising sharpening of wits in a particular direction, but
this precocious specialization is always paid for by later apathy and
dullness.

\marginpar{The attitude of the artist}

That art originated in play is a common saying. Whether or not the
saying is historically correct,
it
suggests that harmony of mental playfulness and seriousness describes
the artistic ideal. When the artist is preoccupied overmuch with means
and materials, he may achieve wonderful technique, but not the artistic
spirit \emph{par excellence}. When the animating idea is in excess of
the command of method, æsthetic feeling may be indicated, but the art of
presentation is too defective to express the feeling thoroughly. When
the thought of the end becomes so adequate that it compels translation
into the means that embody it, or when attention to means is inspired by
recognition of the end they serve, we have the attitude typical of the
artist, an attitude that may be displayed in all activities, even though
not conventionally designated arts.

\marginpar{The art of the teacher culminates in nurturing this attitude}

That teaching is an art and the true teacher an artist is a familiar
saying. Now the teacher's own claim to rank as an artist is measured by
his ability to foster the attitude of the artist in those who study with
him, whether they be youth or little children. Some succeed in arousing
enthusiasm, in communicating large ideas, in evoking energy. So far,
well; but the final test is whether the stimulus thus given to wider
aims succeeds in transforming itself into power, that is to say, into
the attention to detail that ensures mastery over means of execution. If
not, the zeal flags, the interest dies out, the ideal becomes a clouded
memory. Other teachers succeed in training facility, skill, mastery of
the technique of subjects. Again it is well---so far. But unless
enlargement of mental vision, power of increased discrimination of final
values, a sense for ideas---for principles---accompanies this training,
forms of skill ready to be put indifferently to any end may be the
result. Such modes of technical skill may display themselves,
according
to circumstances, as cleverness in serving self-interest, as docility in
carrying out the purposes of others, or as unimaginative plodding in
ruts. To nurture inspiring aim and executive means into harmony with
each other is at once the difficulty and the reward of the teacher.

\section{The Far and the Near}

\marginpar{"Familiarity breeds contempt,"}

Teachers who have heard that they should avoid matters foreign to
pupils' experience, are frequently surprised to find pupils wake up when
something beyond their ken is introduced, while they remain apathetic in
considering the familiar. In geography, the child upon the plains seems
perversely irresponsive to the intellectual charms of his local
environment, and fascinated by whatever concerns mountains or the sea.
Teachers who have struggled with little avail to extract from pupils
essays describing the details of things with which they are well
acquainted, sometimes find them eager to write on lofty or imaginary
themes. A woman of education, who has recorded her experience as a
factory worker, tried retelling \emph{Little Women} to some factory
girls during their working hours. They cared little for it, saying,
"Those girls had no more interesting experience than we have," and
demanded stories of millionaires and society leaders. A man interested
in the mental condition of those engaged in routine labor asked a Scotch
girl in a cotton factory what she thought about all day. She replied
that as soon as her mind was free from starting the machinery, she
married a duke, and their fortunes occupied her for the remainder of the
day.

\marginpar{since only the novel demands attention,}

Naturally, these incidents are not told in order to encourage methods of
teaching that appeal to the
sensational,
the extraordinary, or the incomprehensible. They are told, however, to
enforce the point that the familiar and the near do not excite or repay
thought on their own account, but only as they are adjusted to mastering
the strange and remote. It is a commonplace of psychology that we do not
attend to the old, nor consciously mind that to which we are thoroughly
accustomed. For this, there is good reason: to devote attention to the
old, when new circumstances are constantly arising to which we should
adjust ourselves, would be wasteful and dangerous. Thought must be
reserved for the new, the precarious, the problematic. Hence the mental
constraint, the sense of being lost, that comes to pupils when they are
invited to turn their thoughts upon that with which they are already
familiar. The old, the near, the accustomed, is not that \emph{to} which
but that \emph{with} which we attend; it does not furnish the material
of a problem, but of its solution.

\marginpar{which, in turn, can be given only through the old}

The last sentence has brought us to the balancing of new and old, of the
far and that close by, involved in reflection. The more remote supplies
the stimulus and the motive; the nearer at hand furnishes the point of
approach and the available resources. This principle may also be stated
in this form: the best thinking occurs when the easy and the difficult
are duly proportioned to each other. The easy and the familiar are
equivalents, as are the strange and the difficult. Too much that is easy
gives no ground for inquiry; too much of the hard renders inquiry
hopeless.

\marginpar{The given and the suggested}

The necessity of the interaction of the near and the far follows
directly from the nature of thinking. Where there is thought, something
present suggests and indicates something absent. Accordingly unless the
familiar
is presented under conditions that are in some respect unusual, it gives
no jog to thinking, it makes no demand upon what is not present in order
to be understood. And if the subject presented is totally strange, there
is no basis upon which it may suggest anything serviceable for its
comprehension. When a person first has to do with fractions, for
example, they will be wholly baffling so far as they do not signify to
him some relation that he has already mastered in dealing with whole
numbers. When fractions have become thoroughly familiar, his perception
of them acts simply as a signal to do certain things; they are a
"substitute sign," to which he can react without thinking. (\emph{Ante},
p. 178.) If, nevertheless, the situation as a whole presents something
novel and hence uncertain, the entire response is not mechanical,
because this mechanical operation is put to use in solving a problem.
There is no end to this spiral process: foreign subject-matter
transformed through thinking into a familiar possession becomes a
resource for judging and assimilating additional foreign subject-matter.

\marginpar{Observation supplies the near, imagination the remote}

The need for both imagination and observation in every mental enterprise
illustrates another aspect of the same principle. Teachers who have
tried object-lessons of the conventional type have usually found that
when the lessons were new, pupils were attracted to them as a diversion,
but as soon as they became matters of course they were as dull and
wearisome as was ever the most mechanical study of mere symbols.
Imagination could not play about the objects so as to enrich them. The
feeling that instruction in "facts, facts" produces a narrow Gradgrind
is justified not because facts in themselves are limiting, but because
facts are dealt
out
as such hard and fast ready-made articles as to leave no room to
imagination. Let the facts be presented so as to stimulate imagination,
and culture ensues naturally enough. The converse is equally true. The
imaginative is not necessarily the imaginary; that is, the unreal. The
proper function of imagination is vision of realities that cannot be
exhibited under existing conditions of sense-perception. Clear insight
into the remote, the absent, the obscure is its aim. History,
literature, and geography, the principles of science, nay, even geometry
and arithmetic, are full of matters that must be imaginatively realized
if they are realized at all. Imagination supplements and deepens
observation; only when it turns into the fanciful does it become a
substitute for observation and lose logical force.

\marginpar{Experience through communication of others' experience}

A final exemplification of the required balance between near and far is
found in the relation that obtains between the narrower field of
experience realized in an individual's own contact with persons and
things, and the wider experience of the race that may become his through
communication. Instruction always runs the risk of swamping the pupil's
own vital, though narrow, experience under masses of communicated
material. The instructor ceases and the teacher begins at the point
where communicated matter stimulates into fuller and more significant
life that which has entered by the strait and narrow gate of
sense-perception and motor activity. Genuine communication involves
contagion; its name should not be taken in vain by terming communication
that which produces no community of thought and purpose between the
child and the race of which he is the
heir.

\end{document}
